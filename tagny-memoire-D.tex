\documentclass[12pt,a4paper,openright]{book}
\usepackage[T1]{fontenc}
\usepackage[utf8]{inputenc}
\usepackage[square]{natbib}
\usepackage[french]{babel}
\bibliographystyle{authordate2}
%\bibliographystyle{plain}
\usepackage{amsmath}
\usepackage{mathtools}
\usepackage{amsfonts}
\usepackage{amssymb}
\usepackage{xcolor, graphicx}
\usepackage{eurosym}
\graphicspath{ {gfx/} }
\usepackage{mathpazo}
\usepackage{fancyhdr}
\usepackage{color}
\definecolor{gray}{rgb}{0.4,0.4,0.4}
\definecolor{darkblue}{rgb}{0.0,0.0,0.6}
\definecolor{cyan}{rgb}{0.0,0.6,0.6}
\usepackage{listings}
\lstset{
	basicstyle=\footnotesize,
	breaklines=true,
	showstringspaces=false,
	inputencoding=utf8,
	extendedchars=false,
	literate=%
	 {á}{{\'a}}1 {é}{{\'e}}1 {í}{{\'i}}1 {ó}{{\'o}}1 {ú}{{\'u}}1
	{Á}{{\'A}}1 {É}{{\'E}}1 {Í}{{\'I}}1 {Ó}{{\'O}}1 {Ú}{{\'U}}1
	{à}{{\`a}}1 {è}{{\`e}}1 {ì}{{\`i}}1 {ò}{{\`o}}1 {ù}{{\`u}}1    
	{À}{{\`A}}1 {È}{{\'E}}1 {Ì}{{\`I}}1 {Ò}{{\`O}}1 {Ù}{{\`U}}1
	{ä}{{\"a}}1 {ë}{{\"e}}1 {ï}{{\"i}}1 {ö}{{\"o}}1 {ü}{{\"u}}1
	{Ä}{{\"A}}1 {Ë}{{\"E}}1 {Ï}{{\"I}}1 {Ö}{{\"O}}1 {Ü}{{\"U}}1
	{â}{{\^a}}1 {ê}{{\^e}}1 {î}{{\^i}}1 {ô}{{\^o}}1 {û}{{\^u}}1
	{Â}{{\^A}}1 {Ê}{{\^E}}1 {Î}{{\^I}}1 {Ô}{{\^O}}1 {Û}{{\^U}}1
	{Ã}{{\~A}}1 {ã}{{\~a}}1 {Õ}{{\~O}}1 {õ}{{\~o}}1
	{œ}{{\oe}}1 {Œ}{{\OE}}1 {æ}{{\ae}}1 {Æ}{{\AE}}1 {ß}{{\ss}}1
	{ű}{{\H{u}}}1 {Ű}{{\H{U}}}1 {ő}{{\H{o}}}1 {Ő}{{\H{O}}}1
	{ç}{{\c c}}1 {Ç}{{\c C}}1 {ø}{{\o}}1 {å}{{\r a}}1 {Å}{{\r A}}1
	{€}{{\euro}}1 {£}{{\pounds}}1 {«}{{\guillemotleft}}1
	{»}{{\guillemotright}}1 {ñ}{{\~n}}1 {Ñ}{{\~N}}1 {¿}{{?`}}1 {°}{{$^\circ$}}1 
}

\lstdefinelanguage{XML}
{
	morestring=[b]",
	morestring=[s]{>}{<},
	morecomment=[s]{<?}{?>},
	stringstyle=\color{black},
	identifierstyle=\color{darkblue},
	keywordstyle=\color{cyan},
	morekeywords={xmlns,version,type}% list your attributes here
}

\usepackage{setspace}
\usepackage{nomencl}
\usepackage[titles]{tocloft}
\usepackage{titlesec}
\usepackage{array,multirow}
\usepackage{longtable}
%\usepackage{subfigure}
\usepackage[font=small,singlelinecheck=off,justification=centering]{caption}
\usepackage[font=footnotesize]{subcaption}
\usepackage{fancyvrb}
\usepackage{booktabs}% http://ctan.org/pkg/booktabs
\newcommand{\tabitem}{~~\llap{\textbullet}~~}
%\usepackage[left=3cm,right=3cm,top=2.5cm,bottom=2.5cm]{geometry}%
%\usepackage[margin=1.2in]{geometry}
%\usepackage{geometry}
\usepackage[titlenotnumbered,ruled,vlined,noend,linesnumbered,french, onelanguage]{algorithm2e}
%\usepackage{nomencl}
\usepackage{setspace}
\usepackage[toc,page]{appendix}
\usepackage{nameref}
\usepackage{mathrsfs}
\usepackage{float} 
\usepackage{bbding}
\usepackage{pstricks} %  [usenames,dvipsnames]
\usepackage{epsfig}
\usepackage{pst-grad} % For gradients
\usepackage{pst-plot} % For axes
%\usepackage{underscore} % for undescores in the bib file
\usepackage[hidelinks]{hyperref} % rule "load hyperref as the last package" 
\usepackage[shortlabels]{enumitem}
\setlist[itemize]{label=\textbullet}

\newcommand{\titlefr}{Analyse sémantique d'un corpus exhaustif de décisions jurisprudentielles}
\newcommand{\titleen}{A semantic analysis of a comprehensive corpus of judicial decisions}

\author{Gildas TAGNY NGOMPE}
\title{\titlefr}

\titleformat{\chapter}{\Huge\bfseries}{\chaptername\ \thechapter}{0pt}{\vskip 20pt\raggedright}%
\titlespacing{\chapter}{0pt}{10pt}{30pt}%

%{\normalfont\huge\bfseries}{\chaptertitlename\ \thechapter}{20pt}{\Huge}   
%\titlespacing*{\chapter}{0pt}{-50pt}{40pt}

\makeatletter

\newcommand{\twodots}{\mathrel{{.}\,{.}}\nobreak}
\newcommand{\avg}{\mathop{\mathrm{avg}}\limits} % average
\renewcommand*{\p@section}{\S\,}
\renewcommand*{\p@subsection}{\S\,}
\renewcommand*{\p@subsubsection}{\S\,}
\makeatother
\renewcommand{\bibname}{Bibliographie}
\newlength{\larg}
\setlength{\larg}{16cm}
\setlength{\cftchapnumwidth}{2.5cm}
\setlength{\cftbeforechapskip}{0.25cm}
\setlength{\cftbeforepartskip}{0.4cm}
\setlength{\cftbeforesecskip}{0.2cm}
%\renewcommand{\cftpartpresnum}{Partie }
\renewcommand{\cftchappresnum}{\chaptertitlename \hspace{0.2cm}}
\renewcommand{\cftsecpresnum}{\hspace{0.2cm}}
%\renewcommand\citeform[1]{\bf #1}
\newcommand{\argmax}{\mathop{\mathrm{argmax}}\limits}
\renewcommand{\max}{\mathop{\mathrm{max}}\limits}
\newcommand{\argmin}{\mathop{\mathrm{argmin}}\limits}
\renewcommand{\min}{\mathop{\mathrm{min}}\limits}
\newtheorem{postulat}{Postulat}[section]
\newcommand*{\andname}{and}
      \addto \captionsenglish {\renewcommand*{\andname}{and}}
      \addto \captionsfrench  {\renewcommand*{\andname}{et}}
\newcommand{\norm}[1]{\left\lVert#1\right\rVert}
\newcommand{\setsize}[1]{{\vert#1\vert}}

\addto\captionsenglish{
	\renewcommand{\labelitemi}{$\bullet$}
	\renewcommand{\labelitemii}{$\cdot$}
	\renewcommand{\labelitemiii}{$\diamond$}
	\renewcommand{\labelitemiv}{$\ast$}
}

%%entete_pied
%\newcommand{\sujet}{Analyse Sémantique d'un Corpus Exhaustif de Décisions Jurisprudentielles}
%\newcommand{\auteur}{Gildas TAGNY NGOMPÉ}
\newcommand{\numpage}{\bf \large \thepage}
\newcommand{\institut}{IMT Mines Alès - ED 583 Risques et Sociétés}
\newcommand{\entreprise}{LGI2P / IMT Mines Alès \& EA 7352 CHROME / UNIMES  }
%\newcommand{\doc}{\footnotesize{{\bf \auteur}, {\it élève-ingénieur à l'iai} (2011-2012)}}

\hypersetup{
       %backref=true,                           % Permet d'ajouter des liens dans
       %pagebackref=true,                       % les bibliographies
       %hyperindex=true,                        % Ajoute des liens dans les index.
       colorlinks=true,                        % Colorise les liens.
       breaklinks=true,                        % Permet le retour à  la ligne dans les liens trop longs.
       urlcolor= black,                         % Couleur des hyperliens.
       linkcolor= black,                        % Couleur des liens internes.
       %bookmarks=true,                         % Crées des signets pour Acrobat.
       %bookmarksopen=true,                     % Si les signets Acrobat sont créés,
       citecolor=black,                           % les afficher complètement.
       pdftitle={\@author, M\'emoire de Thèse Doctorat en Informatique de l'IMT Mines Alès - \entreprise - \institut},  % Titre du document.
                                               % Informations apparaissant dans
       pdfauthor={\@author},                      % dans les informations du document
       pdfsubject={\@title},           % sous Acrobat.
       pdfkeywords={Fouille de textes juridiques, Annotation de document, extraction d'information}
    }

\newlength{\eptrait}
\setlength{\eptrait}{0.05mm}

\fancypagestyle{plain}{
\pagestyle{fancy}
\fancyhf{}
%\fancyfoot[L]{\it \auteur,  \og \sujet \fg{}}
\fancyhead[C]{}%\sujet}
\fancyfoot[R]{\numpage}
%\fancyhead[L]{}%\institut}
\fancyhead[R]{}%\entreprise}
}
%\fancyfoot[C]{\tiny{IAI}}
\headsep=15pt
\renewcommand{\headrulewidth}{\eptrait}
\pagestyle{fancy}
%\usefont{OT1}{ptv}{m}{n}
%\renewcommand{\headrulewidth}{1pt}
%\lfoot{\auteur, \sujet}
\rhead{}
\rfoot{\numpage}
\cfoot{}
%\lhead{}%\institut}
%\chead{\sujet}
\rhead{}%\entreprise}
\renewcommand{\footrulewidth}{\eptrait}

%\makenomenclature
%\renewcommand{\nomname}{Glossaire}

\renewcommand{\nomentryend}{.}
\renewcommand{\nomlabel}[1]{\hfil {\bf #1}\hfil:}
\addto\captionsfrench{%
%\renewcommand{\figurename}{\footnotesize {\scshape \it Figure} \it}
%\renewcommand{\tablename}{\footnotesize {\scshape \it Tableau} \it}
\renewcommand{\figurename}{Figure}
\renewcommand{\tablename}{Tableau}
\renewcommand{\listfigurename}{Liste des figures}
%\renewcommand{\contentsname}{Sommaire}
}
\newtheorem{hypothese}{Hypothèse}[section]
\newcommand{\p}{\mathbb{P}}
\newcommand{\bulle}{\textcolor[rgb]{0.00,0.00,1.00}{\bullet}}
\newenvironment{proof}[1][Proof] {\noindent \textcolor[rgb]{0.00,0.00,1.00}{\textbf{#1:}} \\ } {\ \textcolor[rgb]{0.00,0.00,1.00}{\textbf{\rule{0.5em}{0.5em}}} \bigskip}
\newcommand{\figureref}[1]{Figure \ref{#1} Page \pageref{#1}}
\newcommand{\tableref}[1]{Tableau \ref{#1} Page \pageref{#1}}
\newcommand{\R}{\mathbb{R}}
\newcommand{\N}{\mathbb{N}}
\newcommand{\E}{\mathbb{E}}
\newcommand{\D}{\mathbb{D}}
\newcommand{\cov}{\text{cov}}
\newcommand{\cor}{\text{cor}}
\newcommand{\cog}{\text{cog}}
\newcommand{\y}{\mathbf{y}}
\newcommand{\Step}{\underline{\textbf{Step}}}
\newcommand{\eps}{\varepsilon}
\newcommand{\z}{\mathbf{z}}
\newtheorem{resultat}{Résultat}
\newtheorem{condition}{Condition}
\newtheorem{proposition}{Proposition}
\newtheorem{property}{Property}
\newtheorem{conjecture}{Conjecture}
\newtheorem{corollary}{Corollary}
\newtheorem{criterion}{Criterion}
\newtheorem{stat}{Statement}
\newtheorem{axiom}{Axiom}
\newtheorem{definition}{Definition}
\newtheorem{example}{Example}
\newtheorem{lemma}{Lemma}
\newtheorem{remark}{Remarque}
\newtheorem{theorem}{Theorem}
%\input{tcilatex}
%\numberwithin{equation}{section}

\setcounter{tocdepth}{3}
\setcounter{secnumdepth}{3}

\pagestyle{fancy}
%\renewcommand{\chaptermark}[1]{\markboth{#1}{}}
\fancyhf{}
\setlength{\headheight}{15pt}
\fancyhead[LE,RO]{\thepage}
\fancyhead[LO]{\itshape\nouppercase{\rightmark}}
\fancyhead[RE]{\itshape\nouppercase{\leftmark}}

\renewcommand{\headrulewidth}{0pt}

%\pagestyle{headings}
\date{}

\begin{document}
\nocite{}
% http://www.sciencespo-lille.eu/sites/default/files/guide_preparer_et_rediger_un_memoire_de_recherche.pdf

\frontmatter
% changement du style des chapitres
%style_chapitre
\renewcommand{\chaptermark}[1]{\markboth{#1}{}}
\renewcommand*\thesection{\roman{section}}
\renewcommand*\thesubsection{\thesection.\alph{subsection}}
\makeatletter
\def\thickhrulefill{\leavevmode \leaders
\hrule height 1ex \hfill \kern \z@}
\def\@makechapterhead#1{%
\vspace*{10\p@}%
{\parindent \z@
{\reset@font
%\usefont{OT1}{phv}{b}{n}
\large \bf Chapitre \thechapter\par\nobreak}%
\par\nobreak
\vspace*{20\p@}
\interlinepenalty\@M
%\usefont{OT1}{ptm}{b}{n}
{\raggedright \Large  \bfseries #1}%
\par\nobreak
\vskip 20\p@
%\hrule height 2pt
\par\nobreak
\vskip 45\p@
}}
\def\@makeschapterhead#1{%
\vspace*{5\p@}%
{\parindent \z@
{\raggedleft \reset@font
\scshape \vphantom{\@chapapp{} \thechapter}
\par\nobreak}%
\par\nobreak
\vspace*{10\p@}
\interlinepenalty\@M
%\usefont{OT1}{ptm}{b}{n}
{\raggedright \Large \bfseries #1}%
\par\nobreak
\vskip 20\p@
\hrule height 2pt
\par\nobreak
\vskip 20\p@
}}

%%\maketitle
\includepdf[page={1}]{content/tagny-cover_IMT_Ales_ED_RS.pdf}
%\chapter*{Remerciements}
\addcontentsline{toc}{chapter}{Résumé}
\label{sec:acknowledgement}
\begin{flushright}
%\Huge{\texttt{au quatro!}}

\end{flushright}
Ces quatre années de thèse n'ont pas été toujours faciles. Tout au long de mes travaux de recherche et de rédaction, j'ai néanmoins reçu un précieux soutient administratif, financier, et moral de plusieurs organismes et personnes. 
Ce mémoire est l'occasion pour moi de tous les remercier.

En premier, je remercie mes directeurs et encadrants de thèse pour leur confiance, leur disponibilité, leurs contributions, leur effort et leur patience. Je leur prie de m'excuser pour en avoir abusé très souvent. Merci Jacky et Stéphane pour avoir accepté de diriger cette thèse. Ça a été un honneur pour moi de vous avoir eu comme directeurs de thèse. Merci Guillaume pour avoir initié ce sujet très passionnant sur lequel j'espère avoir l'occasion de continuer à contribuer. Merci aussi pour les nombreuses journées de travail que nous avons passé ensemble. Elles m'ont été indispensables pour mieux comprendre les problèmes métiers des juristes, d'obtenir des données annotées, et de valider mes résultats. Merci à Sébastien pour les nombreuses sessions de travail durant lesquelles nous avons formulé les problèmes métiers en problèmes informatiques, réfléchi sur des approches, et vérifié les résultats expérimentaux. 

Je remercie Madame Sandra BRINGAY, Professeur de l'Université de Montpellier, et 
Monsieur Boughanem MOHAND, Professeur de l'Université Toulouse III Paul Sabatier, de m'honorer en acceptant d'être les rapporteurs de ma thèse. Je remercie aussi Madame Françoise Seyte, Maître de Conférences (HDR) de l'Université de Montpellier, et  Monsieur Fabrice MUHLENBACH, Maître de Conférences de l'Université Jean Monnet de Saint-Étienne, de me faire l'honneur d'être examinateurs de ma thèse.

Je remercie l'IMT Mines Alès pour avoir financer mes travaux. Je remercie aussi l'IMT Mines Alès et l'Université de Nîmes pour m'avoir accueilli. Ces remerciements s'adressent en particulier à Yannick VIMONT, ancien directeur du LGI2P, à Jacky MONTMAIN, actuel directeur du LGI2P, et à Benoît Roig, Président de l'université de Nîmes et ancien directeur de l'équipe d'accueil CHROME. Je remercie en particulier Valérie Roman, Claude Badiou, et Édith  Teychene, et Corinne VINCENT, pour leur aide précieuse pour les démarches administratives.

Je suis reconnaissant envers les chercheurs et doctorants de l'IMT mines Alès et de l'équipe Chrome pour leur gentillesse et les bons moments que j'ai partagé avec eux. C'était le quotidien avec ma collègue de bureau Valentina. C'était des discussions scientifiques avec mes collègues doctorants Pierre-Antoine, Diadie, Jean-Christophe, et bien d'autres. C'était le foot, le basket, et le ski avec Michel, Abdelhak, Sébastien, Baptiste, Blazo, Alexandre, Diadie, Mouhamadou, Frank, Behrang et les autres. C'était les trajets quotidiens entre Nîmes et Alès avec Frank, Christelle, Cécile et les autres. C'était des week-ends avec Clément et Julien.

Je remercierai enfin ma famille pour son soutien distant. Je sais que je ne communique pas beaucoup, mais..
\chapter*{Résumé}
\addcontentsline{toc}{chapter}{Résumé}

\textbf{Mots clés:} Analyse de données textuelles, corpus jurisprudentielle, extraction d'information, classification de documents,regroupement non supervisée de documents.
%Analyse de données textuelles, corpus jurisprudentielle, annotation d'entités nommées judiciaires, classification, extraction d'information structurée, catégorisation non supervisée.
%\textbf{Mots clés:} Automatisation de l'analyse de la jurisprudence, extraction d'information, étiquetage de sequence, prédiction structurée, apprentissage de métrique, catégorisation non-supervisée

%abstract
\chapter*{Abstract}
\addcontentsline{toc}{chapter}{Abstract}

\textbf{Mots clés:} text mining, jurisprudence

%legal descriptive analytics, legal textual data analytics, jurisprudencial decisions


%\textbf{Keywords}: Automation of the analysis of court decisions, information extraction, sequence labeling, structured prediction, metric learning, clustering


%liste des matières
\addcontentsline{toc}{chapter}{\contentsname}
\tableofcontents

%liste des figures
\addcontentsline{toc}{chapter}{\listfigurename}
\listoffigures
%
%%liste des tableaux
\addcontentsline{toc}{chapter}{\listtablename}
\listoftables

\mainmatter

\chapter*{Introduction générale}
\label{chap:intro}
\addcontentsline{toc}{chapter}{\nameref{chap:intro}}

\section{Contexte et motivations}
\label{sec:intro:contexte}
Une décision jurisprudentielle peut être définie soit comme  le résultat rendu par les juges à l'issue d'un procès, soit comme un document décrivant une affaire judiciaire. Un tel document rapporte, notamment,  les faits, les procédures judiciaires antérieures, le verdict des juges, et certaines explications associées. Dans cette thèse, nous désignons par \og décision \fg{} le document, et par  \og résultat\fg{} la conclusion, ou réponse finale des juges. Une jurisprudence\footnote{\url{http://www.toupie.org/Dictionnaire/Jurisprudence.htm}} est un ensemble de décisions rendues par les tribunaux ; elle représente la manière dont ces derniers interprètent les lois pour résoudre un problème juridique donné (type de contentieux). Les juristes doivent alors collecter ces documents, les sélectionner, et les analyser afin de mener, par exemple, des recherches empiriques en droit \citep{ancel2003expulsion, jeandidier2006pensions}. Les avocats exploitent aussi les décisions passées pour anticiper les résultats des juges. Ils peuvent ainsi mieux conseiller leurs clients sur le risque judiciaire que ces derniers encourent, et sur la stratégie à adopter pour un type de contentieux. 

Cette activité de collecte et d'analyse centrale pour de nombreux métiers du droit est aujourd'hui généralement effectuée de manière manuelle. Elles est par conséquent sujette à plusieurs difficultés notamment liées à l'accès et à l'exhaustivité des documents traités dans un contexte d'étude spécifique. Il faut ici notamment souligner que les documents sont dispersés dans les nombreux tribunaux, et que les procédures administratives ne facilitent pas toujours leur accès du fait de la nécessité de préserver la confidentialité des parties. En effet, les décisions n'étant la plupart du temps pas anonymisées, elles restent alors inaccessibles aux juristes qui en font la demande. Un certain nombre de documents sont néanmoins accessibles sur internet grâce à des sites de publication de données ouvertes gouvernementales, comme \url{http://data.gouv.fr} en France, \url{https://www.judiciary.uk} en Grande Bretagne, \url{http://www.scotusblog.com/} aux Etats-Unis, et \url{https://www.scc-csc.ca/} au Canada. Ces sites publient régulièrement des décisions récemment prononcées.  Il existe aussi des moteurs de recherche juridiques qui permettent de retrouver des décisions intéressantes. Cependant, qu'ils soient payants ( LexisNexis\footnote{\url{https://www.lexisnexis.fr/}}, Dalloz\footnote{\url{http://www.dalloz.fr}}, Lamyline\footnote{\url{http://lamyline.lamy.fr}},...) ou gratuits (CanLII\footnote{\url{https://www.canlii.org}}, Légifrance\footnote{\url{https://www.legifrance.gouv.fr}}, ...), les critères de recherche offerts par leurs moteurs de recherche d'information limitent grandement la pertinence des résultats pouvant être obtenus. En effet, il ne s'agit en général que de combinaisons de mots-clés et autres métadonnées (date, type de juridiction, ...), ou d'expressions régulières, comme l'illustre la Figure \ref{fig:intro:juriSearchForm}. Ces critères n'appréhendent pas la sémantique juridique, et ne permettent pas la plupart du temps aux juristes, sinon difficilement, la constitution d'échantillons pertinents pour leurs études. 

\begin{figure}[!htb]
	\centering
%	\begin{subfigure}[t]{0.95\textwidth}
%		\centering
%		\includegraphics[width=0.9\textwidth]{legifrance.PNG}
%		\caption{Formulaire de Légifrance}
%	\end{subfigure}%

	\begin{subfigure}[t]{0.45\textwidth}
		\centering
		\includegraphics[scale=0.4]{dalloz.png}
		\caption{Formulaire de Dalloz}
	\end{subfigure}\hfill
	\begin{subfigure}[t]{0.55\textwidth}
		\centering
		\fbox{\includegraphics[scale=0.35]{lexisnexis.png}}
		\caption{Formulaire de LexisNexis}
	\end{subfigure}%
	\caption{Exemples de critères des moteurs de recherche juridique}\label{fig:intro:juriSearchForm}
\end{figure}

Plus de 4 millions de décisions sont prononcées en France par an d'après les chiffres du ministère français de la justice (Tableau \ref{tab:intro:nbdecisionstats}). Dans ce contexte, l'exhaustivité, ou tout au moins la représentativité d'une analyse menée de manière traditionnelle, manuellement, est aussi fortement limitée du fait de l'énorme volume de documents existants. 
Au regard de la croissance rapide du nombres de décisions accumulées chaque année, on imagine facilement que même une étude sur une question très précise nécessite la constitution laborieuse d'un large corpus de décisions. Par ailleurs, il peut s'avérer très pénible de lire les décisions pour en identifier les données d'intérêt. Les documents sont très souvent longs et complexes dans leur style de rédaction. Par exemple, les phrases comprennent très souvent plusieurs clauses discutant d'aspects différents. On y retrouve aussi des références à des jugements antérieurs, et des omissions.


Il est évident qu'une automatisation du traitement des corpus de décisions s'impose pour répondre aux diverses difficultés d'accès, de volumétrie, et de complexité liées à la compréhension des décisions. L'automatisation ferait gagner du temps aux juristes lors de tâches de traitement préalables à leur raisonnement d'experts, tout en leur fournissant une vue pertinente de la jurisprudence. D'autre part, \citet{cretin2014justicecomplexe} fait remarquer que la justice est complexe dans son organisation (Figure \ref{orgjusticefrance}) et son fonctionnement, et que son langage est pratiquement incompréhensible. Il est donc presque impossible pour les "profanes" d'estimer leurs droits et le risque judiciaire qu'ils encourent dans leur quotidien sans consulter un initié du droit. L'automatisation de l'analyse jurisprudentielle pourrait ainsi améliorer l'accessibilité du droit dans ce cas.  L'exigence pour le profane étant l'exacte pertinence des ressources, leur accessibilité, et l'intuitivité du processus de leur exploitation \citep{narazenko2017legalnlpintro}. Le traitement automatique de la jurisprudence constituerait alors une aide précieuse non seulement pour les professionnels du droit, mais aussi pour les particuliers et les entreprises soucieux de voir l'issue de leur affaire leur être favorable. Par exemple, en comparant le montant qu'on peut espérer d'une juridiction et le coût d'un procès, on peut plus aisément se décider entre un arrangement à l'amiable et la poursuite du litige en justice \citep{langlaischappe2009ecoresolutionlitige}. 

\begin{figure}[!htb]
	\centering \includegraphics[width=0.9\textwidth]{organisation_justice_francaise_grand.jpg}
	
	\textit{\scriptsize{Source: \url{http://www.justice.gouv.fr/organisation-de-la-justice-10031/}}}  
	\caption{Organisation des institutions judiciaires françaises} \label{orgjusticefrance}
\end{figure}

\section{Objectifs}
%\textcolor{red}{Description d'une approche traditionnelle, exemple d'études, difficultés de ces études}
 Ce mémoire propose des stratégies et modèles visant à automatiser l'extraction d'information à partir des décisions françaises. Le but est de faciliter la constitution et l'analyse descriptive de corpus de décisions de justice. L'approche traditionnelle d'analyse d'un contentieux \citep{ancel2003expulsion} consiste à :
 \begin{enumerate}
 	\item \textbf{Choisir un échantillon représentatif}: collection des décisions suivant des contraintes définies:  période précise, couverture géographique, types d'affaires, etc.
 	\item \textbf{Sélectionner les décisions}: élimination des décisions qui ne correspondent pas au type de demande d'intérêt.
 	\item \textbf{Élaborer la grille d'analyse}: création d'un modèle de grille qui permettra d'enregistrer les informations potentiellement importantes. Chaque ligne de la grille correspond à une demande, et les colonnes font référence aux différents types d'informations qu'il est possible d'extraire sur une demande. Ces variables vont de la procédure suivie, aux solutions proposées, en passant par la nature de l'affaire. Les champs à remplir ne sont pas connus à l'avance ; ce n'est généralement qu'au cours de la lecture des décisions que l'on distingue les informations pertinentes pour l'étude.
 	\item \textbf{L'analyse des décisions et l'interprétation des informations}: saisie des décisions et calculs statistiques dans un logiciel tableur.
 \end{enumerate}
 
\citet{ancel2003expulsion} évoque principalement le problème de la différence entre l'état capté de la jurisprudence et son état présent. D'une part en effet, les longs délais de travail sont caractéristiques de ces études. Nous avons pour exemple, l'étude menée  par l'équipe de \citet{jeandidier2006pensions} pour l'analyse empirique des déterminants de la fixation de pensions alimentaires pour enfant lors de divorce. Cette analyse a duré 9 mois pour l'extraction manuelle des informations et la modélisation par régression de la relation entre les déterminants extraits et les pensions alimentaires accordées.  D'autre part, il est impossible d'observer l'évolution des pratiques judiciaires dans le temps et dans l'espace du fait de la faible taille de l'échantillon choisi. Notre principal objectif est donc de proposer des solutions pour un traitement rapide et efficace d'une grande masse de décisions. 
 
 La problématique de notre étude est de \og \textbf{capter automatiquement la sémantique d'un corpus jurisprudentiel pour comprendre la prise de décision des juges sachant que l'interprétation subjective des règles juridiques rend l'application de la loi non déterministe} \fg{}. Cette question intéresse des entreprises telles que LexisNexis, et plusieurs startups  telles que Predictice\footnote{\url{http://predictice.com}} et CASE LAW ANALYTICS\footnote{\url{http://caselawanalytics.com}}. Afin d'y répondre, nous nous intéressons aux concepts manipulés par les experts, au centre desquels nous retrouvons les demandes des parties (prétentions) qui feront l'objet d'une décision. En effet, l'analyse sémantique d'un corpus jurisprudentiel vise
 %  \textcolor{red}{la plupart du temps} 
 l'identification de connaissances sur les demandes présentent dans les décisions. Ces demandes sont associées à plusieurs concepts  qui enrichissent la compréhension de la décision (Figure \ref{fig:intro:demande-central}).
 \begin{figure}[!htb]
 	\centering
 	\includegraphics[scale=0.7]{demande-central.png}
 	\caption{La demande au centre de la compréhension des décisions}
 	\label{fig:intro:demande-central}
 \end{figure} 

Une demande peut ainsi être caractérisée par :
 \begin{itemize}
	\item le résultat associé qui est décrit par une polarité (\og accepte \fg{} ou \og rejette \fg{}), souvent lié à un quantum accordé, par exemple 5000 euros de dommages et intérêts ou 2 mois d'emprisonnement ;
	\item le fondement ou la norme juridique qui est la règle qui est associée et qui légitime la prétention ou le résultat ;	
	\item l'objet qui a été demandé (par ex. dommages et intérêts) ;
	\item les circonstances factuelles dans lesquelles sont formulées les demandes ; elles caractérisent ainsi les d'affaires ;
	\item les divers arguments apportés par les parties (resp. les juges) pour justifier leurs requêtes (resp. leurs solutions).
 \end{itemize}

Ces concepts descriptifs d'une demande couvrent très souvent l'essentiel de l'information pertinente pour les experts. 

Les travaux de cette thèse s'inscrivent dans un projet qui vise, entre autres, à automatiser l'extraction de l'ensemble de ces informations et de les structurer afin d'enrichir une base de connaissances contenant des informations détaillées de la jurisprudence française. Une telle base permettrait notamment de mener des études sur différents critères comme la juridiction, le type de demande, ou encore les circonstances du litiges, dans différents contextes de prise de décision juridique. Elle aurait aussi tout naturellement une importance certaine pour l'étude  de la définition de modèles prédictifs variés, e.g. de l'application du droit, par exemple pour la prédiction des types de demandes à effectuer dans le cadre d'un litige. 

Le projet comprend deux phases principales : une phase d'indexation des connaissances de la masse des décisions, suivie d'une analyse prédictive. La phase d'indexation doit déjà permettre de réaliser automatiquement, de manière exhaustive, des analyses descriptives. Ces dernières consistent, par exemple, à comparer le nombre d'acceptations à la fréquence des rejets. Par conséquent, le système doit apprendre à reconnaître dans les décisions, les informations pertinentes sur les prétention et résultats associés. La phase d'analyse prédictive consiste à regrouper des paquets de décisions similaires (même résultat sur la même prétention dans les circonstances similaires), pour découvrir les facteurs influençant le sens du résultat (par ex. le fait que \og le revenu de l'époux soit le plus élevé du foyer\fg{} encourage les juges à accorder la pension alimentaire à l'épouse). En effet, c'est la connaissance de ces facteurs circonstanciels qui permet à l'expert de pouvoir anticiper les décisions judiciaires.

 La chaîne de traitement à mettre en \oe uvre se compose de quatre étapes principales qui s'enchaînent comme le présente la figure \ref{fig:intro:pipeline-globale}. Notre étude s'intéresse donc aux problématiques liées la constitution de la base de connaissance et à son exploitation dans un contexte d'analyses descriptives. Celles-ci sont décrites dans la suite.
\begin{figure}
	\includegraphics[width=\textwidth]{pipeline-cassandra.pdf}
	\caption{Chaîne d'analyse du corpus jurisprudentiel à mettre en \oe uvre} \label{fig:intro:pipeline-globale}
\end{figure} 


\subsection{Collecte, gestion et pré-traitement des documents}

 Le volume de décisions prononcées croît très rapidement (Tableau \ref{tab:intro:nbdecisionstats}). 
 \begin{table}[!htb]
 	\small
 	\begin{center}
 		\begin{tabular}{|l|l|l|l|l|l|}
 			\hline
 			\textbf{Justice}	& \textbf{2013}      & \textbf{2014}      & \textbf{2015}      & \textbf{2016}      & \textbf{2017}      \\ \hline
 			civile         & 2 761 554 & 2 618 374 & 2 674 878 & 2 630 085 & 2 609 394 \\ \hline
 			pénale         & 1 303 469 & 1 203 339 & 1 206 477 & 1 200 575 & 1 180 949 \\ \hline
 			administrative & 221 882   & 230 477   & 228 876   & 231 909   & 242 882   \\ \hline
 		\end{tabular}
 		
 		\textit{\scriptsize{Source: \url{http://www.justice.gouv.fr/statistiques-10054/chiffres-cles-de-la-justice-10303/}}}  
 	\end{center}
 	\caption{Nombre de décisions prononcées en France par an de 2013 à 2017}\label{tab:intro:nbdecisionstats}
 \end{table}
Il est donc nécessaire de trouver des moyens pour collecter le maximum de documents bruts non-structurés, les pré-traiter, et organiser leur gestion afin de les indexer en local pour faciliter leur traitement. Les décisions de cours d'appel de justice civile sont les plus accessibles à partir des moteurs de recherche juridique (LexisNexis, Dalloz, LamyLine, Legifrance, etc.) et de la grande base de données JuriCa. Cependant, l'accès à ces décisions est généralement payant, et le nombre de documents simultanément téléchargeables est très faible sur les sites payants (généralement 10 à 20 décisions au maximum à la fois). De plus, le nombre de téléchargements par jour est limité. La base JuriCa est la plus grosse base de décisions de cours d'appel en France. Elle est gérée par la Cour de cassation. L'accès à cette base est offert par le Service de Documentation, des Etudes et du Rapport\footnote{\url{https://www.courdecassation.fr/institution_1/composition_56/etudes_rapport_28.html}} (SDER). L'accès est payant pour les professionnels et gratuit pour les universités et centres de recherche en partenariat avec le SDER. Légifrance, le moteur de recherche du ministère de la justice, fournit quant à lui un accès public et gratuit à un nombre considérable de documents. Les décisions y sont identifiées à l'aide de numéros consécutifs et accessibles à partir d'un service web. Ce dernier a l'avantage de proposer des décisions de tous les ordres et de tous les degrés. Cependant, les décisions des juridictions du premier degré (appelées jugements) restent plus rares sur internet et principalement disponibles auprès des tribunaux.  La disponibilité des décisions du second degré ou d'appel (appelées arrêts) en justice civile est l'une des raisons pour lesquelles notre étude s'est portée sur celles-ci.

Les décisions existent sous divers formats PDF, DOC, DOCX, RTF, TXT, XML, etc. Il arrive parfois qu'un fichier téléchargé comprenne plusieurs décisions (sur LexisNexis par exemple). Nous avons par conséquent préféré convertir tous les documents au format plein texte pour homogénéiser les traitements. Par ailleurs, les décisions sont collectées à partir de diverses sources pouvant contenir des documents identiques. Il se pose donc un problème d'identification unique des décisions pour éviter des redondances. Pour cela, nous avons défini une convention de nommage des fichiers. Ce dernier repose sur 3 informations: le type de juridiction (tribunal, cour d'appel, ...), la ville, et le numéro R.G. (registre général) qui est l'identifiant unique de la décision au sein de la juridiction. Par exemple, le numéro \og CAREN1606137 \fg{} identifie la décision de numéro R.G. \og 16/06137 \fg{} de la cour d'appel (\og CA \fg{}) de la ville de Rennes (\og REN \fg{}). Ces 3 informations sont présentes dans les premières lignes de la décision, et sont facilement identifiables à l'aide d'une routine à base de règles simples. D'autre part, certains moteurs de recherche ne fournissent souvent qu'un résumé au lieu du contenu original des décisions. Il est important de supprimer ces fichiers du corpus.

\subsection{Extraction de connaissances}
\label{subsec:intro:ie}
Les problématiques d'extraction de connaissances constitue la pierre angulaire de cette thèse car les informations sur les demandes, les parties, les juges, les juridictions et les faits conditionnent la qualité des prévisions du sens du résultat pour un type de demande considéré.  La difficulté découle de l'état non-structuré des documents, et de la complexité et la spécificité du langage employé. L'extraction des connaissances nécessite de mettre en \oe uvre des techniques de fouille de textes adaptées à la nature des éléments à identifier. Nous avons ainsi abordé l'annotation des références de l'affaire (juridiction, ville, participants, juges, date, numéro R.G., normes citées, ...), l'extraction des demandes et résultats correspondants, et l'identification des circonstances factuelles.

Les métadonnées de référence sont des segments de texte qu'on peut directement localiser dans le document. Leur reconnaissance est donc semblable à celle des entités nommées. C'est une problématique intensivement étudiée en traitement automatique du langage naturel \citep{yadav2018surveyNeuralNER} dans plusieurs travaux et compétitions, aussi bien pour des entités communes \citep{tjong2003introCoNLL,grishman1996muc6}, que pour des entités spécifiques à un domaine \citep{kim2004bioNer, persson2012nbbioner,hanisch2005prominer}, et dans diverses langues \citep{li2018wcpbioner,alfred2014malayner,amarappa2015kannada}. 

Le problème d'extraction des demandes et de la réponse correspondante des juges consiste à reconnaître pour chaque prétention : son objet, son fondement, le quantum demandé, le sens du résultat, et le quantum accordé. La paire demande-résultat s'apparente donc à des entités structurées comme les évènements \cite{ace2005event} qui sont décrits par un type, un terme-clé, des participants, un temps, une polarité.

Le problème d'identification des circonstances factuelles consiste à constituer des regroupements de décisions mentionnant une certaine catégorie de demande (objet+fondement). Le but est, comme indiqué précédemment, de repérer les différentes situations dans lesquelles cette catégorie de demande est formulée. Chacun des groupes représente donc une situation particulière partagée par les membres du groupe mais bien distinctes de celles reflétées par les autres groupes. Ce problème évoque des problématiques de similarité entre texte, de catégorisation non supervisé (\textit{clustering}), et de \og modélisation thématique \fg{} (\textit{topic modeling}).  La similarité pourra faire l'objet, dans un travail futur, d'identification des raisons.

A l'issue du processus d'extraction, les données extraites sont destinées à enrichir progressivement une base de connaissances. La structuration des données au sein d'une base facilite les diverses analyses automatiques applicables aux décisions et demandes judiciaires. 

\subsection{Analyse descriptive}
L'analyse descriptive exploite l'ensemble des connaissances extraites et organisées pour répondre aux diverses questions que l'on pourrait se poser sur l'application de la loi. Il est intéressant par exemple de comparer les fréquences de résultats positifs et négatifs pour une catégorie de prétention donnée dans une situation précise. Les quanta extraits servent à visualiser les différences entre les montants accordés et réclamés. D'autres analyses plus complexes permettraient d'étudier l'évolution dans le temps et les différences dans l'espace de l'opinion des juges.


\section{Méthodologie}
\label{sec:intro:methodologie}

Comme illustrées précédemment (\ref{subsec:intro:ie}), les problématiques propres aux textes juridiques trouvent généralement des analogies avec les problèmes d'analyse de données textuelles (\textit{text mining}). Ainsi, les méthodes issues de ce domaine sont applicables aux textes juridiques. Cependant, quelques adaptations sont généralement nécessaires pour obtenir des résultats de bonne qualité hors des domaines pour lesquels ces approches ont été développées \citep{Waltl2016lexia}. De plus, la recherche en fouille de textes est souvent réalisée sur des échantillons qui ne reflètent pas toujours la complexité des données réelles. Effectuant l'une des premières études d'analyse sémantique des décisions françaises, nous avons axé notre travail sur le rapprochement des problèmes liés à l'analyse des décisions jurisprudentielles à ceux généralement traités en analyse de textes. Il s'agit ensuite d'établir des protocoles d'évaluation et d'annotation manuelle de données. Selon les problématiques identifiées et les protocoles d'évaluations définis, des méthodes adaptées ont été proposées et expérimentées sur les données réelles annotées manuellement par un expert juriste.

\section{Résultats}
\label{sec:intro:résultats}
Une chaîne de traitement pour le sectionnement et l'annotation des métadonnées est proposée. L'applicabilité de deux modèles probabilistes, les champs aléatoires conditionnels ou CRF (\textit{conditional random fields}) et les modèles cachés de Markov ou HMM (\textit{hidden Markov Model}), est étudiée en considérant plusieurs aspects de la conception des systèmes d'extraction d'entités nommées. Le sectionnement a pour but d'organiser l'extraction des informations qui sont réparties dans des sections selon leur nature. 

Par la suite, nous proposons une méthode d'extraction des demandes et résultats en fonction des catégories présentes dans la décision. L'approche consiste en effet à identifier dans un premier temps les catégories présentes (objet+fondement) par classification supervisée. Un vocabulaire d'expression des demandes et résultats est exploité pour identifier les passages. Puis à l'aide de termes propres à chacune des catégories identifiées, les trois attributs (quantum demandé, sens du résultat, quantum accordé) des paires demande-résultat sont reconnus. 

Par ailleurs, nous analysons l'extraction du sens du résultat par classification binaire des documents. L'objectif est de s'affranchir de l'identification préalable de l'expression des demandes et résultats. En effet, les décisions comprenant des demandes d'une catégorie donnée semblent ne contenir, dans une forte proportion, qu'une seule demande. Nous pensons qu'il n'est par conséquent pas nécessaire d'identifier l'expression de cette dernière pour en déterminer le sens. A partir d'une représentation adéquate du contenu de la décision, il est possible de classer cette dernière à l'aide d'un modèle de classification supervisée de documents.

L'identification des circonstances factuelles, quant à elle, est modélisée comme une tâche de regroupement non supervisé de documents. Nous proposons dans ce cas une méthode d'apprentissage d'une distance entre textes, à l'aide d'un algorithme de régression. La métrique apprise est utilisée dans l'algorithme des \og K-moyennes \fg{} (\textit{k-means}) \citep{forgey1965kmeans} et celui des \og K-medoïdes \fg{} (\textit{k-medoids}) \citep{kaufman1987kmedoids}, et comparée à d'autres distances établies en recherche d'information.

\section{Structure de la thèse}
\label{sec:intro:organisation}

La thèse est organisée en 6 chapitres. Le chapitre \ref{chap:literature} positionne nos travaux par rapport à ceux qui ont été réalisés précédemment sur des problématiques d'analyse automatique de décisions de justice. Le chapitre \ref{chap:structuration} présente les architectures et modèles proposés pour la structuration des décisions et la reconnaissance des entités juridiques ; il discute notamment des différents résultats empiriques obtenus par application des modèles CRF et HMM. Ensuite, le chapitre \ref{chap:quanta} détaille le problème  d'extraction des demandes, puis présente notre méthode et les résultats obtenus. Le chapitre \ref{chap:sensresultat} traite de l'extraction du sens du résultat par classification directe des décisions, cela en comparant différents algorithmes de classification et représentations des textes. Le chapitre \ref{chap:similarite} discute de l'usage de l'apprentissage proposé d'une distance qui est comparée à d'autres distances pour la découverte des circonstances factuelles. Enfin, le chapitre \ref{chap:demo} présente les résultats de scénarios d'analyses descriptives pour illustrer l'exploitation potentielle de nos propositions sur un corpus de grande taille. 
 % INCLUDE: introduction

% changement du style des chapitres
\renewcommand{\chaptertitlename}{Chapitre }
\renewcommand{\thechapter}{\arabic{chapter}}
\renewcommand{\thesection}{\thechapter.\arabic{section}}
\renewcommand\thesubsection{\thesection.\arabic{subsection}}
\renewcommand\chaptermark[1]{%
	\markboth{\MakeUppercase{\chaptername\ \thechapter.\ {#1}}}{#1}}

%{\usefont{OT1}{ptm}{b}{n}
 \titleformat{\chapter}[display]
{\bfseries\Large\filleft}
{\filright {\chaptertitlename } 
\Large\thechapter}
{1ex}
{\titlerule[3pt]%
\vspace{2pt}%
\titlerule
\vspace{2ex}%
\filright}
[\vspace{2ex}%
\titlerule]

%\renewcommand{\chaptertitlename}{Chapitre }
%\renewcommand{\thepart}{\arabic{part}}
%{\usefont{OT1}{ptm}{b}{n}
 \titleformat{\part}[display]
{\bfseries\Huge\filleft}
{\filright {\partname } %\textcolor{blue}
\Huge}
{1ex}
{\titlerule[3pt]%
\vspace{5pt}%
\titlerule
\vspace{5ex}%
\filright} % \textcolor{blue}
[\vspace{5ex}%
\titlerule]

%
%}

% % partie
%  \titleformat{\part}[frame]
% {\normalfont}
% {\filright
% \footnotesize
% \enspace \bf \Huge \thepart\enspace}
% {18pt}
% {\Large\bfseries\filcenter}

%\chapter{Analyse sémantique de Corpus Textuel par Traitement Automatique du Langage Naturel}
\chapter{Analyse automatique de corpus judiciaires}
\label{chap:literature}


% justice prédictive: limites: fiabilité mathématiques, exhaustivité, résultats différents d'un outils à un autre, quelles données analysées? % necessité: réduire le risque d'erreur d'une 
L'étude bibliographique de ce chapitre est focalisée sur l'application de techniques d'analyse de données textuelles judiciaires avec un accent particulier sur les décisions. L'état de l'art plus technique est décrit dans les chapitres qui traitent, dans la suite, des méthodes que nous avons mises en \oe{}uvre.

\section{Introduction}

% rapport aux théories juridiques: réalisme vs formalisme
Les deux grand paradigmes de jugement se distinguent par l'importance qu'ils accordent aux règles juridiques \citep{tumonis2012legalrealism}. D'une part, les adeptes du Formalisme Juridique, plus partie pertinent dans le droit civil, considèrent que toutes les considérations normatives ont été incorporées dans les lois par leurs auteurs. D'autre part, l'école du Réalisme Juridique, plus proche du droit commun, permet un pouvoir discrétionnaire entre les jugements en raisonnant selon le cas. Les premières tentatives d'anticipation des comportements judiciaires s'appuyaient sur une formalisation des lois. Le \og droit computationnel \fg{} est la sous discipline de  l'\og informatique juridique\footnote{Application les techniques modernes de l'informatique à l'environnement juridique, et par conséquent aux organisations liées au droit} \fg{} qui en est née. Il  s'intéresse, en effet, au raisonnement juridique automatique  axé sur la représentation sémantique riche et plus formelle de la loi, des régulations, et modalités de contrat \citep{love2005computationallaw}. Il vise à réduire la taille et la complexité de la loi pour la rendre plus accessible. Plus précisément, le \og droit computationnel \fg{} propose des systèmes répondant à différentes questions, comme \og Quel montant de taxe dois-je payer cette année? \fg{} (planification juridique), \og Cette régulation contient-elle des règles en contradiction\fg{} \og L'entreprise respecte-t-elle la loi?" (vérification de la conformité) \citep{Genesereth2015computationallaw}. Les techniques pro Formalisme Juridique étaient déjà critiquées au début des années 60, d'être insuffisante car focalisant excessivement sur les règles juridiques qui ne représentent qu'une partie de l'institution juridique \citep{llewellyn1962jurisprudence}. Pour anticiper le comportement judiciaire, plusieurs variables plus ou moins contrôlables, comme le temps, le lieu et les circonstances, doivent aussi être nécessairement prise en compte \citep{ulmer1963quantitative}. Les avocats s'appuyant sur la recherche de précédents, \citet{ulmer1963quantitative} conseille de se concentrer sur les motifs réguliers que comprennent les données pour réaliser des analyses quantitatives. Nous exploitons ainsi la masse de décisions pour identifier de telles régularités car une collection suffisante d'une certaine forme de données révèle des motifs qui une fois observés sont projetables dans le futur \citep{ulmer1963quantitative}. Il s'agit donc de raisonnements à base de cas qui se distinguent du raisonnement à base de règles.

% Généralités sur l'application du text mining / IA en général aux documents juridique: objectifs, données, conférences, commercialisation, activités gouvernementales, inquiétudes ...
Les premiers outils automatiques d'anticipation des décisions étaient généralement des systèmes experts juridiques. Ces derniers résonnent  sur de nouvelles affaires en imitant la prise de décision humaine généralement par la logique et souvent par analogie. Ils s'appuient sur un raisonnement à base de règle c'est-à-dire à partir d'une représentation formelle des connaissances des experts ou du domaine. En droit, il s'agit de la connaissance qu'à l'expert des normes juridiques et de l'ordre des questions à traiter lors du raisonnement sur un cas (appris par expérience). Le modèle explicite de domaine nécessaire ici se trouve dans une base de connaissances où les normes juridiques sont représentées sous forme de \og SI ... ALORS ...\fg{}, et les faits sont généralement représenté dans la logique de prédicat. Un système d’experts juridiques doit s’appuyer sur une base de connaissances juridiques exhaustive et disposer d’un moteur d’inférence capable de trouver les règles pertinentes et le moyen efficace, par déduction, de les appliquer afin d’obtenir la solution du cas actuel aussi rapidement que possible. Des systèmes experts ont échoué dans leur tentative de prédire les décision de justice \citep{leith2010risefall}. La première raison découle de ce que \citet{Berka2011rbr-cbr} a appelé le \og goulot d'acquisition de connaissances \fg{} c'est-à-dire le problème d'obtention des connaissances spécifiques à un domaine d’expertise sous la forme de règles suffisamment générales. L'autre raison tient à l'interprétation ouverte du droit et à la complexité de la formalisation applicable sans tenir compte des particularités de l'affaire.

Contrairement au raisonnement à base de règle, le raisonnement à bas de cas concerne une recherche de solution, une classification ou toute autre inférence pour un cas courant à partir de l'analyse d'anciens cas et de leur solution \citep{moens2002case-basedreasoning}. Un tel système juridique résout les nouveaux cas en rapprochant les cas déjà réglés et en adaptant leurs décisions \citep{Berka2011rbr-cbr}. Le raisonnement fondé sur des cas connaît un succès croissant dans la prédiction de l'issu d'affaire plus aux États-Unis qu'ailleurs. Pour exemple, \citet{katz2014predicting} entraînent des forêts aléatoires sur les cas de 1946-1953 pour prédire si la Cour Suprême infirmera ou confirmera une décision de juridiction inférieure. Ils ont réussi à atteindre 69,7\% des décisions finales pour 7 700 cas des années 1953-2013; des résultats qu'ils ont légèrement améliorés plus récemment en augmentant le nombre d'arbre et la quantité de données \citep{katz2017predictsupremecourt}. D'autre part, \cite{Ashley2009classifCases} ont obtenu une précision de 91,8\% pour prédire la partie qui sera favorisée (plaignant/défendeur) pour les affaires d'appropriation illicite de secrets commerciaux. Contrairement à \citep{katz2014predicting} qui catégorisent les caractéristiques de valeurs prédéfinies pour caractériser la décision débattue, les tribunaux et les juges (opinions politiques, origine de l'affaire, identifiant du juge, raison et sens du dispositif de la cour inférieure), \cite{Ashley2009classifCases} identifient, par classification, des facteurs pouvant influencer la décision. Les valeurs des caractéristiques de ces différents travaux sont prédéfinis et très limités, et ne reflètent pas, par conséquent, la grande diversité de catégories qu'on peut retrouver dans les décisions. 

% introduction des sections suivantes
Nous voulons alimenter les analyses quantitatives de corpus jurisprudentiels en proposant des méthode d'extraction de connaissances dont les références des affaires (juge, date, juridiction, etc.), les règles juridiques associées, les demandes de parties, les réponses des tribunaux, et les liens entre ces données. Les juges apportent une réponse à chaque demande, et par conséquent une partie peut voir ses demandes soit toutes acceptées ou rejetées, soit une partiellement accordées. Un juriste sera donc plus intéressé à formuler les demandes qui ont de meilleures chances d'être acceptées, pour un type de contentieux précis,  qu'à prévoir une victoire du procès. C'est la raison pour laquelle notre analyse se situe à un niveau de granularité plus fin (la demande), contrairement aux travaux sur la prédiction qui traitent d'un résultat global sur la décision (par ex. confirmer/infirmer ou gagner/perdre).  L'identification de ces diverses connaissances est possible par l'analyse sémantique des textes judiciaires grâce aux méthodes du traitement automatique du langage naturel et de l'analyse (ou fouille) de données textuelles. Cependant, l'application de ces techniques exigent certaines adaptations pour surmonter les divers défis posés par les documents juridiques en général \citep{narazenko2017legalnlpintro}: textes très longs et en grande quantité, corpus régulièrement mis à jour, influence subjective de facteurs sociaux et d'opinions politiques, couvertures de problématiques économiques, sociales, politiques très variées, langage complexe, etc. . Dans la suite, nous passons en revue des travaux qui ont été menés dans ce sens pour traiter de problématiques proches des nôtres, en particulier celles décrites précédemment dans l'introduction (Section \ref{subsec:intro:ie}). 

\section{Annotation et extraction d'information}

L'annotation consiste à enrichir les documents pour les préparer à d'autres analyses, faciliter la recherche d'affaires pertinentes, et faire la lumière sur des connaissances linguistiques sous-jacentes au raisonnement juridique. Les éléments annotés peuvent être de très courts segments de texte mentionnant des entités juridiques \citep{Waltl2016lexia, wyner2010extractlegalelts} comme la date, le lieu (juridiction), les noms de juges, des citations de loi.  L'annotation de passages plus longs identifie des concepts juridiques plus complexes comme les faits \citep{wyner2010extractlegalelts, wyner2010casefactors, Shulayeva2017recognfactprincip}, les définitions \citep{Waltl2016lexia}, des citations de principes juridiques \citep{Shulayeva2017recognfactprincip}, ou des arguments \citep{WynerMoens2010mineargument}. 

Différentes méthodes ont été expérimentées pour la reconnaissance d'information dans les documents judiciaires. C'est le cas des modèles probabilistes HMM et CRF que nous étudions dans le chapitre \ref{chap:structuration}. Ils peuvent être combinés à d'autres approches dans un système global. Après avoir segmenter les documents à l'aide d'un modèle CRF, \citet{dozier2010legalnerr} ont combiné plusieurs approches pour reconnaître des entités dans les décisions de la cour suprême des États-Unis. Ils ont définis des détecteurs distincts à base de règles pour identifier séparément la juridiction (zone géographique), le type de document, et les noms des juges, en plus de l'introduction d'une recherche lexicale pour détecter la cour, ainsi qu'un classifieur entraîné pour reconnaître le titre. Ces différents détecteurs ont atteint des performances prometteuses, mais avec des rappels limités entre $ 72 \% $ et $ 87 \% $. Suivant la complexité des éléments à extraire, un système peut comprendre des indexes lexicaux pour les motifs simples et non-systématiques (indicateurs de mentions de résultats ou de parties) et les règles pour des motifs plus complexes et systématiques (par ex. noms de juges, énoncés de décisions) \citep{Waltl2016lexia, wyner2010extractlegalelts}. \cite{cardellino2017legalNERCL} quant à eux ont utilisé le CRF et les réseaux de neurones sur des jugements de la Cour Européenne des Droits de l'Homme. Les basses performances qu'ils rapportent pour l'extraction dans les jugements illustre bien la difficulté de la détection d'entités juridiques. Plus récemment encore, \citet{andrew2018legalNerAndRelation} obtiennent de bons résultats en combinant l'extraction d'entités non-juridiques par CRF à celle des relations entre ces dernières par une grammaire GATE JAPE \citep{thakker2009gatejape} sur des décisions du Luxembourg rédigées en français.

Pour la détection des arguments, par contre, \citet{moens2007NBvsMaxent4arguments} discutent de l'application d'une classification binaire des phrases: \textit{argumentative} / \textit{non-argumentative}. Ils comparent notamment le classifieur bayésien multinomial et le classifieur d'entropie maximum tout en explorant plusieurs caractéristiques textuelles.

% argument (Grammaire) :\cite{WynerMoens2010mineargument} http://wyner.info/research/Papers/WynerMochalesPalauMoensMilward2009.pdf
% terminologie : https://pdfs.semanticscholar.org/4d49/2d103672723d5683e4fc5b468e49ffaece3b.pdf

\section{Classification des documents}
La classification permet d'organiser un corpus en rangeant les documents dans des catégories prédéfinies.  \cite{Aletras2016predictDecisionECHR} identifient s'il y a eu une violation d'un article choisi de la convention des droits de l'homme sur les jugements \footnote{HUDOC ECHR Database: \url{http://hudoc.echr.coe.int}} de la Court Européenne des Droits de l'Hommes (ECHR). Avec un SVM (Machine à Vecteurs de Support) et une représentation vectorielle basée sur les plus fréquents n-grammes et le cluster de leur vecteur de plongement sémantique (word2vec), ils obtiennent une précision moyenne de 79\% sur les 3 articles qu'ils ont manipulés. Notons tout de même la sélection des régions du documents où sont extraits les N-grammes (circonstances, faits, lois, ...). Cette sélection est un ajustement de la représentation des texte qui paraît nécessaire pour obtenir de bon résultat. La structuration préalable des documents est utile pour réduire le bruit qui occupe généralement plus d'espace que les passages d'intérêt.  \citet{medvedeva2018echrCristalBall} étendent ces travaux à neuf articles tout démontrant empiriquement, entre autres, la possibilité de prédire la violation des articles sur des périodes futures à celles des données d'entraînement. \cite{sulea2017legalEnsSVM} traitent, d'autre part, l'identification des résultats dans des arrêts \footnote{Documents de \url{https://www.legifrance.gouv.fr}} de la Court Française de Cassation. Après un essai avec un SVM \citep{Sulea2017predictareadecision}, ils améliorent les résultats à l'aide d'un classifieur ensembliste de SVM à moyenne de probabilités, parvenant ainsi à des F1-mesures de plus de 95\%. 

Par ailleurs, \cite{Ashley2009classifCases} entraînent un classifieur (les plus-proches-voisins) pour chacun des 27 facteurs prédéfinis pour savoir s'il s'applique à la décision (phase SMILE). La partie remportant le procès est prédite par un algorithme séquentiel qui compare les parties (plaignant et défendeur) suivant le niveau de préférence des questions juridiques dégagées par les facteurs tel qu'observé dans la base d'entraînement (phase IBP).  D'autres catégorisations, comme la formation judiciaire ou la période du prononcé \citep{Sulea2017predictareadecision,sulea2017legalEnsSVM}, sont toutes aussi utiles pour faciliter la recherche d'information.

\section{Similarité}
% https://scholar.google.com/scholar?oe=utf-8&client=firefox-b-ab&um=1&ie=UTF-8&lr&cites=2644458803665738328
% https://www.google.com/url?sa=t&rct=j&q=&esrc=s&source=web&cd=6&cad=rja&uact=8&ved=2ahUKEwik05PbjdvdAhUI_qQKHU9UC6QQFjAFegQIAxAC&url=http%3A%2F%2Fweb2py.iiit.ac.in%2Fpublications%2Fdefault%2Fdownload%2Finproceedings.pdf.8d3930f256a00e9c.436f6d707574655f323031315f53757368616e74615f4b756d61722e706466.pdf&usg=AOvVaw3CQX2nPEbeTXt6LhlRoOj6
% quel sémantique fonde la similarité dans chaque travaux? ou comment est défini la similarite entre les documents (dans la sémantique experte) ?
% quelle métrique formalise / numérise / mesure la similarité?
% comment sont évalués les méthodes explorées? contexte d'utilisation et métriques d'évaluation?
% comment sont représentés les documents?

La similarité entre texte est indispensable pour, entre autres applications, retrouver des textes similaires et catégoriser les documents. Parmi les questions liées à l'estimation automatique de la similarité entre documents, on distingue: la sémantique experte qui fonde cette similarité, sa métrique de mesure, la représentation des documents, le contexte d'exploitation et les métriques d'évaluation. La  mesure de similarité doit être définie de sorte à rapprocher ou éloigner les documents suivant l'aspect sémantique qu'on veut révéler. \citet{nair2018judgsimassorule} arrivent à exploiter les citations de lois et précédents car les jugements du droit commun comprennent des liens aux décisions d'affaires similaires antérieures. Ils analyse le réseaux de citations d'un corpus de 597 documents, à l'aide de règle d'association générées par pour . Certaines métriques traditionnels, comme la distance cosinus \citep{thenmozhi2017legalprecedretriev}, ont été utilisées sur les décisions judiciaires mais pas toujours avec succès. La raison peut venir notamment de la représentation des textes qui doit accentuer l'aspect sous-jacent de la similarité. \citet{ma2018wmdchinesecase} proposent donc de s'appuyer sur une ontologie des concepts et relations du corpus judiciaire. L'idée est de calculer la similarité sur un résumé du texte qui compacte le texte uniquement sur les aspects pertinents. Cette méthode permet ainsi de mieux capter la sémantique pure des jugements, d'avoir une meilleure précision, de réduire la complexité temporelle inhérente à l'exploitation de long document notamment avec la métrique WMD \citep{kusner2015wordmoverdist}.

\cite{kumar2011judgmentsimilarity, nair2018judgsimassorule, branting2017autoJudiDocAnalysis}

Les aspects influençant l'estimation de la similarité: l'abstraction à l'information sur laquelle repose la nature de la similarité, la représentation des documents, et la métrique de similarité / dis-similarité employé

\section{Conclusion}
\label{sec:literature:conclusion}
%\subsection{Types d'approches appliquées}
En résumé, l'analyse des données textuelles juridiques a pour but la structuration des documents et l'organisation sémantique de corpus. Le domaine est très actif depuis déjà plusieurs décennies, au point où des librairies de développement, spécifiques au domaine, commencent à voir le jour \citep{bommarito2018lexnlp}.
Un grand nombre de travaux défendent des études de faisabilité. Ils se limitent à appliquer une approche basique d'analyse de données sur une faible quantité de données.

On remarque que le concepteur investit un minimum d'intuitivité ou d'ingénierie que ce soit pour la définition des caractéristiques pour les modèles à apprentissage automatique, ou pour définir les règles pour les méthodes à base de règles ou grammaire. 

%\subsection{Évaluation et qualité}
 Certains articles récents font l'effort de reporter des résultats quantifiés d'évaluation de l'accord inter-annotateurs et du système développé (par ex. \citep{Shulayeva2017recognfactprincip}). Les résultats de la classification des documents sont généralement obtenus à l'issu une validation croisée \cite{Sulea2017predictareadecision,sulea2017legalEnsSVM,Aletras2016predictDecisionECHR}. D'autres ne s'appuient que sur des captures d'écran de l'outil développé pour défendre l'architecture proposée (par ex. \citep{wyner2010extractlegalelts,Waltl2016lexia}).

%\cite{Galgani2015lexa} montrent qu'il est possible en un temps raisonnable d'annoter des texte

 % INCLUDE: related work
\chapter{Sectionnement et Annotation d'entités}
\label{chap:structuration}

\section{Introduction}
\label{sec:structuration:motivation}

\textcolor{red}{GENERALISER LE NOMBRE DE SECTION ET PRESENTER LES RESULTATS PUBLIES COMME ETANT UN CAS PARTICULIER}

Ce chapitre traite de la détection de sections et d'entités dans les décisions jurisprudentielles françaises. Ces décisions sont des documents non structurés qui partagent généralement la même structure qu'on peut définir par les sections: \textit{entête}, \textit{exposé du litige}, \textit{motifs}, \textit{dispositif}. Chaque section comprend des information sur l'affaire: 
\begin{enumerate}
\item la section ENTETE contient diverses métadonnées de référence (date, lieu, identifiant, parties impliquées, juges, affaires antérieurs liés, ...); 
\item La section CORPS qu'on peut diviser de manière plus fine pour distinguer:
\begin{itemize}
\item la section LITIGE comprend une description des faits, des procédures (jugements antérieurs, appel, assignations, ...), des demandes et raisonnements des parties; 
\item la section MOTIFS détaille le raisonnement et les arguments des juges;
\end{itemize}
\item la section DISPOSITIF est la décision finale qui résume la réponse des juges aux différentes demandes des parties.
\end{enumerate}

 Même si toutes les décisions respectent cette organisation, la structure à l'intérieur des sections peut varier d'un document à l'autre. Compte tenu de la répartition très sémantique des informations entre les sections, il nous a paru logique de détecter les sections dans un premier temps pour mieux organiser les différentes tâches d'extraction. Par la suite, des données sur les entités ou les demandes par exemple peuvent être plus facilement extraites en fonction des sections où elles se retrouvent généralement. Nous nous focalisons en particulier ici sur la détection d'entités telles que la date à laquelle le jugement a été prononcé, le type de juridiction, sa localisation (ville), les noms de juges, des parties et des avocats. La Table \ref{p4_relevantinfo} liste les différentes entités ciblées et fournit des exemples illustrant la forme de leurs mentions dans les arrêts français (décisions de juridictions de second degré).

\begin{table}[!ht]
\scriptsize
\begin{tabular}[c]{|p{0.17\textwidth}|c|p{0.37\textwidth}|cc|}
\hline
\textbf{Entités} & \textbf{Label} & \textbf{Exemples} & \multicolumn{2}{c|}{\textbf{\#mentions}$^a$}\\
  & & & \textbf{Médiane}$^b$& \textbf{Total}$^c$ \\ \hline
Numéro de registre & \textbf{rg} & "10/02324", "60/JAF/09" & 3 & 1318\\ \hline
Ville & \textbf{ville}& "NÎMES", "Agen", "Toulouse" & 3 & 1304\\ \hline
Juridiction & \textbf{juridiction} & "COUR D'APPEL" & 3 & 1308\\ \hline
Formation & \textbf{formation} & "1re chambre", "Chambre économique" & 2 &  1245\\ \hline
Date de prononcé de la décision & \textbf{date} & "01 MARS 2012", "15/04/2014" & 3 & 1590\\ \hline
Appelant & \textbf{appelant} & "SARL K.", "Syndicat ...", "Mme X ..." & 2 & 1336 \\ \hline
Intimé & \textbf{intime} & - // - & 3 & 1933 \\ \hline
Intervenant & \textbf{intervenant} & - // - & 0 & 51 \\ \hline
Avocat & \textbf{avocat} & "Me Dominique A., avocat au barreau de Papeete" & 3 & 2313\\ \hline
Juge & \textbf{juge} & "Monsieur André R.", "Mme BOUSQUEL" & 4 & 2089\\ \hline
Fonction de juge & \textbf{fonction} & "Conseiller", "Président" & 4 & 2062\\ \hline
Norme & \textbf{norme} & "l' article 700 NCPC", "articles 901 et 903" & 12 & 7641 \\ \hline
Argent & \textbf{argent} & "un euro", "" & 14$^*$ & 1777$^*$ \\ \hline
\noalign{\smallskip}\hline\noalign{\smallskip}
Non-entité & \textbf{O} & \textit{mot ne faisant partie d'aucune mention d'entité} & - & -\\ \hline
\end{tabular} 

$^a$ nombre de mentions d'entités dans le corpus annoté pour les expérimentations

$^b$ nombre médian de mentions par document dans le corpus annoté

$^c$ nombre total d'occurrences dans le corpus annoté

$^*$ Les statistiques sur les sommes d'argent ne concernent que 100 documents annotés (max=106, min=1, moyenne=17.77), contre 500 documents pour les autres entités.
\caption{Entités et labels correspondant utilisés pour labeliser leurs mots. }\label{p4_relevantinfo}
\end{table}


\section{Les décisions de justices françaises}
\label{sec:structuration:probleme}
\subsection{Structure}
Vu la sémantique ou le rôle des différentes sections, il est évident qu'elles sont séparées par des marqueurs bien précis. Une approche intuitive consiste par exemple à définir un algorithme capable de reconnaitre ces marqueurs de transitions à travers des expressions régulières. Cependant, les marqueurs utilisés ne sont pas standards car il n'existe pas de modèle stricte utilisé par toutes les juridictions. par conséquent, les marqueurs de transitions sont souvent différents d'une décision à l'autre et peuvent être des titres ou des motifs à base de symboles (astérisques, tirets, etc.). Il arrive parfois que la transition soit implicite et qu'on ne s'en rende compte que par la forme des lignes, au cours de la lecture. Même les marqueurs explicites sont hétérogènes. D'une part en cas d'utilisation de titres par exemple, la transition de l'entête à l'exposé du litige peut être indiquée par des titres comme "\textit{Exposé}", "\textit{FAITS ET PROCÉDURES}", "\textit{Exposé de l’affaire}", "\textit{Exposé des faits}", etc. Quant au dispositif, il est introduit généralement par l'expression "\textit{PAR CES MOTIFS}" avec souvent quelques variantes qui peuvent être très simples (par ex. "\textit{Par Ces Motifs}") ou exceptionnelles (par ex. "\textit{P A R C E S M O T I F S :}"). Dans certaines décisions, cet expression est remplacée par d'autres expressions comme "\textit{DECISION}", "\textit{DISPOSITIF}", "\textit{LA COUR}", etc. 
D'autre part lors de l'utilisation de symboles, il arrive qu'un même motif sépare différentes sections et même des paragraphes dans le même document.

Une hétérogénéité similaire apparait pour les entités. Les noms de parties et d'avocats sont généralement placés après un mot particulier comme  "\textit{APPELANTS}" ou "\textit{DEMANDEUR}" pour les demandeurs (appelants en juridiction de 2e degré), "\textit{INTIMES}" ou "\textit{DEFENDEUR}" pour les défendeurs (ou intimés), and "\textit{INTERVENANTS}" pour les intervenants. Les noms des individus, sociétés et lieux commencent par une lettre majuscule ou sont entièrement en majuscule. Cependant, certains mots communs peuvent apparaitre aussi en majuscule (par ex. \textit{APPELANTS}, \textit{DÉBATS}, \textit{ORDONNANCE DE CLÔTURE}). Les entités peuvent contenir des chiffres (identifiant, dates, ...), des caractères spéciaux ("/", "-"), des initiaux ou abréviations.  Dans l'entête, les entités apparaissent généralement dans le même ordre (par ex. les appelants avant les intimés, les intimés avant les intervenants). Cependant, plusieurs types d'entités apparaissent dans l'entête, contrairement aux autres sections où seules les normes nous intéressent dans cette étude. L'entête est aussi mieux structurée que les autres sections même si sa structure peut différer entre deux documents.

\subsection{Aquisition}
Les décisions sont disponibles auprès des juridictions mais on note qu'un certain nombre d'entre elles est accessible en ligne. On les y retrouve sous divers formats dont par exemple .rtf sur \url{http://legifrance.gouv.fr}, ou .doc(x) et .txt sur le site web de LexisNexis d'où proviennent les documents de notre étude. Chaque document téléchargé de LexisNexis contient une ou plusieurs décisions de justice. De plus, LexisNexis n'autorise le téléchargement simultané qu'un nombre très faible de décisions qui décroit dans le temps (nous avons remarqué des limites de 100, puis 20, et ensuite 10 au cours de ces 3 dernières années). Par contre, l'identification et le service des documents sur Legifrance permet de pouvoir recolter plus rapidement les documents à l'aide par ex. d'une simple routine de téléchargement itératif.




\section{Reconnaissance d'entités par étiquetage de séquence}
\label{sec:structuration:biblio}

\subsection{Approches générales de détection d'entités}

Quatre catégories d'approches de détection d'entités ont été distinguées \citep{chau2002nerwithNN}:

\begin{itemize}
\item Les \textbf{systèmes à recherche lexicale} sont conçus sur la base d'une liste d'entités précédemment connues, avec leurs synonymes dans le domaine d'intérêt. Par exemple, dans le domaine juridique, un lexique pourrait contenir les identifiant de règles juridiques et les noms des juges. La liste des entités peut être manuscrite par des experts ou apprise à partir d'un ensemble de données étiquetées (phase d'apprentissage). Cependant, il s'avère très difficile de maintenir une telle liste car le domaine pourrait changer régulièrement (nouvelles lois par ex.). De plus, les mentions d'entités peuvent avoir plusieurs variantes. Par exemple, la même règle "Article 700 du code de procédure civile" peut \^etre citée seule et en entier (\textit{article 700 du code de procédure civile}), ou abrégée (\textit{article 700 CPC}), ou encore combinée avec d'autres règles (\textit{articles 700 et 699 du code de procédure civile}). Ces problèmes, y compris les ambiguïtés (par exemple, différentes entités utilisant les mêmes mots), ont limité les premiers systèmes \citep{palmer1997learnedLookup}.

\item Les \textbf{systèmes à base de règles} sur des règles qui décrivent suffisamment, contextuellement, structurellement ou lexicalement, la diversité des mentions d'entités. Ils sont avantageux parce que leurs erreurs sont facilement explicables. La définition manuelle de règles exige malheureusement des efforts considérables, en particulier pour les grands corpus. De plus, un ensemble donné de règles est difficilement réutilisable dans d'autres domaines. Cependant, quelques approches adaptatives ont été conçues pour surmonter ces limites tout en bénéficiant toujours de l'explicabilité du comportement des systèmes à base de règles \citep{siniakov2008gropusrulebased,chiticariu2010adaptativerulebased}.

\item Les \textbf{systèmes statistiques} adaptent les modèles statistiques de langage, issus typiquement des méthodes de compression de texte, pour détecter les entités. Par exemple, \citet{witten1999languagemodel} ont adapté les schémas de compression nommés "Prédiction par Correspondance Partielle".

\item Les \textbf{systèmes basés sur l'apprentissage automatique} exécutent des classifieurs multi-classes sur des segments de texte. Par exemple, un classifieur traditionnel comme le classifieur bayésien naïf peut être entrainer pour détecter les noms de gènes en classifiant les mots du documents à partir d'un ensemble de descripteurs définis manuellement \citep{persson2012nbbioner}. Pour détecter les entités, les algorithmes d'étiquetage de séquence tels que le CRF quant à eux classifient les segments de texte tout en modélisant les transitions entre les labels \citep{finkel2005stanfordcrfner}. Dans ce registre, les architectures d'apprentissage profond réalisent actuellement les meilleures performances sur de multiples tâche d'extraction d'information en général et de reconnaissance d'entités nommées en particulier \citep{lample2016nnner}.
\end{itemize}
Certains travaux ont combiné diverses approches pour extraire les entités à partir de documents juridiques,  par exemple,  par la description de l'information contextuelle en utilisant des règles pour répondre au problème d'ambigüité des méthodes à recherche lexicale \citep{mikheev1999NERlexicalWithRules,hanisch2005prominer}. 

\subsection{Détection d'entités dans des documents juridiques}

Les modèles HMM et CRF ont été utilisés dans de multiples travaux pour reconnaître des entités juridiques. Ils peuvent être combinés à d'autres approches dans un système global. Après avoir segmenter les documents à l'aide d'un modèle CRF, \citet{dozier2010legalnerr} ont combiné plusieurs approches pour reconnaître des entités dans les décisions de la cour suprême des Etats-Unis. Ils ont définis des détecteurs distincts à base de règles pour identifier séparémment la juridiction (zone géographique), le type de document, et les noms des juges, en plus de l'introduction d'une recherche lexicale pour détecter la cour, ainsi qu'un classifieur entrainé pour reconnaître le titre. Ces différents détecteurs ont atteint des performances prometteuses, mais avec des rappels limités entre $ 72 \% $ et $ 87 \% $. \citep{cardellino2017legalNERCL} quant à eux ont utilisé le CRF et les réseaux de neurones (\textcolor{red}{OÙ pays?}). Les basses performances qu'ils rapportent pour l'extraction dans les jugements illustre bien la difficulté de la détection d'entités juridiques. Plus récemment encore, \citet{andrew2018legalNerAndRelation} obtiennent de bons résultats en combinant l'extraction d'entités non-juridiques par CRF à l'extraction de relations par une grammaire GATE JAPE \citep{thakker2009gatejape} sur des décisions du Luxembourg rédigées en français.

La comparaison avec d'autres approches démontre bien que les modèles probabilistes atteignent de très bonnes performances lors de l'extraction d'information dand les documents juridiques. Par exemple, le HMM a été comparé à l'Algorithme de Perceptron à Marges Inégales (PAUM) \citep{li2002PAUM} pour reconnaître les institutions et références d'autres décisions de justice, ainsi que les citations d'actes juridiques (loi, contrat, etc.) dans les décisions judiciaires de la République Tchèque \citep{Kriz2014nerinczechdecisions}. Les deux modèles ont données de bonnes performances avec des scores F1 de $ 89 \% $ et $ 97 \% $ pour le HMM utilisant les trigrammes comme descripteurs de mots, et des scores F1 de $ 87 \% $ et $ 97 \% $ pour le PAUM en utilisant des 5-grammes de lemmes et les rôles grammaticaux (\textit{Part-Of-Speech tag}) comme descripteurs. 

\subsection{Modèles d'étiquetage de texte}

Considérons un texte T comme étant une séquence d'observations $t_{1:n}$, avec chaque $t_i$ étant un segment de texte (mots, ligne, phrase, etc.). En considérant une collection de labels, l'étiquetage de T consiste à affecter à les labels appropriés à chaque $t_i$. La segmentation de T est un étiquetage particulier qui implique de découper T en des groupes qui ne se chevauchent pas (des partitions), tels que les segments d'un groupe constituent nécessairement une sous-séquence de T. En d'autres termes, segmenter T revient à labeliser ses segments en considérant une contrainte particulière (\textcolor{red}{Laquelle?}). Parmi les multiples modèles d'étiquetage de séquences, nous nous sommes proposés d'étudier les performances de quelques modèles établis décrits ci-après.
\subsubsection{HMM}
Un modèle HMM est une machine à état fini défini par un ensemble d'états $ \lbrace s_1, s_2, ..., s_m \rbrace $. Un modèle HMM a pour fonction d'affecter une probabilité jointe 
$ P (T , L) = \prod\limits_i P(l_i \vert l_{i-1})P(T \vert l_i)$  à des paires de séquences d'observations $ T = t_{1: n} $ et de séquence de labels $ L = l_{1:n} $. Etant donnée qu'un HMM est un modèle génératif, chaque label $l_i$ correspond à l'état $s_j$ dans lequel la machine a généré l'observation $t_i$. Il y a autant de labels possibles que d'états. Le processus de labelisation de T consiste à déterminer la séquence de labels $ L^* $ qui maximise la probabilité jointe ($L^* = \arg \max\limits_L P(T, L)$). Une évaluation de toutes les séquences possibles de labels est nécessaire pour déterminer celle qui convient mieux à $ T $. Pour éviter la complexité exponentielle $ O(m^n)$ d'une telle approche, $n$ étant la longueur de la séquence et $m$ le nombre de labels possibles, le processus d'étiquetage utilise généralement l'algorithme de décodage Viterbi \citep{viterbi1967viterbi} qui est basé sur la programmation dynamique. Cette algorithme utilise les paramètres du HMM estimés par apprentissage sur un corpus de textes annotés manuellement:
\begin{itemize}
\item Un ensemble d'états $ \lbrace s_1, s_2, ..., s_m \rbrace $ et un alphabet $ \lbrace o_1, o_2, ..., o_k \rbrace $
\item La probabilité que $ s_j $ génère la première observation $ \pi(s_j), \forall j \in [1 .. m] $
\item La distribution de probabilité de transition $ P (s_i\vert s_j),  \forall i,j \in [1 .. m] $
\item La distribution de probabilité de d'émission $ P(o_i\vert s_j), \forall i \in [1 .. k], \forall j \in [1 .. m]$
\end{itemize}

Les probabilités de transition et d'émission peuvent être inférer en utilisant une méthode de maximum de vraisemblance comme l'algorithme d'espérance maximale. L'algorithme Baum-Welch \citep{welch2003baumwelch} en est une spécification conçu spécialement pour le HMM. L'avantage du HMM réside dans sa simplicité et sa vitesse d'entrainement. Par ailleurs, il est difficile de représenter de multiples descripteurs interactifs pour les segments de texte, tout comme de modéliser la dépendance entre des observations distantes parce que l'hypothèse d'indépendance entre observations est très restrictive (i.e. l'état courant dépend uniquement des état précédents et de l'observation courante). \citet{rabiner1989tutorial} fournit plus de détails sur le HMM.

\subsubsection{CRF}
\label{sec:structuration:biblio:CRF}

Même si l'algorithme Viterbi est aussi utilisé pour appliquer le CRF à l'étiquetage de séquence, les structures du CRF et du HMM diffèrent. Au lieu de maximiser la probabilité jointe $ P(L, T)$ comme le HMM, un CRF \citep{lafferty2001crfie} cherche la séquence de labels $L^*$ qui maximise la probabilité conditionnelle suivante: $$P(L|T) = \frac{1}{Z}\exp \left(\sum\limits_{i=1}^n\sum\limits_{j=1}^F \lambda_j f_j(l_{i-1},l_i,t_{1:n},i)\right)$$ où $Z$ est le facteur de normalisation. Les fonctions potentielles $f(\cdot)$ sont les caractéristiques utilisés par les modèles CRF. Deux types de fonctions caractéristiques sont définies: les caractéristiques de transition qui dépendent des labels aux positions courantes et précédentes ($l_{i-1}$ and $ l_{i}$ respectivement) et de $T$; et les caractéristiques d'état qui sont des fonctions de l'état courant $ l_{i} $ et de la séquence $ T $. Ces fonctions $f(\cdot)$ sont définies à l'aide soit par de fonctions à valeur binaire ou réelle $b(T,i)$ qui combine les descripteurs d'une positiond'une position $i$ dans $T$ \citep{Wallach2004crfintro}. Pour labéliser les références aux règles de loi par exemple, un CRF pourrait inclure par exemple les fonctions potentielles pour labéliser "\textit{700}" dans ce contexte "\textit{... l'article 700 du code de procédure civile...}":
{\scriptsize
\[f_1(l_{i-1},l_i,t_{1:n},i) = \left\lbrace \begin{array}{ll}
b_1(T,i) & \text{si } l_{i-1} = \text{NORME} \wedge l_i = \text{NORME} \\
0 & \text{otherwise}
\end{array} \right.\]
\[f_2(l_{i-1},l_i,t_{1:n},i) = \left\lbrace \begin{array}{ll}
b_2(T,i) & \text{si }l_i = \text{NORME} \\
0 & \text{otherwise}
\end{array} \right.\]
avec
\[b_1(T,i) = \left\lbrace \begin{array}{ll}
1 & \text{si } (t_{i-1} =\text{article) }\wedge (POS_{i-1}=\text{NOM}) \\&  \wedge  (NP1_{i-1}=\text{<unknown>)} \wedge (NS1_{i-1}=\text{@card@)} \\
0 & \text{otherwise} 
\end{array} \right.\]
\[b_2(T,i) = \left\lbrace \begin{array}{ll}
1 & \text{si } (t_i =\text{700) }\wedge (POS_i=\text{NUM})  \wedge (NP1_i=\text{article)} 
\wedge (NS1_i=\text{code)} \\
0 & \text{otherwise}
\end{array} \right.\]
}
$t_i$ étant une observation dans $T$, POS étant le rôle grammatical de $t_i$ (NUM = valeur numerique, NOM=nom), et NP1 et NS1 sont les lemmes des mots avant et après $t_i$, respectivement. Les symboles \textit{<unknown>} et \textit{@card@} encode les lemmes inconnus et les lemmes de numbres respectivement. Pouvant être activées au même moment, les fonctions $f_1$ et $f_2$ définissent des descripteurs se chevauchant. Avec plusieurs fonctions activées, la croyance dans le fait que $l_i = NORME$ est renforcée par la somme $\lambda_1 + \lambda_2$ des poids  des functions activées \citep{Zhu2010CRFlecture}.  Un modèle CRF emploie une fonction $f_j$ lorsuqe ses conditions sont satisfaites et $\lambda_j > 0$. Les diverse fonctions pondérées $f_j$ sont définies par des descripteurs caractérisant le texte et les labels des données d'entrainement. La phase d'entrainement consiste principalement à estimer le vecteur de paramètres $\lambda = (\lambda_1,...,\lambda_F)$ à partir des de textes annotés manuellement $ \lbrace (T_1, L_1), ..., (T_M, L_M) \rbrace $, $ T_k $ étant un texte et $ L_k $ la séquence de labels correspondande. La valeur optimal retenue pour estimer de $\lambda$ est celle maximisant la fonction objectif   
$\sum\limits_ {k = 1} ^ M \log P (L_k \vert T_k) $ sur les données d'entrainement. En général, outre le maximum de vraisemblance, cette optimisation est résolue à l'aide de l'algorithme de descente du gradient dont l'exécution peut-être accélérée à l'aide de l'algorithme L-BFGS \citep{liu1989l-bfgs}.

\subsubsection{Extensions du CRF par les réseaux de neurones: BiLSTM-CRF vs. BiLSTM-CNN-CRF}
2015 https://arxiv.org/pdf/1508.01991.pdf

https://github.com/UKPLab/emnlp2017-bilstm-cnn-crf

dnn crf : https://pdfs.semanticscholar.org/c322/7702dd212965157a615332f3dd78b0f11b5e.pdf

https://towardsdatascience.com/conditional-random-field-tutorial-in-pytorch-ca0d04499463

Deux architectures de réseaux de neuronnes réalisent actuellement les meilleures performances en matière de détection d'entités nommées. Il s'agit du modèle LSTM-CRF proposé par \citet{lample2016nnner} et du LSTM-CNN-CRF de \citet{ma2016lstm-cnns-crf}. On pourrait résumer ces architectures en trois phases. Dans un premier temps, les segments de textes (mots) sont représentés vectoriellement en concatenant 2 vecteurs de plongement sémantique: l'un issu de l'apprentissage morphologique du mot à partir de ces caractères, et l'autre issu de l'apprentissage du contexte général d'occurrence du mot (par ex. CBOW, Skip-gram, Glove, ...). Lors de la seconde phase, deux couches de cellules LSTM enchainées permettent de modéliser le contexte à droite et à gauche de chaque mots du texte. La dernière phase détermine la séquence de label la plus probable pour le texte à l'aide d'une implémentation d'un modèle CRF sous forme de réseau de neuronnes. Le CRF reçoit en entrée la concaténation des contextes à droite et à gauche des mots.

\subsection{Définition des descripteurs d'éléments atomiques}
\textcolor{red}{Pourquoi? Quoi ? et Comment?}

La représentation des éléments atomiques occupe une place importante dans le bon fonctionnement des modèles décrits précédemment. Si les architectures basée sur les LSTM combinent une représentation indépendente de la tâche à des représentations de forme et de contexte inférerées en interne, la définition des descripteurs est plus arbitraire chez les modèles probabilistes. Il s'agit généralement de caractéristiques booléennes comme celles mentionnées dans la description du CRF (\ref{sec:structuration:biblio:CRF}). Par exemple, indiquer si un mot débute par une lettre majuscule permet de mettre en évidence les noms propres. La définition de telles caractéristiques consiste ainsi à fournir au modèle des indices l'aidant à mieux distinguer les différents types d'entités par la forme et le contexte d'occurrence des éléments atomiques. 

Etant donné que les descripteurs dépendent généralement de l'intuition du modélisateur, il est difficile mais nécessaire d'identifier des descripteurs appropriés. Les nombreux travaux de reconnaissance d'entités nommés inspirent des descripteurs candidats qui ont déjà été expérimentés avec succès. De plus, il est possible d'appliquer une sélection des descripteurs afin de réduire tous les candidats définis à un sous-ensemble plus optimal. Cette réduction a pour objectif d'améliorer les performances d'étiquetage, d'accélérer les phases d'extraction de caractéristiques, d'entrainement et de prédiction, et de fournir une meilleure compréhension du comportement des modèles. 

Les méthodes de sélection utilisées dans la littérature sont généralement basées sur des algorithmes Filtre (\textit{filters}) qui associent à chaque descripteur un score, ou des algorithmes Symbiose (\textit{wrappers}) qui préparent, évaluent et comparent des combinaisons de descripteurs. D'une part, les filtres ont l'avantage d'être rapide à exécuter par rapport aux emballages beaucoup plus lents. D'autre part, les filtres sont moins performants car ils ne permettent pas d'éviter les cas de redondances et ne prennent pas en compte l'effet de la combinaison de variables. C'est ainsi que les filtres peuvent être utilisés avant les emballages afin de comparer uniquement les combinaisons des caractéristiques significatives. Au départ proposés et utilisés en classification multi-dimmensionnelle, les algorithmes de sélections de caractéristiques ont été appliqués avec succès pour l'extraction d'entités. \citet{klinger2009FeaturefilterCRF} ont 

\subsection{Sélection du schéma d'étiquetage}
Nous traitons d'entités dont les mentions comprennent un ou plusieurs éléments atomiques. Pour améliorer les performance d'un modèle d'étiquetage, certaines parties des entités peuvent être mise en évidence à travers une représentation appropriée de segment. Nous comparons dans cette étude quelques schémas d'étiquetage dont certains sont décrits par \citet{konkol2015tagModel}. Le schéma IO utilisé par défaut ne met l'accent sur aucune partie de l'entité et affecte le même label à chaque élément de l'entité. D'autres schémas distinguent soit le premier élément (BIO), soit le dernier (IEO), soit les deux (BIEO). Les schémas IEO2 BIO2 sont des variantes respectives des schémas IEO et BIO. Elles utilisent resp. les préfixes E- et B- pour étiqueter les entités à mot unique, contrairement à IEO1 et BIO1 qui utilisent plutôt le préfixe I-. Le modèle BIEO est souvent étendu à BIESO (ou BILOU) dans le cas où on souhaite distinguer les mentions à un seul élément (par ex. ville ou numéro RG). Les lettres des sigles de ces modèles servent de préfixes aux labels et portent la signification suivante:

\begin{itemize}
\item B - "\textit{beginning}": début;
\item I - "\textit{inside}": intérieur;
\item E (ou L, ou M) - "\textit{end}" ou "\textit{last}" ou "\textit{middle}": fin;
\item S (ou U, ou W) - "\textit{single}" ou "\textit{unit}" ou "\textit{whole}": singleton;
\item O - "\textit{outside}": hors de toute entité.
\end{itemize}

La figure \ref{p4_sample-tagmod} illustrate l'utilisation des ces différents modèles sur un extrait de décision de justice:
\begin{figure}[!h]
\tiny
\begin{tabular}{l|ccccccccccc}
 & \textit{composée} & \textit{de} & \textit{Madame} & \textit{Martine} & \textit{JEAN} & , & \textit{Président} & \textit{de} & \textit{chambre} & , & \textit{de} \\ 
IO & O & O & I-JUGE & I-JUGE & I-JUGE & O & I-FONCTION & I-FONCTION & I-FONCTION & O & O \\
BIO & O & O & B-JUGE & I-JUGE & I-JUGE & O & B-FONCTION & I-FONCTION & I-FONCTION & O & O \\
IEO & O & O & I-JUGE & I-JUGE & E-JUGE & O & I-FONCTION & I-FONCTION & E-FONCTION & O & O \\
BIEO & O & O & B-JUGE & I-JUGE & E-JUGE & O & B-FONCTION & I-FONCTION & E-FONCTION & O & O \\
\end{tabular}
\caption{Illustration de différents schémas d'étiquetage}\label{p4_sample-tagmod}
\end{figure}

Il est possible d'aller plus loin en mettant l'accent sur les mots avant  (O-JUGE) et après (JUGE-O) l'entité (JUGE par exemple) et en indicant le début (BOS-O, \textit{begininning of sentence}) et la fin (O-EOS, \textit{end of sentence}) du texte. Le format ainsi obtenu est appelé BMEWO+ \citep{baldwin2009bmewo}.

Un autre intérêt très important de modèle plus complexes que IO est de pouvoir distinguer des entités qui se suivent sans être séparées d'une ponctuation visible. Cet aspect est notamment important dans les décisions de justice par exemple lorsque des noms de parties soient listés dans la section ENTETE en n'étant séparés que d'un simple retour à la ligne (\textcolor{red}{Illustration?}).

\section{Approches proposées}
\label{sec:structuration:proposition}
\subsection{Chaine d'extraction et exploration de caractéristiques pertinentes}
\subsubsection{Architecture}
\begin{figure}[!h]
%\sidecaption
\centering
\includegraphics [width=0.45\textwidth]{structuration-preprocess.png}
\includegraphics [width=0.45\textwidth]{structuration-pipeline-application.png}

{\scriptsize Après la collecte et le pretraitement des documents, l'étiqueteur de ligne est d'abord appliqué pour détecter les sections, puis les étiqueteurs d'entités peuvent être appliqués simultanement dans differentes sections.}
\caption{Application des modèles entrainés d'étiquetage de section et entités.}\label{p4_archAppli}
\end{figure}
Nous proposons de travailler uniquement avec le contenu textuel des documents. Ce contenu est extrait des documents téléchargés en eliminant les éléments inutiles, principalement des espaces vides. Ces éléments sont typiques des documents formatés (.rtf, .doc(x), .pdf). Ils ne fournissent pas une indication standard sur le début des sections. Le choix de ne pas exploiter le formatage des documents permet d'avoir à gérer un nombre plus faible de diversités entre les textes tout appliquant le même processus de traitement à tout document indépendament de sont format d'origine. Une simple architecture d'étiquetage de sections et d'entités juridiques a été conçu avec cet uniformisation des documents comme point d'entrée. Ainsi, les documents sont collectés pui prétraités suivant leur format d'origine (extraction du texte et séparation des décisions apparaissant dans le même document).  Ensuite, après le sectionnement des décisions, les entitées sont identifiées dans les différentes sections. Cette section décrit quelques aspects de conception à prendre en compte dans le but d'obtenir de bons résultats à partir d'un tel système.

\begin{figure}[!h]
%\sidecaption
\centering
\includegraphics [width=\textwidth]{structuration-training.png}
\caption{Entrainement des modèles.}\label{fig:structuration:training}
\end{figure}


L'entrainement des modèles sur les exemples manuellement annotés est aussi décomposé en étapes. Nous proposons de sélectionner le schéma d'étiquetage, puis de sélectionner les sous-ensembles minimaux de caractéristiques manuellement définies, avant d'entrainer les modèles HMM et CRF (Figure \ref{fig:structuration:training}). La sélection de caractéristiques ne concerne pas l'implémentation en réseau de neuronnes du CRF.

\subsubsection{Descripteurs candidats}
\begin{table}[b]
    \centering
    \begin{tabular}{c|c}
         &  \\
         & 
    \end{tabular}
    \caption{Caractéristiques candidates pour l'identification de sections et d'entitées}
    \label{tab:my_label}
\end{table}

\subsection{Réseaux de neuronnes double-tache}
Découpage des 


\section{Expérimentations et discussions}
L'objectif de cette section est de discuter des différents aspects d'optimisation des performance des modèles CRF, HMM et BiLSTM-CRF, et de les comparer lorsqu'ils sont entrainés dans les conditions optimales déterminées empiriquement. Il est question de discuter l'effet des caractéristiques définies, de comparer des algorithmes de sélection de caractéristiques et des schémas d'étiquetage, de vérifier l'effet de l'augmentation des données d'entrainement sur la performance des modèles.

\label{sec:structuration:experimentations}
\subsection{Configuration des expérimentations}
\subsubsection{Annotation des données de référence}
Pour évaluer les méthodes de TAL, \citet{xiao2010corpuscreation} (\textcolor{red}{démontre l'importance...}) suggère de choisir un jeu de données échantillon suffisant en assurant l'équilibre dans la variété des données et la représentativité du langage. Cette en suivant cette recommendation que nous avons prétraité et annoté manuellement un ensemble de 503 documents à l'aide de la plateforme GATE Developer\footnote{https://gate.ac.uk/family/developer.html}. Cet outil permet de marquer avec le pointeur de la souris les passage à annoter; ce qui facilite l'annotation manuelle. Des balises XML sont rajoutées en arrières plan dans le document pour marquer les passages sélectionnés.

Chaque document comprend en moyenne 262,257 lignes et 3955,215 mots. La représentativité du corpus a été simulée en choisissant aléatoirement les décisions tout en faisant variés la ville et l'année. Les deux dernières colonnes du Tableau \ref{p4_relevantinfo} présente la distribution des entités labelisées dans le jeu de données. En se basant sur un sous-ensemble de 13 documents labelisés par 2 annotateurs différents, nous avons calculé des taux d'inter-agrément en utilisant la statistique Kapp de Cohen. Ces mesures d'inter-agrément ont été calculées au niveau des caractère parce que certains mots peuvent être coupés par des annotations incorrectes. (par ex. \textit{<juridiction>cour d'appe</juridiction>\underline{l}} vs. \textit{<juridiction>cour d'appe\underline{l}</juridiction>}), ou les annotateurs pourraient ne pas être d'accord un apostrophe doit être inclu ou pas (par ex. \textit{ \underline{l'}<norme>article 700} vs. \textit{ <norme>\underline{l'}article 700}). Les taux de Kappa de 0,705 et 0,974 ont été obtenu pour l'annotation des entités et des sections respectivement. D'après la catégorisation de \citet{viera2005kappa}, le niveau d'agrément observé est \textit{substentiel} pour les entités (0,61-0,80) et \textit{presque parfait} pour les sections (0,81 - 0,99).

\subsubsection{Protocole d'évaluation}
Nous avons utilisé la Précision, le rappel et la F1-mesure comme mesures d'évaluation car elles sont généralement utilisées comme référence en extraction d'information. Nous comparons deux niveaux de performances: i) à quel point le système est en mesure de bien labeliser chaque élément/mot d'entité (niveau-mot), et ii) à quel point le modèle labélise entièrement une entité (niveau-entité). Nous présentons aussi des performances au niveau micro i.e. les performances du classifieur en général sans distinction des classes. A tous les niveau d'évaluation, la F1-mesure se calcule à l'aide de la formule \ref{NER-f1-mesure}.  
\begin{equation}\label{NER-f1-mesure}
F1 = 2 \times \frac{Precision \times Rappel} {Precision + Rappel}
\end{equation}
%La précision et le rappel quant à eux se calculent suivant les formules
% \begin{equation}\label{NER-precision}
% Precision = \frac{TP}{}
% \end{equation}

\vspace{0.3cm}

\noindent \underline{\textbf{Evaluation au niveau atomique (\textit{token-level)}}}: Cette évaluation mésure la capacité d'un modèle à labeliser les éléments atomiques des entités à identifier. Les valeurs de précision and rappel sont calculés sur les données de test pour chaque label $l$ comme suit:
\[Precision_l = \frac{\text{nombre d'éléments correctement labelisés par le modèle avec } l} {\text{nombre d'éléments labelisés par le modèle avec } l}\]
\[Rappel_l = \frac{\text{nombre d'éléments correctement labelisés par le modèle avec } l} {\text{nombre d'éléments manuellement labelisés par le modèle avec } l}\]

\vspace{0.3cm}

\noindent \underline{\textbf{Evaluation au niveau entité (\textit{entity-level})}}: Cette évaluation mésure le taux d'entités parfaitement identifiées c'est-à-dire seulement ceux dont les éléments atomiques ont été tous correctement labélisés. Les valeurs de précision and rappel sont calculés sur les données de test pour chaque classe d'entité $e$ comme suit :
\[Precision_e = \frac{\text{nombre d'entités de type } e \text{ parfaitement détectées par le modèle}} {\text{nombre d'entités détectées et classifiées } e\text{ par le modèle}}\]
\[Rappel_e = \frac{\text{nombre d'entités de type } e \text{ parfaitement détectées par le modèle}} {\text{nombre d'entitées manuellement classifiées } e}\]

\vspace{0.3cm}

\noindent \underline{\textbf{Evaluation globale (\textit{overall-level})}}: L'évaluation globale donne les performances générales d'un modèle sans distinction des classes ou labels. Elle est réalisée aux deux niveaux décrits précédemment mais independamment du label d'élément ou du type d'entité :
\[Precision = \frac{\text{nombre d'entitées (resp. d'éléments) correctement detectées par le modèle}} {\text{nombre d'entitées (resp. d'éléments) detecteés par le modèle}}\]
\[Rappel = \frac{\text{nombre d'entitées (resp. d'éléments) correctement detectées par le modèle}} {\text{nombre d'entitées (resp. d'éléments)  manuellement annotatées}}\]

\subsubsection{Outils logiciels}
Nous avons utilisé les modèles HMM et CRF tels qu'implémentés dans la librairie Mallet \citep{McCallum2002Mallet}. Les modèles étudiés ont été entrainés par la méthode d'espérance maximale pour ceux basés sur le HMM, et par la méthode L-BFGS pour ceux basés sur le CRF. La \textit{tokenisation} des textes (découpage en élément atomique de type mots), ainsi que la lémmatisation et l'annotation des rôles grammaticaux (\textit{Part-of-Speech tagging}) ont été effetué à l'aide de la fonctionnalité d'annotation de texte français de TreeTagger \footnote{\url{http://www.cis.uni-muenchen.de/~schmid/tools/TreeTagger}}  \citep{schmid1994treetagger}. L'implémentation dans Mallet du LDA \citep{blei2003lda} a permis d'inférer 100 thèmes à partir d'un corpus lémmatisé d'environ 6k documents. Le tableau \ref{p4_topics} 
présente des mots représentatifs trouvés dans les premiers thèmes inférés. L'extraction des autres caractéristiques manuelles a été implémentée pour cette expérimentation. 

Les valeurs de précision, rappel, et F1-mesure ont été calculées à l'aide du script d'évaluation de la campagne CoNLL-2002 \footnote{\url{http://www.cnts.ua.ac.be/conll2002/ner/bin/conlleval.txt}}.

\begin{table}[!h]
\scriptsize
\begin{center}
\begin{tabular}{c|l}
Id thème & Mots représentatifs  \\ \hline
0	& 	préjudice  dommage  somme  subir  réparation  titre  faute  payer  intérêt  responsabilité  \\ \hline
1	& société  salarié  groupe  mirabeau  pouvoir  demande  article  licenciement  cour  titre    \\ \hline
2	& harcèlement  travail  salarié  moral  employeur  fait  attestation  faire  santé  agissements  \\ \hline
3	& vente  acte  prix  vendeur  acquéreur  notaire  condition  clause  vendre  immeuble  \\ \hline
4	& 		travail  poste  reclassement  employeur  médecin  licenciement  salarié  inaptitude  visite  \\ \hline
5	& 	monsieur  nîmes  avocat  appel  barreau  arrêt  madame  disposition  prononcer  président  \\ \hline
6	& 	mademoiselle  madame  non  mesure  décision  tutelle  surendettement  comparant   \\ \hline
7	& transport  marchandise  jeune  sed  éducateur  bateau  navire  transporteur  responsabilité  \\ \hline
8	&congé  salarié  conversion  emploi  plan  convention  employeur  sauvegarde  reclassement  \\ \hline
9	&marque  site  contrefaçon  sous  droit  auteur  joseph  produit  propriété  photographie  \\ \hline
10	&pierre  patrick  bordeaux  bruno  catherine  civil  article  corinne  cour  avocat\\ \hline
\end{tabular}
\end{center}
\caption{Mots représentatifs des 10 premiers thèmes sur les 100 inférés}\label{p4_topics}
\end{table}


\subsection{Sélection du schémas d'étiquetage}
Dans le but dévaluer comment la représentation de segment affecte les performances, nous avons implémenté quatre représentations (IO, IEO2, BIO2, BIEO).  Nos avons réalisé un simple découpage des données en deux ensembles: $25 \%$ pour l'entrainement et $75 \%$ pour les tests. Les performances reportées dans le Tableau \ref{fig:structuration:select-segm-repr} sont les performances globales sur la base de test. Seul l'élément (mot/ligne) est utilisé comme dscripteur. La durée d'entrainement est très longue, particulièrement pour la détection d'entité dans l'entête avec le CRF. Il semble évident que cette durée croit proportionnellement avec le nombre de labels possibles (BIEO exige beaucoup plus de temps, et IO exige le minimum de temps). Le schéma IOE semble être plus rapide à que BIO même s'ils ont le même nombre de labels. Nous remarquons aussi que les représentations complexes n'améliorent pas significativement les résultats par rapport au simple IO qui consomme pourtant le moins de temps.

\begin{table}[h]
\scriptsize
\caption{Comparaison des schémas d'étiquetage.}\label{fig:structuration:select-segm-repr}
\begin{center}
\begin{tabular}{p{0.9cm}|c|cccccccc}
\hline \noalign{\smallskip}
Tâche & Modèle & \multicolumn{3}{c}{Niveau atomique$^a$} & \multicolumn{3}{c}{Niveau entité$^a$} & \multirow{2}{*}{Durée$^b$} & Schéma \\
 & & Précision & Rappel & F1 &  Précision & Recall & F1 &  & \\ \hline %\noalign{\smallskip}\svhline\noalign{\smallskip}
\multirow{8}{*}{Sections}  & \multirow{4}{*}{CRF} & 91.75 & 91.75 & 91.75 & 64.49 & 56.55 & 60.26 &  4.685  & IO \\
&  & 88.95 & 88.95 & 88.95 & 48.12 & 38.26 & 42.63  & 11.877 & IEO2 \\
&  & 87.09 & 87.09 & 87.09 & 46.79 & 37.20 & 41.45 & 12.256 & BIO2 \\
 &  & 86.00 & 86.00 & 86.00 & 58.98 & 41.86 & 48.97  & 35.981 & BIEO \\ \cline{2-10}
& \multirow{4}{*}{HMM} & 32.64 & 32.64 & 32.64 & 22.16 & 18.91 & 20.41 & 6.564 & IO \\
&  & 32.92 & 32.92 & 32.92 & 17.73 & 16.09 & 16.87  &   7.827  & IEO2 \\
 &  & 32.39 & 32.39 & 32.39 & 31.93 & 26.65 & 29.05 & 8.391 & BIO2 \\
  &  & 33.06 & 33.06 & 33.06 & 32.47 & 27.53 & 29.80 & 8.7 & BIEO \\ \hline %
\multirow{8}{1.5cm}{Entités d'entête}  & \multirow{4}{*}{CRF} & 86.86 & 78.96 & 82.73 & 80.84 & 65.17 & 72.17  & 70.525 & IO \\
 &  & 87.77 & 79.65 & 83.51 & 82.46 & 65.19 & 72.82  & 228.751 & IEO2 \\
 &  & 87.41 & 78.14 & 82.51 & 81.66 & 66.80 & 73.49 & 230.865 & BIO2 \\
 &  & 87.72 & 79.55 & 83.44 & 84.38 & 68.35 & 75.53 &  475.249 & BIEO \\ \cline{2-10}
  & \multirow{4}{*}{HMM} & 79.12 & 67.75 & 73.00 & 61.48 & 35.05 & 44.64 & 6.345 & IO \\
  &  & 78.82 & 68.69 & 73.40 & 66.63 & 40.16 & 50.11& 8.298 & IEO2 \\ 
  &  & 80.68 & 67.48 & 73.49 & 70.37 & 45.32 & 55.14 & 7.908 & BIO2 \\
 &  & 80.05 & 69.01 & 74.12 & 74.73 & 50.77 & 60.46 & 9.973 & BIEO \\ \hline
\multirow{8}{*}{Normes}  & \multirow{4}{*}{CRF} & 95.60 & 92.96 & 94.26 & 88.06 & 83.50 & 85.72 & 28 & IO \\%
&  & 95.40 & 93.18 & 94.27 & 88.75 & 85.65 & 87.17 & 32.136 & IEO2 \\
 &  & 95.20 & 93.30 & 94.24 & 85.65 & 83.13 & 84.37 & 50.769 & BIO2 \\
  &  & 95.46 & 91.57 & 93.47 & 88.83 & 84.71 & 86.72 & 50.566 & BIEO \\ \cline{2-10}
  & \multirow{4}{*}{HMM} & 89.83 & 88.78 & 89.30 & 73.74 & 75.02 & 74.37 &  41.389 & IO \\%  
   &  & 88.20 & 89.23 & 88.71 & 78.01 & 81.27 & 79.61 & 44.086 & IEO2 \\
  &  & 89.25 & 87.83 & 88.53 & 73.89 & 76.63 & 75.24 & 46.634 & BIO2 \\
  &  & 87.39 & 88.10 & 87.74 & 77.76 & 82.35 & 79.99 & 45.52& BIEO \\ 
\noalign{\smallskip}\hline\noalign{\smallskip}
\end{tabular}
\end{center}

$^a$ Resultats sur une simple division du jeu de données en $25\%$ pour l'entrainement et  $75\%$ pour les tests (entrainement limité à 100 itérations au max)

$^b$ Durée d'entrainement en secondes avant l'arrêt de l'entrainement
\end{table}


\subsection{Sélection des descripteurs}
Pour comparer les méthodes BDS et SFFS, nous exploitons le schéma IO. Durant nos expérimentations, le SFFS a exécuté 185 entrainements pour le modèle CRF de d'identification des sections. La méthode BDS quant à elle à durée plus de 15h pour 600 sessions d'entrainement. Malgré la sauvegarde des scores F1 pour éviiter d'exécuter plusieurs fois l'entraimenent pour les mêmes sous-ensembles de descripteurs, le processus de sélection est resté toujours très long pour les deux algorithmes. Nous avons testé individuellement chacun des descripteurs candidat pour les modèles HMM. Les résultats sont repportés dans le Tableau \ref{fig:structuration:select-feats}.

Les descripteurs sélectionnés forment ds sous-ensembles inattendus étant donné que certains descripteurs spéciaux des voisins ont été choisis. Par exemple, dans le cas de la détection de section, la ligne suivante semble être beaucoup plus indicatrice que la première. Il est aussi intéressant de noter que les descripteurs basés sur notre observation apparaissent dans les sous-ensembles sélectionnés (par ex. isAfterIntervenant, isKEYWORD). Remarquons aussi que la longueur absolue des lignes (absLength)  joue un rôle majeur dans l'identification des sections vu qu'il a été sélectionné à la fois pour le CRF et le HMM (sélection BDS). Avec ses sous-ensembles sélectionnés de descripteurs, les modèles sont plus performants que lorsqu'ils ne doivent exploiter que l'élément ou tout l'ensemble des descripteurs candidats.  Cette amélioration des résultats reste insignifiante lorsqu'on considère la longue durée d'exécution des algorithmes. Ainsi, un algorithme meilleur et plus rapide devrait être utilisé à la place du SFFS et du BDS.

\begin{table}[!h]
\scriptsize
\caption{Effects of selected feature subsets on results.}\label{fig:structuration:select-feats}
\begin{center}
\begin{tabular}{l|c|ccc|ccc|c}
\hline\noalign{\smallskip}
Detection Task & Tagger & \multicolumn{3}{c}{niveau atomique$^a$} & \multicolumn{3}{c}{niveau entité$^a$}& Sous-ensemble \\
 & & Précision & Rappel & F1 &  Précision & Rappel & F1 & sélectionné\\
\noalign{\smallskip}\hline\noalign{\smallskip}
\multirow{7}{*}{Sections} 		& \multirow{4}{*}{CRF} & 99.31 & 99.31 & 99.31 & 90.28 & 90.68 & 90.48 & BDS$^{b1}$  \\
  				&  & 99.55 & 99.55 & \textbf{99.55} & 85.69 & 85.84 & 85.76 & \textbf{SFFS}$^{b2}$ \\
                &  & 99.36 & 99.36 & 99.36 & 88.16 & 88.39 & 88.27 & ALL* \\
                &  & 91.75 & 91.75 & 91.75 & 64.49 & 56.55 & 60.26 & token \\  \cline{2-9}
                 & \multirow{3}{*}{HMM} & 90.99 & 90.99 & \textbf{90.99}  & 4.18 & 3.63 & 3.89 & \textbf{absLength} \\ 
 & & 86.97 & 86.97 & 86.97 & 4.08 & 3.30 & 3.65 & relLength \\   
  &  & 37.59 & 37.59 & 37.59  & 18.81 & 18.81 & 18.81 & token \\ \hline
\multirow{7}{*}{Entités d'entête}	& \multirow{4}{*}{CRF} & 94.00 & 91.42 & 92.69 & 92.26 & 88.76 & 90.47 & BDS$^{c1}$  \\
				&  & 94.10 & 91.93 & \textbf{93.00} & 92.64 & 88.96 & 90.76 & \textbf{SFFS}$^{c2}$  \\ 
                &  & 94.20 & 91.86 & 93.02 & 93.05 & 89.59 & 91.28 & ALL \\
                &  & 86.86 & 78.96 & 82.73 & 80.84 & 65.17 & 72.17 & token \\ \cline{2-9}
                  &  \multirow{3}{*}{HMM}  & 76.90 & 80.41 & \textbf{78.61} & 62.66 & 52.16 & 56.93 &  \textbf{token} \\ 
  &    & 66.48 & 69.67 & 68.04 & 39.34 & 28.36 & 32.96 &  lemma\_W0 \\ 
  &    & 39.63 & 37.50 & 38.54 & 15.49 & 5.35 & 7.95 &  POS \\ \hline
\multirow{6}{*}{Normes} 			& \multirow{4}{*}{CRF} & 95.91 & 96.72 & 96.31 & 91.14 & 90.45 & 90.80 & \textbf{BDS}$^{d1}$ \\ 
				&  & 95.68 & 95.45 & 95.57 & 90.34 & 88.27 & 89.29 & SFFS$^{d2}$ \\ 
                &  & 95.07 & 96.69 & 95.87 & 90.87 & 90.64 & 90.76 & ALL \\
                &  & 95.60 & 92.96 & \textbf{94.26} & 88.06 & 83.50 & 85.72 & token \\ \cline{2-9}
                 &  \multirow{2}{*}{HMM} & 89.21 & 94.25 & 91.66 & 72.67 & 77.28 & 74.90 &  \textbf{token} \\ 
  &   & 90.31 & 92.81 & 91.54 & 69.24 & 69.46 & 69.35 &  lemma\_W0 \\ 
%  \noalign{\smallskip}\svhline\noalign{\smallskip}
%  & & token-level & entity-level & \\ \hline
\noalign{\smallskip}\hline\noalign{\smallskip}
\end{tabular}
\end{center}

$^a$ Resultats sur un simple découpage des données de $25\%$ pour l'entrainement,  $75\%$ pour le test avec 100 itérations d'entrainement au maximum  pour le CRF, et $80\%$ pour l'entrainement et $20\%$ pour le test avec 50 itérations au maximum pour l'entrainement du HMM

$^{b1}$ Selection par BDS pour les sections : [p0, n0, relNum, absLength, t0, t1, t2]

$^{b2}$ Selection par SFFS pour les sections: [n0, nRelLength, relNum, t0, t1, t2]

 $^{c1}$ Selection par BDS pour les entités d'entête:  [POSW1, isAfterAPPELANT, numInLine, w-2topic0, POSW2, isAfterINTERVENANT, isAfterINTIME, POSW-2, isLONELYINITIAL, token, lemma\_W0, lemmaW-2, isALLPUN, w-1, w1, w2, isALLCAP]

$^{c2}$ Selection par SFFS pour les  entités d'entête: [numInLine, w-2topic0, lemmaW-2, isAfterINTERVENANT, isAfterINTIME, w-1, w1, w2, isALLCAP, token]

$^{d1}$ Selection par BDS pour les normes: [POSW1, w-2topic0, isKEYWORD, lemmaW2, DIGIT-IN, token, lemmaW1, lemmaW-2, POS, isALLPUN, w-1, w2, PUN-IN, w-2]

$^{d2}$ Selection par SFFS pour les normes: [POSW1, lemmaW-2, w-1, DIGIT-IN]
\end{table}



\subsection{Evaluation détaillée pour chaque classe}
Nous discutons ici la capacité des modèles à identifier individuellement chaque type d'entité et de section. Les test ont été conduits avec tous les descripteurs pour les modèles CRF. Seuls \textit{absLength} et \textit{token} ont été utilisé comme descripteurs dans les modèles HMM pour l'identification des sections et des entités resp.. Le schéma d'étiquetage est IO. Le nombre d'itération maximal a été fixé à 500 pour assurer la convergence lors de l'entrainement même si les modèles HMM ne convergeaient jamais après 500 itérations. Les Tableaux \ref{tab:structuration:perf-detail-token} et \ref{tab:structuration:perf-detail-entity} présentent les résultats d'une 5-fold validation croisée au niveau des éléments et des entités, resp.. D'un point de vue général, les modèles HMM se comportent assez bien au niveau élément avec un seul descripteur, particulièrement pour l'identification des sections et des normes. Le modèle HMM est capable de labeliser les normes grâce à la mention des mêmes normes entre les décisions et la syntaxe presque standard que respectent les mentions (\verb=article [IDENTIFIANT] [TEXTE D'ORIGINE]=). Le modèle HMM n'est cependant pas efficace pour détecter entièrement les mots des entités d'où le faible score enregistré au niveau entité. Quant aux CRF, leurs résultats sont bons pour toutes les tâches et à tous les niveaux d'évaluation malgré certaines limites observées sur l'identification des mentions de parties.

\begin{table}[!h]
\centering
\scriptsize
\begin{subfigure}[t]{0.45\textwidth}
\centering
\begin{tabular}{|l|ccc|}
\hline
        & Precision  &  Recall   & F1 \\\hline
I-corps &   92.46 &  95.25 &  93.83 \\
I-dispositif &   53.44 &  48.46 &  50.83 \\
I-entete &   97.91 &  91.93 &  94.83 \\\hline
Overall &   90.63 &  90.63 &  90.63 \\\hline
 \noalign{\smallskip}\hline\noalign{\smallskip}
I-appelant &   34.46 &  16.87 &  22.65 \\
I-avocat &   85.17 &  98.75 &  91.46 \\
I-date  &   75.67 &  72.45 &  74.02 \\
I-fonction &   88.81 &  64.46 &  74.70 \\
I-formation &   79.38 &  94.38 &  86.23 \\
I-intervenant &   82.07 &  38.04 &  51.98 \\
I-intime &   50.40 &  68.09 &  57.93 \\
I-juge  &   73.40 &  88.73 &  80.34 \\
I-juridiction &   85.15 &  98.37 &  91.28 \\
I-rg    &   68.53 &  22.14 &  33.47 \\
I-ville &   91.50 &  82.41 &  86.72 \\\hline
Overall &   76.21 &  82.26 &  79.12 \\\hline
 \noalign{\smallskip}\hline\noalign{\smallskip}
I-norme &   88.23 &  93.70 &  90.89 \\\hline
\end{tabular}
\caption{Modèles HMM avec les descripteurs \textit{absLength} and \textit{token} pour l'identification resp. de section et d'entités et le schéma d'étiquetage IO.}\label{tab:structuration:perf-detail-token-hmm}
\end{subfigure} 
\hfill
\begin{subfigure}[t]{0.45\textwidth}
\centering
\begin{tabular}{|l|ccc|}
\hline
         & Precision &  Recall  & F1 \\\hline
I-corps &   99.57 &  99.69 &  99.63 \\
I-dispositif &   98.63 &  97.59 &  98.11 \\
I-entete &   99.51 &  99.55 &  99.53 \\\hline
Overall &   99.48 &  99.48 &  99.48 \\\hline
 \noalign{\smallskip}\hline\noalign{\smallskip}
I-appelant &   84.34 &  76.27 &  80.10 \\
I-avocat &   98.02 &  98.15 &  98.09 \\
I-date  &   98.00 &  96.60 &  97.30 \\
I-fonction &   95.23 &  95.13 &  95.18 \\
I-formation &   98.80 &  99.45 &  99.12 \\
I-intervenant &   83.38 &  68.26 &  75.07 \\
I-intime &   82.54 &  83.33 &  82.93 \\
I-juge  &   97.55 &  97.23 &  97.39 \\
I-juridiction &   98.91 &  99.69 &  99.30 \\
I-rg    &   97.81 &  97.44 &  97.62 \\
I-ville &   98.94 &  99.15 &  99.04 \\\hline
Overall &   95.13 &  94.51 &  94.82 \\\hline
 \noalign{\smallskip}\hline\noalign{\smallskip}
I-norme &   97.14 &  96.09 &  96.62 \\\hline
\end{tabular}
\caption{Modèles CRF avec tous les descripteurs et le schéma IO}\label{tab:structuration:perf-detail-token-crf}
\end{subfigure} 
\caption{Précision, Rappel, F1-mesures pour chaque type d'entité et section au niveau atomique.}\label{tab:structuration:perf-detail-token}
\end{table}

\begin{table}[!h]
\centering
\scriptsize
\begin{subfigure}[t]{0.45\textwidth}
\centering
\begin{tabular}{|l|ccc|}
\hline
        & Precision &  Recall  & F1 \\\hline
corps   &    0.99 &   0.99 &   0.99 \\
dispositif &   12.05 &   7.33 &   9.11 \\
entete  &   10.47 &  10.50 &  10.48 \\\hline
Overall &    7.22 &   6.27 &   6.71 \\\hline
 \noalign{\smallskip}\hline\noalign{\smallskip}
appelant &   17.84 &   5.60 &   8.52 \\
avocat  &   44.29 &  39.15 &  41.56 \\
date    &   66.87 &  62.15 &  64.43 \\
fonction &   89.84 &  64.13 &  74.84 \\
formation &   61.50 &  65.86 &  63.61 \\
intervenant &   14.29 &   4.00 &   6.25 \\
intime  &   30.28 &  27.47 &  28.80 \\
juge    &   73.54 &  83.21 &  78.07 \\
juridiction &   81.31 &  87.66 &  84.37 \\
rg      &   68.53 &  22.41 &  33.77 \\
ville   &   89.52 &  84.70 &  87.05 \\\hline
Overall &   64.59 &  54.56 &  59.15 \\\hline
 \noalign{\smallskip}\hline\noalign{\smallskip}
norme   &   71.94 &  78.45 &  75.05 \\\hline
\end{tabular}
\caption{Modèles HMM avec les descripteurs \textit{absLength} and \textit{token} pour l'identification resp. de section et d'entités et le schéma d'étiquetage IO.} \label{tab:structuration:perf-detail-entity-hmm}
\end{subfigure} 
\hfill
\begin{subfigure}[t]{0.45\textwidth}
\centering
\begin{tabular}{|l|ccc|}
\hline
         & Precision &  Recall  & F1 \\\hline
corps   &   89.57 &  90.10 &  89.83 \\
dispositif &   98.02 &  97.82 &  97.92 \\
entete  &   92.11 &  92.48 &  92.29 \\\hline
Overall &   93.22 &  93.47 &  93.34 \\\hline
 \noalign{\smallskip}\hline\noalign{\smallskip}
appelant &   84.05 &  77.29 &  80.53 \\
avocat  &   90.97 &  90.30 &  90.63 \\
date    &   97.96 &  96.60 &  97.27 \\
fonction &   96.89 &  96.94 &  96.92 \\
formation &   98.40 &  98.95 &  98.68 \\
intervenant &   62.50 &  40.00 &  48.78 \\
intime  &   79.31 &  78.93 &  79.12 \\
juge    &   96.58 &  96.35 &  96.47 \\
juridiction &   98.86 &  99.54 &  99.20 \\
rg      &   97.57 &  98.02 &  97.79 \\
ville   &   98.85 &  99.15 &  99.00 \\\hline
Overall &   93.77 &  92.93 &  93.35 \\\hline
 \noalign{\smallskip}\hline\noalign{\smallskip}
norme   &   92.66 &  91.38 &  92.01 \\\hline
\end{tabular}
\caption{Modèles CRF avec tous les descripteurs et le schéma IO}\label{tab:structuration:perf-detail-entity-crf}
\end{subfigure} 
\caption{Précision, Rappel, F1-mesures pour chaque type d'entité et section au niveau entité.}\label{tab:structuration:perf-detail-entity}
\end{table}


\subsection{Analyse des erreurs}
\subsubsection{Confusion de classes}
Certaines erreurs sont probablement dûe à la proximité des entités de types différents. D'après la matrice de confusion (Figure \ref{fig:structuration:matrices-confusions}), les \textit{intervenants} sont souvent classifiés comme \textit{intimé} ou \textit{avocat} probablement parce qu'il s'agit d'individus mentionnés les uns à la suite des autres dans l'entête (les \textit{intervenants} sont mentionnés juste après les \textit{avocats} des \textit{intimés}). Certaines mentions d'\textit{appelant} sont aussi classifiées comme \textit{intimés} dans plusieurs documents. La proximité crèe aussi des confusions entre des sections qui se suivent c'est-à-dire entre ENTETE et CORPS, et entre CORPS ET DISPOSITIF.  

\begin{figure}[h!]
    \centering
    %\includegraphics{}
    \textcolor{red}{Matrices de confusion}
    \caption{Matrices de confusion}
    \label{fig:structuration:matrices-confusions}
\end{figure}

\subsubsection{Redondance des mentions d'entitées}
Il est aussi intéressant de remarquer que certaines entitées sont repétées dans le document. Par exemple, les des parties sont mentionnées précédemment à une autre mention qui donne plus de détail sur eux. Certaines normes sont aussi citées de manière repétée et en alternant souvent les formes allongées et les formes longues. Malgré le fait que les mentions repétées ne sont pas identiques, de telles redondance aident à réduire le risque de manquer une entité. Cette aspect peut être exploité afin de combler l'imperfection des modèles.

\subsubsection{Découpage atomique}

\subsubsection{Impact de la quantité d'exemples annotées}
Certaines expérimentations ont été conduites pour évaluer comment les modèles s'améliorent lorsqu'on augmente le nombre de données d'entrainement. Pour cela, nous avons calculé les performances des modèles pour de différentes taille de la base d'entrainement. Les données ont été divisées en $75\%-25\%$ pour resp. l'entrainement et le test. Seules 20 fractions de l'ensembles d'entrainement ont été testées (de 5\% à 100\%). A chasue session entrainement-test, le même jeu de test a été utilisé pour les différentes fractions de l'ensemble d'entrainement. Les courbes d'apprentissage des modèles CRF et HMM sont représentées resp. sur les Figures \ref{fig:structuration:learning-curves-crf} et \ref{fig:structuration:learning-curves-hmm}. Il est évident que les scores F1 croissent avec plus de données d'entrainement pour les CRF et HMM, mais cette amélioration devient très faible au-delà de 60\% de données d'entrainement quelque soit la tâche. Il est possible que les exemples rajoutés à partir de là partagent la même structure qu'une majorité d'autres. Ainsi, cette étude doit être étendue à la sélection des exemples les plus utiles. \citet{raman2003exampleSelection} a demonstré les avantages des algorithmes de sélection d'exemples combinés à la sélection de caractéristique pour la classification. Les mêmes méthodes sont probablement applicables à l'étiquetage de séquence.

\begin{figure}[!htb]
\centering
\begin{subfigure}[t]{0.95\textwidth}
\centering
\includegraphics[width=0.95\textwidth]{lc-crf.png}
\caption{CRF} \label{fig:structuration:learning-curves-crf}
\end{subfigure} 

\begin{subfigure}[t]{0.95\textwidth}
\centering
\includegraphics[width=0.95\textwidth]{lc-hmm.png}
\caption{HMM} \label{fig:structuration:learning-curves-hmm}
\end{subfigure}
\caption{Courbes d'apprentissages aux niveaux élément et entité} \label{fig:structuration:learning-curves}
\end{figure}

\section{Conclusion}
\label{sec:structuration:conclusion}
	% INCLUDE: structuration
\chapter{Identification des demandes}
%\chapter{Extraction des demandes et résultats correspondants}
\label{chap:quanta}

\textit{\small \textbf{Résumé.} Ce chapitre aborde le problème d'identification automatique, dans une décision, des éléments structurants les demandes formulées. L'identification manuelle réalisée à travers la lecture exige beaucoup d'effort à cause de la complexité du contenu dans lequel les demandes sont mélangées à d'autres informations (des demandes de nature différente, des arguments, des faits, etc.). L'automatisation de cette tâche métier vise à aider les experts à rapidement comprendre les réclamations des parties et les réponses correspondantes des juges.  Une demande est abstraite par cinq attributs : la norme qui la fonde, son objet, l'interprétation du résultat ($s_r$), le quantum demandé ($q_d$), et celui obtenu ($q_r$). La norme et l'objet forment ensemble la catégorie de la demande. L'annotation manuelle des données d'évaluation suit un protocole que nous avons précisément défini avec l'expert du projet. Ce protocole recommande des cycles d'annotation consistant chacun à constituer un ensemble de décisions et à y identifier toutes les demandes d'une seule catégorie donnée car il serait difficile d'annoter simultanément des données pour toutes les catégories qui sont très nombreuses. L'approche proposée extrait à chaque application les demandes d'une seule catégorie et est formulée en trois tâches. La présence de la catégorie est déterminée  par classification de la décision. Ensuite, les quanta et le sens du résultat sont identifiés à proximité de termes appris de la catégorie dans les sections adéquates identifiées à l'aide d'un modèle à base de CRF comme décrit au chapitre \ref{chap:structuration}. Enfin, les demandes sont formées en mettant en correspondance les éléments précédemment déterminés. Réalisées par validation croisées, nos expérimentations comparent une douzaine de méthodes statistiques d'extraction de termes-clés et quatre algorithmes de classification pour l'extraction de 6 catégories prédéfinies de demandes. Les résultats montrent que la détection de catégorie est facile quelque soit l'algorithme utilisé ($F_1$-mesure comprise entre 98.8 \% et 100 \%). Il résulte aussi que l'extraction des demandes nécessite de sélectionner la méthode d'extraction de terminologie la mieux adaptée à la catégorie. Cette sélection préalable permet d'observer, sur les données de test, des $F_1$-mesures comprises entre 33.09 \% et 71.43 \% pour les champs $q_d$, $q_r$, et $s_r$, et entre 28.65 \% et 58.99 \% pour les triplets $(q_d,s_r, q_r)$.}

\section{Introduction}
\label{sec:quanta:introduction}

%\textcolor{red}{Introduire, chaque chapitre, de manière explicite  par la contribution, discuter l'impacte de chaque méthode explorée, contribution sur la constitution des datasets, accompagnement des experts. dans un encadré}

Au c\oe{}ur de l'analyse des décisions de justice se trouve le concept de demande. Il s'agit d'une réclamation ou requête effectuée par une ou plusieurs parties aux juges. Une partie peut demander des dommages-intérêts en réparation d'un préjudice subi ou à l'issu d'un divorce, des indemnités auxquelles elle pense avoir droit, ou encore une étude d'expert, etc. Les demandes sont fondamentales car l'argumentation au cours d'une affaire a deux buts : faire accepter ses demandes, et faire rejeter celles de la partie adverse. L'extraction des demandes et des résultats correspondants, dans un corpus, permet ainsi de récolter des données informant de la manière dont sont jugés des types de demandes d'intérêt. Les informations qui nous intéressent sont la catégorie de la demande, le quantum (montant) demandé, le sens du résultat (par ex. la demande a-t-elle été acceptée ou rejetée ?), et le quantum obtenu (décidé par les juges). Pour pouvoir extraire les demandes et les résultats, il est nécessaire de comprendre comment ceux-ci sont exprimés et co-référencés dans les décisions jurisprudentielles. Leur énoncé peut comporter des expressions plus ou moins complexes, dont souvent des références à des jugements antérieurs, des agrégations ou des restrictions (\figureref{fig:quanta:expr-dmd-rst}).

\begin{figure}[ht]
	\footnotesize
\begin{center}
	\begin{subfigure}[t]{0.95\textwidth}
		\fbox{\parbox{\textwidth}{Jennifer M., Catherine M. et Sandra M. ... demandent à la Cour de :
				
				- les recevoir régulièrement appelantes incidentes du \textcolor{blue}{jugement du 23/05/2014} ;
				
				- infirmer \textcolor{blue}{le dit jugement} en \textcolor{brown}{toutes ses dispositions} ; ...
				
				Statuant à nouveau ...
				
				- \textcolor{brown}{les condamner au paiement d'une somme de  3 000,00 \euro{} pour procédure abusive et aux entiers dépens} ; }}
		\caption{Exemples d'énoncés de demandes}\label{fig:quanta:expr-dmd}
	\end{subfigure} 
	
	
	\begin{subfigure}[t]{0.95\textwidth}
		\fbox{\parbox{\textwidth}{La cour, ...  
				
				CONFIRME \textcolor{blue}{le jugement entreprise} en \textcolor{brown}{toutes ses dispositions}.
				
				Y ajoutant
				
				\textcolor{gray}{CONSTATE que Amélanie Gitane P. épouse M. est défaillante à rapporter la preuve
					d'une occupation trentenaire lui permettant d'invoquer la prescription
					acquisitive de la parcelle BH 377 située [...].}
				
				\textcolor{gray}{DEBOUTE Amélanie Gitane P. épouse M. de sa demande en dommages et intérêts.}
				
				\textcolor{gray}{CONDAMNE Amélanie Gitane P. épouse M. aux dépens d'appel.}
				
				\textcolor{gray}{DIT n'y avoir lieu à l'application de l'article 700 du Code de Procédure Civile.}
		}}
		\caption{Exemple d'énoncés de résultats}\label{fig:quanta:expr-rst}
	\end{subfigure}
\end{center}

\textit{\textbf{Source:} extraits de la décision 14/01082 de la cour d'appel de Saint-Denis (Réunion).}

\textit{\textbf{Légende:} énoncés simples en \textcolor{gray}{gris}, références en \textcolor{blue}{bleu}, et agrégations en \textcolor{brown}{marron}.}
	\caption{Illustrations de la complexité des énoncés de demandes et de résultats.}\label{fig:quanta:expr-dmd-rst}
\end{figure}


\subsection{Données cibles à extraire}

\subsubsection{Catégorie de demande}

Une catégorie $c$ de demande regroupe les prétentions qui sont de même nature par le fait qu'elles partagent deux aspects : l'objet demandé (par ex. dommages-intérêts, amende civile, déclaration de créance) et le fondement c'est-à-dire les règles ou normes ou principes juridiques qui fondent la demande (par ex. article 700 du code de procédure civile). Des noms particuliers sont utilisés pour identifier les catégories (Tableau \ref{tab:quanta:exemple-categorie}).

\begin{table}[!htb]
%\scriptsize
\begin{tabular}{|c|p{0.35\textwidth}|p{0.15\textwidth}|p{0.3\textwidth}|}
\hline
\textbf{Label} & \textbf{Expression nominative }                                     & \textbf{Objet}                                                       & \textbf{Fondement}                                                                 \\ \hline
\textit{acpa} & amende civile pour abus de procédure                         & amende civile                                               & Articles 32-1 code de procédure civile + 559 code de procédure civile  \\ \hline
concdel & dommages-intérêts pour concurrence déloyale                  & dommages-intérêts                                           & Article 1382 du code civil                                             \\ \hline
\textit{danais} & dommages-intérêts pour abus de procédure                   & dommages-intérêts                                           & Articles 32-1 code de procédure civile + 1382 code de procédure civile \\ \hline
\textit{dcppc} & déclaration de créance au passif de la procédure collective  & déclaration de créance & L622-24 code de commerce                                               \\ \hline
\textit{doris} & dommages-intérêts pour trouble de voisinage                  & dommages-intérêts                                           & principe de responsabilité pour trouble anormal de voisinage           \\ \hline
\textit{styx} & frais irrépétibles                                          & dommages-intérêts                                           & Article 700 du code de procédure civile                                 \\ \hline
\end{tabular}
\textit{Les labels ont été définis particulièrement dans le cadre du projet, et par conséquent, ils n'existent pas dans le langage juridique.}
\caption{Exemples de catégories de demandes}\label{tab:quanta:exemple-categorie}
\end{table}

\subsubsection{Sens du résultat}

Le sens du résultat est l'interprétation de la décision des juges sur une demande. Nous le notons $s_r$. En général, le sens peut être positif si la demande a été acceptée, et négatif si elle a été rejetée. Il arrive aussi que le résultat soit reporté à un jugement futur ; il s'agit dans ce cas d'un sursis à statuer. 

\subsubsection{Quantum demandé}

Le quantum demandé quantifie l'objet de la demande. Nous le notons $q_d$. Par exemple, dans l'exemple de la Figure \ref{fig:quanta:expr-dmd}, "3000 \euro{}" est le quantum demandé au titre des dommages-intérêts pour procédure abusive. Bien que cette étude ne porte que sur des sommes d'argent, le quantum peut être d'une autre nature comme par exemple une période dans le temps (garde d'enfant, ou emprisonnement, etc.). Toutes les catégories demandes n'ont pas de quantum (par ex. une demande de divorce) et seul le sens du résultat sera la donnée à extraire dans ce cas.


\subsubsection{Quantum obtenu ou résultat}

Le quantum obtenu quantifie le résultat ou la décision des juges. Nous le notons $q_r$. Il ne peut qu'être inférieur ou égal au quantum demandé. Si la demande est rejetée, 
$q_r$ est nul même si cela n'est pas explicitement mentionné dans le document. A noter qu'il doit être de la même nature que le quantum demandé (somme d'argent ou durée).


\subsection{Expression, défis et indicateurs d'extraction}

Les demandes sont en général décrites à la fin de la section d'exposé des faits, procédures, moyens et prétentions des parties (section Litige cf. \ref{sec:structuration:sectionnement-en-4} et \ref{appendix:exemple-decision}). Elles rentrent donc dans les "moyens et prétentions des parties" qui regroupent les demandes et les arguments des parties. Quant aux résultats, ils sont décrits dans la section Dispositif et dans la section Motifs (raisonnement des juges). Les demandes sont exprimées dans différents paragraphes qui correspondent soit à une partie, soit à un groupe de parties partageant les mêmes demandes (par ex. des époux). Les paragraphe sont parfois organisés en liste dont chaque élément exprime une ou plusieurs demandes, ou fait référence à un jugement antérieur. Les résultats ont aussi la forme de liste dans la section Dispositif. Par contre, dans les motifs de la décision, les raisonnements sont organisés en paragraphes, et ordonnés catégorie après catégorie. Le résultat est donné à la fin du groupe de paragraphes associé à la catégorie.


 Cette pseudo-structure n'est pas standard et impose de nombreux défis à relever. En effet, une décision jurisprudentielle porte sur plusieurs demandes de catégories différentes ou similaires. Il est important de faire correspondre un quantum demandé extrait au sens et quantum du résultat qui font référence à la même demande. La séparation des demandes et des résultats rend difficile cette mise en correspondance. Ce problème peut aussi être causé par la redondance des quanta ; par exemple, les résultats exprimés dans les Motifs sont résumés dans le Dispositif. D'autre part, les références aux jugements antérieurs exigent de résoudre des références aux résultats de jugements antérieurs qui sont, généralement, rappelés dans le même document. Notons aussi que les difficultés liées aux agrégations (par ex. "\textit{infirmer ... en toutes ces dispositions}") et aux restrictions/sélections (par ex. "\textit{infirme le jugement ... sauf en ce qu'il a condamné M. A. ...}") méritent d'être résolues. Par ailleurs, les catégories de demandes sont nombreuses\footnote{plus de 500 selon la nomenclature des affaires civiles NAC+.} mais ne sont pas toutes présentes à la fois dans les décisions. Tous ces aspects rendent difficile l'annotation manuelle des données de référence et la modélisation d'une approche d'extraction adéquate. Nous avons cependant identifié des indicateurs qui pourraient être utiles.

On pourrait au préalable annoter les candidats potentiels de quanta. Nous nous sommes intéressés aux demandes dont les quanta sont des sommes d'argent. Les mentions de somme d'argent sont généralement de la forme \og \texttt{[valeur] [monnaie]} \fg{} (par ex. \texttt{3000 \euro}, \texttt{15 503 676 francs}, \texttt{un euro}, \texttt{339.000 XPF}). Des centimes apparaissent parfois (par ex. \texttt{dix huit euros et soixante quatorze centimes}, \texttt{26'977 \euro{}  19}).  Ainsi, il est possible d'annoter les sommes d'argent à l'aide d'une expression régulière. Même s'il est difficile de reconnaître des sommes d'argent écrites en lettre, il faut remarquer que l'équivalent en chiffre est généralement mentionné tout près (par ex. \texttt{neuf mille cinq cent soixante six euros et quatre vingt sept centimes (9566,87 \euro{}  )}). 

La terminologie utilisée est aussi un bon indicateur pour reconnaître des demandes et des résultats. En effet, le vocabulaire utilisé est très souvent propre aux catégories de demandes. Par exemple le dernier élément de la Figure \ref{fig:quanta:expr-dmd} comprend le terme "\textit{pour procédure abusive}" qui est près d'une somme d'argent (\textit{3000 \euro{}}) ; il est donc probable que ce type de terme assez particulier soit un bon indicateur de la position des quanta. Par ailleurs, des verbes particuliers sont utilisés pour exprimer les demandes et résultats : infirmer, confirmer, constater, débouter, dire ... %Comme autres formes récurrentes, on pourrait citer l'ordre des demandes (resp. résultats). Généralement, on a les constats, les références aux jugements antérieurs, les demandes principales (?) et secondaires (?).




% Dans la section suivante, nous discutons de l'analogie de l'extraction des demandes avec d'autres problématiques d'extraction d'information. 


\subsection{Formulation du problème}
\label{sec:quanta:formulation}

Nous avons tenu compte de deux principaux aspects du problème :
\begin{enumerate}
	\item Une décision comprend plusieurs demandes de catégories similaires ou différentes;
	\item  Il existe un grand nombre de catégories (500+) ; ce qui rend difficile l'annotation d'exemples de référence pour couvrir toutes ces catégories.
\end{enumerate}

 L'idée est de pouvoir ajouter progressivement de nouvelles catégories. Nous avons par conséquent opté pour une extraction par catégorie. Cette stratégie permet par ailleurs d'ajouter facilement de nouvelles classes sans avoir à redéfinir les classes déjà entraînées. Une exécution du système d'extraction permet ainsi d'extraire les demandes d'une seule catégorie. Le problème est décomposé en deux tâches :
\begin{description}
	\item[Tâche 1 :] Détecter les catégories présentes dans le document pour appliquer l'extraction  uniquement à ces catégories;
	\item[Tâche 2 :] Pour chaque catégorie $c$ identifiée, extraire les demandes :
	\begin{enumerate}
		\item identification des valeurs d'attributs : quanta demandés ($q_d$), quanta obtenus ($q_r$), et sens du résultat ($s_r$);
		\item mise en correspondance des attributs pour former les triplets ($q_d, s_r, q_r$) correspondants aux paires demande-résultat.
	\end{enumerate}
\end{description}

%L'annotation des candidats de quanta, les sommes d'argent dans notre cas, doit être réalisée préalablement à la seconde tâche.
 
\section{Travaux connexes}
\label{sec:quanta:biblio}
Chacune des tâches précédentes se rapproche d'une tâche couramment traitée en fouille de texte. En effet, la détection de catégories dans les décisions peut être modélisée comme un problème de classification de documents. La tâche d'extraction se rapproche plus quant à elle des problématiques comme l'extraction d'évènements, le remplissage de champs, ou encore l'extraction de relations et la résolution de référencement.

\subsection{Extraction d'éléments structurés}% avec des problématiques d'extraction d'information}

Les demandes ressemblent aux structures telles que les relations ou les évènements. En effet, les champs définis par la compétition d'Extraction Automatique de Contenus ACE (\textit{Automatic Content Extraction}), dans \citet{ace2005relation}, pour les relations et \citet{ace2005event} pour les évènements, se rapprochent de ceux visés lors de l'extraction des demandes comme l'illustre le Tableau \tableref{tab:quanta:analogie-relation-evt}. Plus précisément, une catégorie de demandes correspond à un type d'évènement ou de relation entre deux entités. Les arguments qui participent à l'évènement \og demande \fg{} ou à la relation \og demande-résultat \fg{} sont le quantum demandé et le quantum résultat. Le sens du résultat représente la classe de la structure \og demande \fg{}.

\begin{table}[ht]
	\small
	\begin{tabular}{|p{0.15\textwidth}|p{0.21\textwidth}|p{0.24\textwidth}|p{0.30\textwidth}|}
		\hline		%\textbf{Champs} 
		 & \textbf{Relation \citep{ace2005relation}}  & \textbf{Événement \citep{ace2005event}} & \textbf{Analogie chez les demandes} \\ \hline
		\textbf{Type} & Org-Aff.Student-Alum & Die & Catégorie="Dommages-intérêts pour procédure abusive" \\ \hline
		\textbf{Passage} (\textit{extend}) & \textit{Card graduated from the University of South Carolina}  & "Il est mort hier d'une insuffisance rénale."  & (\textit{Figure \ref{fig:quanta:expr-dmd-rst}}) \\ \hline
		\textbf{Déclencheur (\textit{trigger})} & - & "mort" & "procédure abusive"\\ \hline
		\textbf{Participants ou Arguments  (\textit{arguments})} & Arg1="Card" \linebreak Arg2="the University of South
		Carolina"& Victim-Arg="il" \linebreak Time-Arg="hier"  & Quantum-demandé="3000\euro{}"\linebreak  Quantum-obtenu="0 \euro{}"\ \\ \hline
		\textbf{Classes  (\textit{attributes, classes})} & Asserted & Polarity=POSITIVE, Tense=PAST & Sens-résultat="Rejeté" \\ \hline
	\end{tabular}
	\caption{Exemples d'analogie entre relations, évènements et demandes} \label{tab:quanta:analogie-relation-evt}
\end{table}

\subsection{Approches d'extraction d'éléments structurés}
\label{quanta:related-approaches}
L'extraction d'éléments structurés repose généralement sur une approche modulaire du problème qui le décompose en tâches plus simples. D'une part, on dispose de l'identification des déclencheurs\footnote{Terme-clés indiquant la présence d'un évènement \citep{ace2005event}.} et des arguments. D'autre part, une mise en correspondance relie les arguments et déclencheurs qui participent à la même relation ou au même évènement. Les classes peuvent être déterminées par classification du passage associé. Cette décomposition a permis à de nombreuses méthodes de voir le jour. 

L'approche traditionnelle consiste en une chaîne de traitements enchaînant des modules adaptés à une tâche simple. La sortie d'une étape est l'entrée de la suivante. C'est ainsi que \citet{ahn2006stages} définit un enchaînement de modèles de classification (k-plus-proches-voisins \citep{cover1967knn} vs. classificateur d'entropie maximum \citep{nigam1999maxent}), pour extraire des champs d'évènements dans le corpus d'ACE\citet{ace2005event}. Bien que les différents modules soient plus faciles à développer, ce type d'architecture souffre de la propagation d'erreurs d'une étape à la suivante, ainsi que de la non exploitation de l'interdépendance entre les tâches. Par conséquent, l'inférence jointe des champs est préconisée. Celle-ci peut-être réalisée par une modélisation graphique probabiliste ou neuronale. Par exemple, pour l'extraction d'évènements, \citet{yang2016jointEntityEvt} estiment la probabilité conditionnelle jointe du type d'entité $t_i$, les rôles des arguments $r_{i\cdot}$ et les types d'entités qui remplissent ces rôles $a.$ :  $p_\theta(t_i,r_{i\cdot},a. \vert i, N_i, x)$, $i$ étant un déclencheur candidat, $N_i$ l'ensemble des entités candidates qui sont de potentiels arguments pour $i$, et $x$ est le document. Cette approche obtient 50.6\% de $F_1$-mesure moyenne pour la détection des valeurs d'arguments et 48.4\% pour leur classification dans leur rôle respectif. Par ailleurs, \citet{nguyen2016jointtrgarg} illustrent l'utilisation des réseaux de neurones profonds avec une couche pour la prédiction du déclencheur, une autre pour le rôle des arguments, et la dernière encode la dépendance entre les labels de déclencheurs et les rôles d'arguments. Cette approche obtient 62.8\% de $F_1$-mesure moyenne pour la détection des valeurs d'arguments et 55.4\% pour leur classification dans leur rôle respectif. %\textcolor{red}{[PERFORMANCE DE LEUR METHODE]} 

L'annotation du corpus de l'ACE est un marquage des champs dans le texte, et par conséquent, la position ou l'occurrence des champs est indiquée (\og annotation au niveau du segment de mots \fg{}). Comme dans notre cas, les données peuvent être annotées dans un tableau, hors des textes d'où elles sont issues. Il est donc nécessaire de retrouver leur position sans supervision. \citet{palm2017e2e-dnn} proposent dans cette logique une architecture de réseaux de neurones point-à-point qu'ils ont expérimentés sur des corpus de requêtes de recherche de restaurant et films \citep{liu2013mitmovierestaurant} ou de réservation de billets d'avion \citep{price1990atis}. Ils se sont intéressés au problème de remplissage de champs en apprenant la correspondance entre les textes et les valeurs de sorties. Leur modèle est basé sur les réseaux de pointeurs \citep{vinyals2015pointernetworks} qui sont des modèles séquence-à-séquence avec attention, dans lesquels la sortie est une position de la séquence d'entrée. Le modèle proposé consiste en un encodeur de la phrase et des contextes, plusieurs décodeurs (un pour chaque champ). L'application de cette architecture à l'extraction des demandes serait confrontée à deux obstacles majeurs auxquels il faut répondre au préalable. Premièrement, les décisions judiciaires ont des contenus de plusieurs centaines à plusieurs milliers de lignes contrairement aux requêtes manipulées par \citet{palm2017e2e-dnn}  dont la plus longue ne comprend que quelques dizaines de mots. La complexité des architectures neuronales de TALN augmente rapidement en espace et en temps avec la longueur des documents manipulés. Deuxièmement, nous disposons de très peu de données annotées ; entre 23 et 198 documents annotés dans notre cas contre plusieurs milliers pour les expérimentations de \citet{palm2017e2e-dnn}.

L'avantage de l'utilisation des réseaux de neurones vient de leur capacité à apprendre automatiquement des caractéristiques pertinentes contrairement aux modèles probabilistes qui exigent très souvent une ingénierie manuelle des caractéristiques. Par contre, il est beaucoup plus facile d'utiliser les modèles probabilistes sur des corpus de faible taille et de longs textes comme c'est le cas pour le problème d'identification des demandes judiciaires.


\subsection{Extraction de la terminologie d'un domaine}
\label{sec:quanta:extract-terminologie-domaine}
L'identification des attributs peut être facilitée grâce à leur proximité avec des termes-clés caractéristiques des catégories de demandes au même titre que les \og déclencheurs \fg{} aident à identifier les évènements.
Ne disposant pas au préalable de la liste des termes pertinents pour l'extraction des demandes, il est possible de les apprendre. Il existe à cet effet plusieurs métriques statistiques de pondération de termes généralement employées en recherche d'information et en classification de texte comme méthodes de sélection de caractéristiques. Ces métriques sont qualifiées de poids globaux car calculées à partir des occurrences dans un corpus, à la différence des poids locaux (Tableau \ref{tab:sensresultat:metriq_locales}) calculés à partir des occurrences dans un document. Quelques métriques sont formulées ici en utilisant les notations du Tableau \ref{tab:quanta:notations_metriques} définies pour une base d'apprentissage.


%Les approches d'extraction de terminologie peuvent être organisés dans les principaux groupes suivants \citet{gomez2004overviewontologielearning,lossio2014biotexcvalue}: 
%\begin{enumerate}
%	\item les approches à base de techniques linguistiques qui consistent à reconnaitre les termes-clés à l'aide de motifs linguistiques \citep{gaizauskas2000linguistictermrecognition}. comme des expressions régulières de groupes grammaticaux à l'instar de la combinaison $(Adj\vert Nom)^+Nom$.
%	\item les approches à base de techniques statistiques qui permettent d'affecter des scores aux termes d'un corpus et par conséquent de les ranger par ordre de pertinence. On a pour exemple la métrique IDF (\textit{Inverse Document Frequency}) de \citet{sparck1972idf}. 
%	\item les approches à base d'algorithmes d'apprentissage automatique à l'instar des architectures d'apprentissage profond entrainées par \citet{gharbieh2017deeptermlearning} pour apprendre des expressions à mots multiples.
%	\item Les approches hybrides combinent différentes méthodes par exemple la méthode C-value \citep{frantzi2000CValueNCValue} réalise une sélection préalable de candidats par des filtres linguistiques, puis pondère les différents candidats pour en distinguer les plus pertinents.
%\end{enumerate}
%\subsubsection{Méthodes statistiques}
% LISTES POUR CHAQUE CATÉGORIE
\begin{table}[!htb]
	\centering
	\begin{tabular}{lp{0.8\textwidth}}
		\hline\noalign{\smallskip}
		Notation & Description \\
		\noalign{\smallskip}
		\hline
$t$ & un terme \\
$d$& un document \\
$\vert t \vert$ & longueur de $t$ (nombre de mots) \\
$c$ & la catégorie (domaine ciblé) \\
$\overline{c}$ & la classe complémentaire ou négative \\
$D$& ensemble global des documents de taille $N = \vert D \vert$ \\
$D_{c}$& ensemble des documents de $c$ de taille $\vert D_{c} \vert$ \\% = $N^+$\\
$D_{\overline{c}}$& ensemble des documents de $\overline{c}$ de taille $\vert D_{\overline{c}} \vert$\\%  = $N^-$\\
$N$& nombre total de documents\\
$N_{t}$& nombre de documents contenant $t$\\
$N_{\overline{t}}$& nombre de documents ne contenant pas $t$\\
$N_{t,c}$ & nombre de documents de $c$ contenant $t$\\
$N_{\overline{t},c}$ & nombre de documents de $c$ ne contenant pas $t$\\%   = $b$\\
$N_{t,\overline{c}}$ & nombre de documents de $\overline{c}$ contenant $t$\\%   = $c$\\
$N_{\overline{t},\overline{c}}$ & nombre de documents de $\overline{c}$ ne contenant pas $t$\\%   = $d$\\
%$\vert D \vert$ & nombre total de documents ($\vert D \vert = \vert D_{c} \vert + \vert D_{\overline{c}} \vert$)\\
%$DF_c$ & proportion de documents du corpus appartenant à $c$ (probabilité qu'un texte pris au hasard soit de la classe $c$ )\\
%$DF_t$& proportion de documents du corpus contenant $t$ (\textit{Document Frequency})\\
$DF_{t \vert c}$ & proportion de documents contenant $t$ dans le corpus de $c$ ($DF_{t \vert c} = \frac{N_{t,c}}{\vert D_c \vert}$) \\
$DF_{c \vert t}$ & proportion de documents appartenant à $c$ dans l'ensemble de ceux qui contiennent $t$  \\
\hline
\end{tabular}
\caption{Notation utilisée pour formuler les métriques} \label{tab:quanta:notations_metriques}
\end{table}

% Lorsque la fonction n'utilise que le corpus représentatif du domaine d'intérêt, elle mesure ainsi le degré d'importance du terme dans le langage d'un domaine particulier ou une classe particulière de documents. Par exemple, la fonction \og fréquence de document \fg{} (\textit{DF - Document Frequency}) mesure l'importance d'un terme en lui affectant la proportion de documents qui le contiennent dans un corpus global considéré. De telles fonctions sont qualifiées de non-supervisées contrairement à celles qui se calculent entre différents corpus  et par conséquent utilisent les labels des données d'entraînement \citep{lan2009termweighting, wu2017balancingtermweight}. Ces dernières mesurent l'importance du terme dans la discrimination (ou la différence) entre le domaine d'intérêt et les autres corpus. Un exemple simple s'obtient en faisant la différence de \og fréquence de document \fg{} d'un terme entre un corpus $c$ et sont complémentaire $\overline{c}$. 

\subsubsection{Métriques non-supervisées}
Les métriques non-supervisées affectent un score à un terme en rapport avec l'importance de ce dernier dans le corpus global $D$. Parmi ces métriques, on retrouve par exemple la fréquence inverse de document (\textit{inverse document frequency}) $idf$ \citep{sparck1972idf} et ses variantes $pidf$ \citep{wu1981pidf}  et $bidf$ \citep{jones2000bm25idf} accordent plus d'importance aux termes rares. Elles considèrent en fait qu'un terme rare est plus efficace pour la distinction entre des documents. Par conséquent, elles sont efficaces en recherche d'information mais moins indiquées en classification de textes où le but est plutôt de séparer des catégories \citep{wu2017balancingtermweight}. Elles se formulent comme suit :
\[idf(t) = \log_2\left(\frac{N}{N_t}\right), pidf(t) = \log_2\left(\frac{N}{N_t} - 1\right), bidf(t) = \log_2\left(\frac{N_{\overline{t}} + 0.5}{N_t + 0.5}\right)\]

Il est possible de prendre explicitement en compte le fait que les termes peuvent comprendre plusieurs mots (n-grammes) et avoir des tailles différentes (nombre de mots). La $\text{C-value}$ \citep{frantzi2000CValueNCValue}, par exemple, distingue la fréquence du terme et de ses sous-termes (termes imbriqués) par la formule : %  t \mbox{ n'est imbriqué dans aucun terme candidat}
\[\text{C-value}(t) = \begin{cases} \log_2(\vert t \vert) \cdot (N_t - \frac{1}{\vert T_t \vert} \cdot \sum\limits_{b \in T_t} N_b), & \mbox{si } t \mbox{ est imbriqué} \\ \log_2(\vert t \vert) \cdot N_t, & \mbox{sinon,} \end{cases}\]
$T_t$ étant l'ensemble des termes candidats qui contiennent $t$.


\subsubsection{Métriques supervisées}
\label{sec:quanta:poids-globaux-superv}
Les métriques supervisées mesurent l'information contenue dans les labels des documents de la base d'apprentissage. Pour un terme $t$, elles expriment généralement la différence de proportion qui existe entre les occurrences de $t$ dans $D_c$ et ses occurrences dans $D_{\overline{c}}$. Elles sont ainsi mieux adaptées à la distinction entre catégories. Parmi les nombreuses métriques existantes, nous avons expérimenté les suivantes : 
\begin{description}
	\item[La différence de fréquence] $\Delta_{DF}$ consiste simplement à calculer la différence entre les proportions de documents contenant $t$ respectivement dans $c$ et $\overline{c}$ :
	\[\Delta_{DF}(t,c) = DF_{t \vert c} - DF_{t \vert \overline{c}}\]
	\item[Le gain d'information] $ig$ \citep{yang1997IGandIMandCHIandTS} estime la quantité d'information apportée par la présence ou l'absence d'un terme $t$ sur l'appartenance d'un document à une classe $c$ :
	\begin{equation*} % A VERIFIER
	ig(t, c) = \splitdfrac{\frac{N_{t,c}}{N} \log_2 \left(\frac{N_{t,c}N}{N_{t}}\right)
		 + \frac{N_{\overline{t},c}}{N} \log_2 \left(\frac{N_{\overline{t},c}N}{N_{\overline{t}}\vert D_c \vert}\right)}
	{+ \frac{N_{t,\overline{c}}}{N} \log_2 \left(\frac{N_{t,\overline{c}}N}{N_{t}\vert D_{\overline{c}}\vert}\right)
	+ \frac{N_{\overline{t},\overline{c}}}{N} \log_2 \left(\frac{N_{\overline{t},\overline{c}}N}{N_{\overline{t}}\vert D_c \vert}\right)}
	\end{equation*}
	\item[La fréquence de pertinence] $rf$ \citep{lan2009rf} a comme intuition de considérer que  plus la fréquence d'un terme $t$ est élevée dans $D_c$ relativement à sa fréquence dans $D_{\overline{c}}$, plus il contribue à distinguer les documents de $c$ de ceux de $\overline{c}$. Elle est calculée par la formule :
	\[rf(t,c) = \log\left(2 + \frac{N_{t,c}}{max(1, N_{t,\overline{c}})}\right)\]
	\item[Le coefficient du $\chi^2$] \citep{schutze1995chi2} estime le manque d'indépendance entre $t$ et $c$. Par conséquent, une grande valeur de  $\chi^2(t,c)$ indique une relation étroite entre $t$ et $c$. Elle est calculée par la formule :
	\[\chi^2(t,c) = \frac{N ((N_{t,c} N_{\overline{t},\overline{c}}) - (N_{t,\overline{c}} N_{\overline{t},c}))^2}{N_t N_{\overline{t}} \vert D_c \vert  \vert D_{\overline{c}} \vert }\]
	\item[Le coefficient de corrélation $ngl$] de Ng, Goh et Low \citep{ng1997ngl} est la racine carré du $\chi^2$ \citep{schutze1995chi2} :
	\[ngl(t,c) = \frac{\sqrt{N} (N_{t,c} N_{\overline{t},\overline{c}}) - (N_{t,\overline{c}} N_{\overline{t},c})}{\sqrt{N_t N_{\overline{t}} \vert D_c \vert  \vert D_{\overline{c}} \vert }}.\]
	L'intuition est de ne regarder que les termes qui proviennent de $D_c$ et qui indiquent l'appartenance à $c$. Une valeur positive de $ngl$ signifie que $t$ est corrélé avec $c$, lorsqu'une valeur négative signifie que $t$ est corrélé à $\overline{c}$.
	\item[Le coefficient $gss$] de Galavotti, Sebastiani, et Simi
	 \citep{galavotti2000gss} est une fonction simplifiée du $ngl$ \citep{ng1997ngl} :
	\[gss(t,c) = (N_{t,c} N_{\overline{t},\overline{c}}) -  (N_{t,\overline{c}} N_{\overline{t},c}).\]
   Le facteur $N$ a été éliminé car il est le même pour tous les termes. Le facteur $\sqrt{N_tN_{\overline{t}}}$ est supprimé car il accentue les termes extrêmement rares qui ne sont pas efficaces pour la classification de textes. Le facteur  $\sqrt{\vert D_c \vert \vert D_{\overline{c}} \vert}$ est éliminé car il accentue les catégories extrêmement rares, ce qui tend à réduire l'efficacité micro-moyennée (efficacité calculée globalement sur le corpus de test sans distinction du label des éléments).
   \item[Le coefficient de Maracuilo] ($mar$)  \citep{marascuilo1966multcomparison} qui se calcule par la formule :
    \begin{equation*} mar(t, c) =  
    \,
    \dfrac{\left(
    	\splitdfrac{\splitdfrac{\splitdfrac{(N_{t,c} - N_{t}N_{t,c}/N)^2}{+ (N_{t,\overline{c}} - N_{t} \vert D_{\overline{c}} \vert /N)^2}}{+ (N_{\overline{t},c} - \vert D_c \vert N_{\overline{t}}/N)^2}}{+ (N_{\overline{t},\overline{}} - N_{\overline{t}} \vert D_{\overline{c}} \vert /N)^2}\right)}{N}.
    \end{equation*}
    C'est un test de proportion multivariée. Nous proposons de l'utiliser pour comparer les proportions d'occurrences d'un terme $t$ dans différents corpus. % Autrement dit, il s'agit de tester l'homogénéité des textes du corpus contenant $t$. %Lorsque $ mar(t, c) \geq 3.84$ on accepte l'hypothèse selon laquelle la proportion de textes pour lesquels $t$ prédit $c$ est significative avec un risque d'erreur de $5\%$.\textcolor{red}{référence**}
    %\item[La distance de Kullback-Leibler] $kld$ se formule comme suit:
    %\[kld(t, c)=(N_{t,c} / N_{t}) * \log (\frac{N_{t,c} N}{N_{t}\vert D_c \vert})\]
    %adaptée pour la classification de texte par \citet{bigi2003kld} est la métrique symétrique de la divergence de Kullback-Leibler ou entropie relative qui mesure comment deux distributions de probabilité sont différents
    \item[Le \og delta lissé d'$idf$\fg{}], $dsidf$ \citep{paltoglou2010dsidfANDdbidf}, est une version lissée du delta $idf$ ($didf$) de \citet{martineau2009didf} ($didf(t,c)=\log_2\left(\frac{\vert D_{\overline{c}} \vert N_{t,c}}{\vert D_c \vert N_{t,\overline{c}}}\right)$). $dsidf$ se formule comme suit :
    \[dsidf(t,c) = \log_2\left(\frac{\vert D_{\overline{c}} \vert (N_{t,c} + 0.5)}{\vert D_c \vert (N_{t,\overline{c}} + 0.5)}\right)\].
    \item[Le delta BM25 d'$idf$], $dbidf$ \citep{paltoglou2010dsidfANDdbidf}, est une autre variante plus sophistiquée du $didf$ qui se calcule comme suit :
    \[dbidf(t,c) = \log_2\left(\frac{( \vert D_{\overline{c}} \vert  - N_{t,\overline{c}} + 0.5) \vert (N_{t,c} + 0.5)}{(\vert D_c \vert - N_{t,c} + 0.5) (N_{t,\overline{c}} + 0.5)}\right)\]
\end{description}

\subsubsection{Discussions}
A l'exception de la $\text{C-value}$, ces métriques ne tiennent pas explicitement compte de la taille des termes dans les situations où on souhaiterait manipuler des termes de tailles différentes. \citet{brown2013ngram1100languages} propose que soit affecté à un n-gramme $t$ le poids $\left(\frac{N_t}{N}\right)^{0.27} \vert t \vert^{0.09}$, une formule obtenue empiriquement pour l'identification du langage d'un document. Par ailleurs, la méthode $\text{C-value}$ \citep{frantzi2000CValueNCValue} propose un produit similaire avec le logarithme de la longueur à la place des puissances. Il est par conséquent évident que le produit lissé de la longueur du terme (puissance ou logarithme) avec les métriques décrites précédemment, permet de favoriser les longs termes qui, bien que rares, sont très souvent plus pertinents que certains termes plus courts. Aussi, le temps  pour calculer ces différentes métriques devient rapidement long, surtout pour des $n$-grammes de mots de taille variée (nombre de mots). Pour compter rapidement les occurrences des n-grammes des corpus, nous avons utilisé la librairie SML\footnote{\url{http://www.semantic-measures-library.org}} \citep{harispe2013semlib} lors des expérimentations.


\section{Méthode}

\subsection{Détection des catégories par classification}
\label{sec:quanta:classification}
Étant donné l'ensemble $D_{\overline{c}}$ des documents ne comprenant aucune demande de la catégorie d'intérêt $c$, nous proposons de modéliser la tâche de détection des catégories en une tâche de classification de documents. Pour chaque catégorie $c$, un modèle de classification binaire est entraîné pour déterminer si un document $d$ contient une demande de la catégorie $c$. Nous avons particulièrement expérimenté quatre algorithmes traditionnellement utilisés comme approches de base. Il s'agit du classifieur bayésien naïf \citep{duda1973patternclass}, de l'arbre de décision C4.5 \citep{quinlan1993c4.5}, des $k$-plus-proches-voisins ($k$NN) \citep{cover1967knn}, de la machine à vecteurs de support (SVM) \citep{vapnik1995statlearning}. Ces algorithmes sont décrits en détail dans le chapitre \ref{chap:sensresultat} qui est axé sur la classification des documents. Les labels utilisés correspondent aux catégories d'intérêt. Par exemple, un document sera labellisé $danais$ s'il contient des demandes de dommages-intérêts pour abus de procédure, et $nodanais$ sinon. Etant donné le grand nombre de métriques de pondération existantes, la métrique choisie est celle qui fournit la meilleure performance sur les données d'apprentissage.

\subsection{Extraction basée sur la proximité entre sommes d'argent et termes-clés}
\label{sec:quanta:extraction}

Diverses approches d'extraction d'information existent (\ref{quanta:related-approaches}). Il est important de proposer dans un premier temps une approche basique explorant la solvabilité du problème du fait de ses multiples spécificités dont l'annotation d'une seule catégorie dans un document qui en contient plusieurs, l'annotation dans un tableau et donc à l'extérieur du document, la très faible quantité des données annotées, la multiplicité des demandes et des catégories dans un même document. Par conséquent, nous proposons ici une chaîne d'extraction à base de termes-clés, applicable pour chaque catégorie de demande. Il s'agit d'une approche qui tente de reproduire une lecture naïve du document en se basant sur des expressions couramment employées pour énoncer les demandes et résultats. La méthode consiste en deux phases dont une phase d'apprentissage des termes-clés de la catégorie, à proximité desquels seront identifiés les attributs durant la phase d'extraction des demandes comme l'illustre la \figureref{fig:quanta:exemple-proximite}. On remarque en effet que, naïvement, le seul fait que $1500$ euros soit aussi proche des termes-clés \textit{amende civile} et \textit{pour procédure abusive} signifie bien qu'il s'agit du quantum demandé comme amende civile pour procédure abusive.

\begin{figure}[!htb]
	\small
	\centering
	\begin{subfigure}[t]{0.95\textwidth}
		\fbox{\parbox{\textwidth}{
				" ... 
				
				- débouter M. S. de ... % l' ensemble de ses demandes
				
				\textbf{- le condamner à payer une amende civile de 1.500 euros pour procédure abusive} ...
				
				- le condamner à payer la somme ..."
		}}
		\caption{Extrait original d'un énoncé de demande avant marquage}\label{fig:quanta:exemple-proximite-original}
	\end{subfigure} 
	
	
	\begin{subfigure}[t]{0.95\textwidth}
		\fbox{\parbox{\textwidth}{
				" ... 
				
				- débouter M. S. de ... % l' ensemble de ses demandes
				
				- le \textbf{<demande categorie="acpa">}\underline{condamner} à payer une <terme-clef categorie="acpa">\textbf{amende civile}</terme-clef> de <argent> \textbf{1.500 euros} </argent> <terme-clef categorie="acpa"> \textbf{pour procédure abusive}</terme-clef> ...
				
				- le\textbf{</demande>} \underline{condamner} à payer la somme ..."
		}}
		\caption{Énoncé, sommes d'argent, et termes-clés marqués}\label{fig:quanta:exemple-proximite-marquage}
	\end{subfigure}
	%\caption{Exemple de passage de demande: le quantum demandé est une somme d'argent à proximité des terme-clés}
	\caption{Illustration de la proximité des quantas et termes-clés}
	\label{fig:quanta:exemple-proximite}
\end{figure} 

\subsubsection{Pré-traitement}

Le pré-traitement est nécessaire pour :
\begin{enumerate}
	\item sectionner le document en 4 sections Entête, Litige, Motifs, Dispositif ;
	\item annoter les sommes d'argent (en chiffre) à l'aide de l'expression régulière \og \texttt{[0-9]([0-9]|[',.]|\textbackslash s)*\textbackslash s*([Ee]uro[s]\{0,1\} |franc[s]\{0,1\} \\ |\euro|F|XPF|CFP|EUR|EUROS|[i])( |\$)}\fg{} ;
	\item annoter les énoncés de demandes et de résultats respectivement dans les sections Litige et Dispositif:  pour cela, les mots introductifs du Tableau \tableref{tab:quanta:mots-introductifs} sont employés car ils indiquent le début d'un énoncé indépendamment de la catégorie; 	cette technique de recherche de passages à l'aide de listes de termes-clés prédéfinies a déjà été employé par  \citet{wyner2010extractlegalelts} pour annoter les énoncés de résultats en considérant toute phrase contenant un terme de jugement : \textit{affirm, grant, deny, reverse, overturn, remand, ...}
	 
	\begin{table}[!htb]
		\centering
		%\small
		%\textcolor{red}{ajouter une colonne résultat\_a}
		\begin{tabular}{|p{0.28\textwidth}|p{0.3\textwidth}|p{0.12\textwidth}|p{0.11\textwidth}|}
			\hline
			\textbf{Demande} & \multicolumn{3}{c|}{\textbf{Résultat} (organisé par polarité ou sens)} \\ \hline
			& \textbf{accepte}  &\textbf{sursis à statuer} & \textbf{rejette}  \\ \hline
			\textit{accorder, admettre, admission, allouer, condamnation, condamner, fixer, laisser, prononcer, ramener, surseoir} & \textit{accorde, accordons, admet, admettons, alloue, allouons, condamne, condamnons, déclare, déclarons, fixe, fixons, laisse, laissons, prononce, prononçons} & \textit{réserve, réservons, sursoit, sursoyons} & \textit{déboute, déboutons, rejette, rejetons} \\ \hline
		\end{tabular}
		\caption{Mots introduisant les énoncés de demandes et de résultats}\label{tab:quanta:mots-introductifs}
	\end{table}

\end{enumerate} 

\subsubsection{Apprentissage des termes-clés d'une catégorie}
Les termes-clés sont identifiés à l'aide de méthodes statistiques d'extraction ou sélection de terminologie. La base d'apprentissage comprend les corpus $D_c$ et $D_{\overline{c}}$ dont les documents ont été pré-traités. % La méthode est sémi-supervisé car ne disposant pas d'exemples de termes-clés, des échantillons de documents labellisés dans les catégories $c$ et $\overline{c}$ sont disponibles. 
 Le processus d'apprentissage des termes se déroule comme suit :
 
 \begin{enumerate}
 	\item Restreindre le contenu de chaque document de $D_c$ à la concaténation des énoncés de demande et résultats contenant des sommes d'argent de valeur égale à celle des quanta annotés.
 	\item Restreindre chaque document de  $D_{\overline{c}}$ à la concaténation des énoncés  de demande et résultats contenant des sommes d'argent.
 	\item A l'aide d'une métrique global $g$, calculer le score des termes du corpus $D_c \cup D_{\overline{c}}$. Ce score est multiplié par le logarithme de la longueur du terme pour favoriser les termes longs: $g'(t,c) = \log_2(\vert t \vert) \times g(t, c)$.
 	\item Normaliser les scores en appliquant à chaque score original ($g'(t,c)$) la formule $g'_{norm}(t,c) = \frac{g'(t,c) - \min\limits_{t_k} (g'(t_k,c))}{\max\limits_{t_k} (g'(t_k,c)) - \min\limits_{t_k} (g'(t_k,c))}$. 
 	\item Trier les termes par ordre décroissant de score.
 	\item Sélectionner les premiers termes qui obtiennent les meilleurs performances sur la base d'apprentissage.
 \end{enumerate}


\subsection{Application de l'extraction à de nouveaux documents}
A l'aide des termes-clés appris, l'extraction des données de couples demandes-résultats se déroule comme suit:
\begin{enumerate}
	\item reconnaître et marquer les occurrences des termes dans le document;
	\item extraire les quanta demandés ($q_d$) et résultats ($q_r$) à proximité des termes-clés respectivement dans les énoncés de demande et résultat qui contiennent des sommes d'argent et un terme-clé;
	\item le mot introductif de l'énoncé résultat indique le sens du résultat ($s_r$) tel que catégorisé dans le Tableau \ref{tab:quanta:mots-introductifs};
	\item relier les attributs ($q_d, s_r, q_r$) de chaque paire demande-résultat:% longcommonsubsequence puis ordre des quanta 
	\begin{enumerate}
		\item former les paires (énoncé de demande, énoncé de résultat) similaire (nous utilisons la métrique de \og la plus longue sous-séquence commune \fg{} \citep{hirschberg1977algorithms_LCS, bakkelund2009lcs}) 
		\item pour chaque paire d'énoncés formée, relier les quanta demandés et quanta résultats en considérant que les quanta correspondants apparaissent dans le même ordre dans les deux énoncés.
	\end{enumerate}
\end{enumerate}

\section{Résultats expérimentaux}

Nous analysons ici la capacité de l'approche proposée à reconnaître efficacement les catégories de demandes présentes dans les documents, et à extraire les valeurs des attributs des différentes paires demandes-résultats qui y sont exprimées.  Sont discutés les données et métriques d'évaluation employées, ainsi que des résultats expérimentaux observés avec des exemples annotés pour les six catégories du Tableau \ref{tab:quanta:exemple-categorie}. 

\subsection{Données d'évaluation}
L'annotation manuelle d'exemples s'effectue pour une catégorie à la fois afin que la tâche soit plus facile pour les experts. Le protocole d'annotation se déroule en 3 étapes: 
\begin{enumerate}
	\item définir une catégorie $c$ par son objet et sa norme juridique;
	\item former un corpus $D_c$ de documents contenant des demandes de $c$, et un autre $D_{\overline{c}}$ de documents n'en contenant pas; 
	\item extraire toutes les demandes de catégories $c$ mentionnées dans $D_c$, pour annoter les données des paires demande-résultat dans un tableau comme celui illustré par le  Tableau \ref{tab:tab:quanta-annotations};
	
	\begin{table}[!htb]
		\includegraphics[width=\textwidth]{tab-annotations.png}
		\scriptsize{Les noms des champs sont sur les 2 premières lignes et les demandes sont données en exemple pour la catégorie \textit{dommages-intérêts sur le fondement de l'article 700 du code de procédure civile} (décision 14/06911 de la cour d'appel de Lyon).}
		\caption{Extrait du tableau d'annotations manuelles des demandes.} \label{tab:tab:quanta-annotations}
	\end{table}
\end{enumerate}

%Le résultat d'une phase d'annotation comprend ainsi un tableau des demandes, deux corpus de décisions $D_{c}$ et $D_{\overline{c}}$.

 %Toutes les demandes du corpus $D_{c} \cup D_{\overline{c}}$  annoté manuellement, sont considérées inscrites dans le tableau des annotations manuelles.
 
 La répartition des données d'évaluation est donnée par la Figure \ref{fig:quanta:hist-repartition-docs}.  
 
 \begin{figure}[!htb]
 	\includegraphics[width=\textwidth]{chartDataset.png}
 	\caption{Répartitions des demandes dans les documents annotées.}\label{fig:quanta:hist-repartition-docs}
 \end{figure}
 
 Il faut aussi noter que bien que l'annotation manuelle des demandes et des résultats soit réalisée dans un tableau (annotation externe au contenu), elle reste une tâche très difficile. Le très faible nombre de documents annotés manuellement en témoigne. Le nombre  maximum de documents annotés pour une catégorie est seulement de 198 (barres vertes de \textit{danais}). 

\subsection{Métriques d'évaluation}
\paragraph{Reconnaissance de catégories par classification}

La classification des documents est évaluée en utilisant les métriques précision (P), rappel (P), f1-mesure (F1). %\textcolor{red}{Quelle moyenne: macro / micro?}

\paragraph{Extraction des attributs des paires demande-résultat}
 Nous évaluons les approches proposées sur l'extraction de 3 données: le quantum demandé $q_d$, le sens du résultat $s_r$ et le quantum obtenu $q_r$. Une demande est donc un triplet $(q_d, s_r, q_r)$. Il est possible d'évaluer le système pour un sous-ensemble $x$ de $\lbrace q_d, s_r, q_r \rbrace$ sur les demandes extraites d'un corpus annotées $D$ de test. Nous utilisons les métriques traditionnellement employées en extraction d'information: la précision ($Precision_{c,x,D}$), le rappel ($Rappel_{c,x,D}$), et la F1-mesure ($F1_{c,x,D}$). Ces mesures sont définies à partir des nombres de vrais positifs ($TP$), faux positifs ($FP$) et faux négatifs ($FN$) calculés au niveau d'un document $d$:
\begin{itemize}
\item  $TP_{c, x, d}$ est le nombre de demandes extraites de $d$ par le système, qui sont effectivement de la catégorie $c$ (demandes correctes);
\item  $FP_{c, x, d}$ est le nombre de demandes extraites de $d$ par le système, mais qui ne sont pas des demandes de $c$ (demandes en trop);
\item  $FN_{c, x, d}$ est le nombre de demandes annotées comme étant de $c$ mais qui n'ont pas pu être extraites par le système (demandes manquées).
\end{itemize}

Au niveau d'un corpus d'évaluation $D$, ces métriques sont sommées: 
$TP_{c,x,D} = \sum\limits_{d \in D} TP_{c,x,d}$ \hfill $FP_{c,x,D} = \sum\limits_{d \in D} FP_{c,x,d}$ \hfill $ FN_{c,x,D} = \sum\limits_{d \in D} FN_{c,x,d}$.

Une donnée observée (par exemple \og 3 000 \euro \fg) est bien extraite automatiquement si sa valeur (le nombre $3000$) correspond à celle du quantum annoté dans le tableau. Nous considérons que les unités monétaires, entre les quanta extraits et ceux manuellement annotés, sont identiques.

\subsection{Détection des catégories par classification}
Les implémentations de la bibliothèque Weka \citep{frank2016weka} ont permis d'utiliser plusieurs modèles de classification: le modèle Bayésien naïf (NB), l'arbre de décision C4.5 (implémenté sous l'appelation J48), les k-plus-proches-voisins (KNN), et le SVM. 
 A chaque entraînement, s'exécute une sélection de modèles par validation croisée sur les données d'entraînement. Elle a pour but de sélectionner la métrique locale et la métrique globale appropriée. Les résultats obtenus par 5-\textit{folds} validation croisée sont présentés sur le Tableau \ref{tab:quanta:resultat-detect-cat}.  
 
\begin{table}[!h]
	\scriptsize
	\begin{center}
	\begin{tabular}{l|ccc|ccc|ccc|ccc}
		\hline\noalign{\smallskip}
		&   \multicolumn{3}{|c|}{NB}    &    \multicolumn{3}{|c|}{C4.5}   &  \multicolumn{3}{|c|}{KNN}  & \multicolumn{3}{|c}{SVM}     \\       
		\noalign{\smallskip}
		\hline
		\noalign{\smallskip}
		%Categorie  
		& $P$     & $R$     & $F_1$    & $P$     & $R$     & $F_1$    & $P$     & $R$     & $F_1$    & $P$     & $R$     & $F_1$    \\        
		\noalign{\smallskip}
		\hline
		\noalign{\smallskip}
		acpa    & 1.0 & 1.0 & 1.0 & 0.996 & 0.955 & 0.972 & 1.0 & 1.0 & 1.0 & 0.996 & 0.955 & 0.972 \\
		concdel & 1.0 & 1.0 & 1.0 & 1.0 & 1.0 & 1.0 & 1.0 & 1.0 & 1.0 & 0.995 & 0.967 & 0.979 \\
		danais  & 0.988 & 0.989 & 0.988 & 0.996 & 0.995 & 0.995 & 0.995 & 0.995 & 0.995 & 0.993 & 0.993 & 0.993 \\
		dcppc   & 1.0 & 1.0 & 1.0 & 1.0 & 1.0 & 1.0 & 1.0 & 1.0 & 1.0 & 1.0 & 1.0 & 1.0 \\
		doris   & 1.0 & 1.0 & 1.0 & 1.0 & 1.0 & 1.0 & 1.0 & 1.0 & 1.0 & 1.0 & 1.0 & 1.0 \\
		styx    & 1.0 & 1.0 & 1.0 & 0.984 & 0.983 & 0.983 & 1.0 & 1.0 & 1.0 & 1.0 & 1.0 & 1.0 \\
		\hline
	\end{tabular}
\end{center}

$P$= Précision, $R$=Rappel, $F_1$ = $F_1$-mesure
	\caption{Evaluation de la détection de catégories.}\label{tab:quanta:resultat-detect-cat}
\end{table}

D'après les résultats, la tâche 1 est relativement aisée pour les algorithmes traditionnels qui détectent parfaitement la présence ou non d'une catégorie dans les documents. Par conséquent, pour toute catégorie $c$, les résultats de l'extraction, dans la suite, ne sont discutés que pour les documents de $c$, car, grâce à l'efficacité de la phase de classification, aucun document de $\overline{c}$ ne sera traité par la phase d'extraction.

\subsection{Extraction de données des paires demandes-résultats}
Les scores des termes-clés candidats étant normalisés, si on sélectionne les termes dont les scores sont supérieurs à un seuil fixé, on remarque que chaque métrique d'extraction a un niveau d'efficacité différent entre les catégories de demande (\tableref{tab:quanta:compareGW} avec 0.5 comme  seuil fixé). 

\begin{table}[!htb]
	\centering \small
	\begin{tabular}{|c|c|c|c|c|c|c|c|}
		\hline
		& \textit{acpa} & \textit{concdel} & \textit{danais} & \textit{dcppc} & \textit{doris} & \textit{styx} & \textbf{Moyenne} \\ \hline
		$bidf$ & 37.33 & 32.73 & 23.96 & 20.46 & 8.08 & 28.43 & 25.17 \\ \hline
		$\chi^2$ & \textbf{54.55} & 25.88 & 43.97 & 28.35 & 13.11 & 52.73 & 36.43 \\ \hline
		$dbidf$ & 37.58 & 24.63 & 56.25 & \textbf{29.06} & 11.58 & 52.73 & 35.31 \\ \hline
		$\Delta_{DF}$ & \textbf{54.55} & 25.55 & 48.16 & 28.1 & \textbf{19.64} & 52.73 & \textbf{38.12} \\ \hline
		$dsidf$ & 37.58 & 25.25 & \textbf{56.42} & 26.05 & 8.72 & \textbf{53.46} & 34.58 \\ \hline
		$gss$ & \textbf{54.55} & 25.11 & 48.16 & 28.1 & \textbf{19.64} & 52.73 & \textbf{38.05} \\ \hline
		$idf$ & 38.78 & 32.73 & 22.31 & 20.53 & 8.27 & 25.22 & 24.64 \\ \hline
		$ig$ & 4 & 12.4 & 45.21 & 14.99 & 16.74 & 51.13 & 24.08 \\ \hline
		$marascuilo$ & \textbf{54.55} & 23.65 & 43.97 & 26.67 & 17.91 & 52.73 & 36.58 \\ \hline
		$ngl$ & 42.02 & 23.97 & 52.31 & 27.21 & 13.29 & 53.2 & 35.33 \\ \hline
		$pidf$ & 26.19 & \textbf{33.71} & 21.83 & 20.46 & 8.76 & 27.68 & 23.11 \\ \hline
		$rf$ & 41.11 & 33.09 & 55.72 & 28.56 & 14.93 & 51.23 & 37.44 \\ \hline
	\end{tabular}
%\caption{$F1_{c,(q_d, s_r, q_r), D_c}$ moyenne pour une 5-\textit{fold} validation croisée pour chaque métrique de sélection de termes pour un seuil égal à $0.5$}
\caption{Comparaison des pondérations globales suivant la $F_1$-mesure.} \label{tab:quanta:compareGW}
\end{table}

Par conséquent, la métrique et le seuil doivent être bien sélectionnés en fonction de la catégorie de demandes traitée. En choisissant, pour ces méta-paramètres, les valeurs les plus efficaces pour l'extraction sur la base d'apprentissage, les résultats du \tableref{tab:quanta:resultDetailExtraction} sont observés. Les améliorations sont à noter notamment pour trois catégories. Le score $F_1$ sur l'extraction des triplets $(q_d,s_r, q_r)$ passe de 54.55 au maximum  à 58.99 (plus de 4\%) pour \textit{acpa}, de 29.06 à 29.41 pour \textit{dcppc}, de 19.64 à 29.08 (près de 10\%) pour \textit{doris}. Les baisses de performances observées pour les autres catégories est comparativement très faibles (moins de 2\%).
 

\begin{table}[!htb]
\footnotesize
\begin{center}
	\begin{tabular}{l|l|l|llll|llll}
			 \hline
		&                 &                       &          \multicolumn{4}{c|}{Données d'entraînement}      &      \multicolumn{4}{c}{Données de test}       \\ \hline
		$c$ & Données & $\vert V_c \vert$ & $P$     & $R$     & $F_1$                       & \%Docs & $P$     & $R$     & $F_1$              & \%Docs \\ \hline
		\multirow{5}{*}{\textit{acpa}}  & $q_d$ & 1 & 86.4 & 56.37 & 68.13 & 56.37 & 68.33 & 54 & 58.99 & 46 \\
		& $q_r$ & 1 & 100 & 65.09 & 78.74 & 65.09 & 93.33 & 63 & 71.43 & 55 \\
		& $s_r$ & 1 & 100 & 65.09 & 78.74 & 65.09 & 93.33 & 63 & 71.43 & 55 \\
		& $(s_r, q_r)$ & 1 & 100 & 65.09 & 78.74 & 65.09 & 93.33 & 63 & 71.43 & 55 \\
		& $(q_d,s_r, q_r)$ & 1 & 86.4 & 56.37 & 68.13 & 56.37 & 68.33 & 54 & 58.99 & 46 \\ \hline
		\multirow{5}{*}{\textit{concdel}}  & $q_d$ & 26 & 49.33 & 44.02 & 45.31 & 24.17 & 73.2 & 29.72 & 33.29 & 26.67 \\
		& $q_r$ & 26 & 48.3 & 42.66 & 44.1 & 22.5 & 75.73 & 28.89 & 34.3 & 26.67 \\
		& $s_r$ & 26 & 46.52 & 40.89 & 42.36 & 22.5 & 74.93 & 26.39 & 33.09 & 26.67 \\
		& $(s_r, q_r)$ & 26 & 46.52 & 40.89 & 42.36 & 22.5 & 74.93 & 26.39 & 33.09 & 26.67 \\
		& $(q_d,s_r, q_r)$ & 26 & 42.43 & 37.41 & 38.68 & 20.83 & 68.27 & 23.06 & 28.65 & 23.33 \\ \hline
		\multirow{5}{*}{\textit{danais}} & $q_d$ & 37 & 77.71 & 48.71 & 59.68 & 37.3 & 79.25 & 47.5 & 59 & 37.3 \\
		& $q_r$ & 37 & 77.68 & 48.71 & 59.67 & 37.03 & 77.78 & 46.46 & 57.79 & 36.22 \\
		& $s_r$ & 37 & 77.05 & 48.33 & 59.19 & 37.03 & 77.78 & 46.46 & 57.79 & 36.22 \\
		& $(s_r, q_r)$ & 37 & 77.05 & 48.33 & 59.19 & 37.03 & 77.78 & 46.46 & 57.79 & 36.22 \\
		& $(q_d,s_r, q_r)$ & 37 & 74.45 & 46.65 & 57.16 & 35.81 & 74.41 & 44.38 & 55.23 & 34.59 \\ \hline
		\multirow{5}{*}{\textit{dcppc}}   & $q_d$ & 35 & 45.71 & 36.64 & 40.66 & 34.05 & 44.64 & 40.73 & 41.75 & 31.4 \\
		& $q_r$ & 35 & 78.99 & 63.21 & 70.2 & 59.33 & 75.48 & 64.51 & 68.41 & 53.82 \\
		& $s_r$ & 35 & 84.73 & 67.85 & 75.33 & 63.24 & 81.21 & 69.14 & 73.51 & 57.43 \\
		& $(s_r, q_r)$ & 35 & 78.99 & 63.21 & 70.2 & 59.33 & 75.48 & 64.51 & 68.41 & 53.82 \\
		& $(q_d,s_r, q_r)$ & 35 & 34.2 & 27.39 & 30.41 & 28.03 & 31.66 & 28.55 & 29.41 & 25.37 \\ \hline
		\multirow{5}{*}{\textit{doris}}  & $q_d$ & 8 & 31.98 & 35.76 & 32.94 & 7.75 & 37.48 & 35.9 & 36.63 & 7.12 \\
		& $q_r$ & 8 & 35.73 & 39.72 & 36.69 & 8.63 & 39.43 & 38.47 & 38.89 & 7.12 \\
		& $s_r$ & 8 & 35.06 & 39.56 & 36.24 & 9.06 & 42.91 & 41.44 & 42.12 & 8.94 \\
		& $(s_r, q_r)$ & 8 & 32.61 & 36.16 & 33.45 & 8.2 & 38.14 & 37.04 & 37.54 & 7.12 \\
		& $(q_d,s_r, q_r)$ & 8 & 24.48 & 27.16 & 25.13 & 5.61 & 29.7 & 28.53 & 29.08 & 7.12 \\ \hline
		\multirow{5}{*}{\textit{styx}} & $q_d$ & 4 & 69.34 & 59.55 & 64.04 & 33.5 & 69.3 & 59.49 & 63.61 & 32 \\
		& $q_r$ & 4 & 75.87 & 65.17 & 70.08 & 31.5 & 74.86 & 64.08 & 68.63 & 28 \\
		& $s_r$ & 4 & 75.87 & 65.17 & 70.08 & 31.5 & 74.86 & 64.08 & 68.63 & 28 \\
		& $(s_r, q_r)$ & 4 & 75.87 & 65.17 & 70.08 & 31.5 & 74.86 & 64.08 & 68.63 & 28 \\
		& $(q_d,s_r, q_r)$ & 4 & 57.61 & 49.44 & 53.19 & 25.5 & 57.24 & 48.36 & 52.08 & 24  \\
			\hline\noalign{\smallskip}
	\end{tabular}
\end{center}

$P$ = Précision, $R$ = Rappel, $F_1$ = $F_1$-mesure
	
\%Docs: proportion de documents dont l'ensemble des données extraites est égale à l'attendu (documents parfaitement traités)

$\vert V_c \vert$: nombre moyen de termes-clés identifiés pour la catégorie $c$
\caption{Résultats détaillés pour l'extraction des données avec sélection automatique de la méthode d'extraction des termes-clés} \label{tab:quanta:resultDetailExtraction}
\end{table}


Ces résultats détaillés font remarquer que les attributs, pris individuellement, présentent d'assez bonnes performances. Cependant, la mise en correspondance des attributs peine toujours à montrer des performances du même rang. On remarque néanmoins que les scores $F_1$ des triplets $(q_d,s_r, q_r)$ sont proches de celles des attributs qui présentent le plus de difficulté. En effet, la sélection préalable permet d'observer sur les données de test, des $F_1$-mesures comprises entre 33.09 \% et 71.43 \% pour les champs $q_d$, $q_r$, et $s_r$, et entre 28.65 \% et 58.99 \% pour les triplets $(q_d,s_r, q_r)$. L'échec de l'extraction des attributs est une des principales causes des faibles performances observées pour la liaison des attributs de paires similaires demande-résultat. Par ailleurs, les données sur le résultat, $s_r$ et $q_r$, sont en générale plus faciles à extraire($F_1$-mesures entre 42.12 et 71.43 sauf pour \textit{concdel}) que le quantum demandé $q_d$ ($F_1$-mesures entre 41.75 et 63.61 sauf pour \textit{concdel}). Il est aussi bien de noter qu'une plus grande quantité d'exemples annotés de documents ne semble pas être la garantie d'une meilleure extraction. On remarque en effet que les meilleures performances sont obtenues pour la catégorie disposant du plus faible nombre d'exemples annotés (\textit{acpa}) avec en moyenne un seul terme-clé appris. Remarquons aussi que la précision est en général nettement supérieure au rappel; ce qui signifie que la méthode a plus tendance à éviter les valeurs erronées au risque de manquer un nombre important de valeurs correctes. Enfin, la proportion de documents parfaitement traités (dans lesquels toutes les demandes de la catégorie ont été extraites) reste inférieur à la moyenne même pour \textit{acpa} donc le corpus ne comprend que des décisions à une seules demande de cette catégorie. L'unique terme-clef appris ne semble pas suffisant pour identifier toutes les demandes de \textit{acpa} (l'\og article 32-1 \fg{} est un bon indicateur par exemple). La catégorie \textit{doris} enregistre la plus faible valeur pour cette proportion, probablement à cause de la présence dans une même décision de plusieurs demandes pour des raisons très variés du trouble du voisinage (préjudice moral, nuisance sonore, préjudice matériel, préjudice de jouissance, etc.) donc malheureusement les termes-clés ne sont pas tous captés par les méthodes statistiques employées (cf. \tableref{tab:quanta:exemples_termes}). %Dans la suite, divers aspects sont décrits pour expliquer les raisons des limites de l'approche proposée.

\subsection{Analyse des erreurs}

En extraction d'éléments structurés, on retrouve trois types d'erreurs \citep{yang2016jointEntityEvt}: les données manquées (faux négatifs), les données en plus des attendues (faux positifs), et les mauvaises classifications (confusions). La confusion n'est pas discutée ici car les annotations ne sont faites que pour une seule classe. %Les confusions sont évitées dans l'approche proposée par la détection préalable des catégories par classification et l'extraction restreinte à une seule catégorie à la fois. 
Etant donné que la précision est en général supérieure au rappel, il est certain que les erreurs sont majoritairement dues aux données manquées comme le confirme le \tableref{tab:quanta:types_erreurs}. 

\begin{table}[!htb]
	\centering\small
	\begin{tabular}{|c|c|c|c|c|}
		\hline
		      &    \multicolumn{2}{|c|}{Données d'entraînement}      &      \multicolumn{2}{|c|}{Données de test}       \\ \hline
		& \textbf{\%erreurs FP} & \textbf{\%erreurs FN} & \textbf{\%erreurs FP} & \textbf{\%erreurs FN} \\ \hline
		$q_d$            & 36.90                 & 63.10                 & 36.52                 & 63.48                 \\ \hline
		$q_r$            & 32.30                 & 67.70                 & 34.32                 & 65.68                 \\ \hline
		$s_r$            & 31.72                 & 68.28                 & 34.11                 & 65.89                 \\ \hline
		$(s_r, q_r)$     & 32.32                 & 67.68                 & 34.39                 & 65.61                 \\ \hline
		$(q_d,s_r, q_r)$ & 37.77                 & 62.23                 & 37.72                 & 62.28                 \\ \hline
	\end{tabular}
\caption{Types et taux d'erreurs (pourcentage en moyenne sur les 6 catégories de demandes)} \label{tab:quanta:types_erreurs}
\end{table}

Trois raisons peuvent expliquer le fait que  peu de données attendues soient extraites. Premièrement, certaines valeurs d'attributs ne sont pas mentionnées dans les sections Litige et Dispositif utilisées (pourcentages inférieurs à 100 dans les Tableaux \ref{tab:quanta:mentionQd} et \ref{tab:quanta:mentionQr}). Par exemple, les quanta résultat de \textit{doris} plus présents dans les Motifs que dans le Dispositif.

\begin{table}[!htb]
	\footnotesize  \centering
	\begin{tabular}{|l|l|l|l|l|l|l|}
		\hline
		%\textbf{Catégorie}
		& \textbf{\#$q_d$} & \textbf{\#$q_d\neq NUL$} & \textbf{\# dans doc.} & \textbf{\# dans Litige} & \textbf{\# dans Motifs} & \textbf{\# dans Dispositif} \\ \hline
		\textit{acpa}               & 23                   & 16                                           & 16 (100\%)                   & 16 (100\%)              & 9 (56.25\%)             & 5 (31.25\%)                 \\ \hline
		\textit{concdel}            & 58                   & 56                                           & 55 (98.21\%)                 & 55 (98.21\%)            & 7 (12.5\%)              & 2 (3.57\%)                  \\ \hline
		\textit{danais}             & 208                  & 182                                          & 182 (100\%)                  & 179(100\%)              & 39 (21.43\%)            & 23 (12.64\%)                \\ \hline
		\textit{dcppc}              & 126                  & 126                                          & 122 (96.83\%)                & 109 (86.51\%)           & 71 (56.35\%)            & 65 (51.59\%)                \\ \hline
		\textit{doris}              & 94                   & 83                                           & 83 (100\%)                   & 82 (98.80\%)            & 21 (25.30)\%            & 6 (7.23\%)                  \\ \hline
		\textit{styx}               & 89                   & 86                                           & 86 (100\%)                   & 86 (100\%)              & 12 (13.95\%)            & 9 (10.47\%)                 \\ \hline
	\end{tabular}

	\textit{Les pourcentages ne sont calculés que pour les valeurs non nulles}
	\caption{Taux de quanta demandés ($q_d$) mentionnés dans les documents annotés} \label{tab:quanta:mentionQd}
\end{table}

\begin{table}[!htb]
	\footnotesize \centering
	\begin{tabular}{|l|l|l|l|l|l|l|}
		\hline
		%\textbf{Catégorie}
		& \textbf{\# $q_r$} & \textbf{\# $q_r\neq NUL$} & \textbf{\# dans doc.} & \textbf{\# dans Litige} & \textbf{\# dans Motifs} & \textbf{\# dans Dispositif} \\ \hline
		\textit{acpa}               & 23                & 6                       & 6 (100\%)             & 3 (50\%)                & 6 (100\%)               & 5 (83.33\%)                 \\ \hline
		\textit{concdel}            & 58                & 8                       & 8 (100\%)             & 2 (25\%)                & 8 (100\%)               & 6 (75\%)                    \\ \hline
		\textit{danais}             & 208               & 23                      & 23 (100\%)            & 15 (65.22\%)            & 22 (95.65\%)            & 20 (86.96\%)                \\ \hline
		\textit{dcppc}              & 126               & 76                      & 75 (98.68\%)          & 55 (72.37\%)            & 56 (73.68\%)            & 64 (84.21\%)                \\ \hline
		\textit{doris}              & 94                & 44                      & 44 (100\%)            & 28 (63.64\%)            & 40 (90.91)\%            & 24 (54.55\%)                \\ \hline
		\textit{styx}               & 89                & 30                      & 29 (96.67\%)          & 16 (53.33\%)            & 22 (73.33\%)            & 29 (96.67\%)                \\ \hline
	\end{tabular}

	\textit{Les pourcentages ne sont calculés que pour les valeurs non nulles}
	\caption{Taux de quanta accordés ($q_r$) mentionnés dans les documents annotés} \label{tab:quanta:mentionQr}
\end{table}

En second, la sélection des termes-clés n'est pas parfaite (Tableau \ref{tab:quanta:exemples_termes}). D'une part, l'ensemble sélectionné ne couvre  pas toutes les situations d'expression de la catégorie (par exemple, pour la catégorie \textit{styx}, le terme \og frais irrépétibles\fg{} est souvent utilisés à la place de \og article 700 du code de procédure civile\fg{}, mais dans très peu d'exemples annotés). D'autre part, certains termes sont trop spécifiques à la base d'apprentissage (par exemple, pour la catégorie \textit{concdel}, des sommes d'argent et autres termes comme \og condamner in solidum les sociétés \fg{} apparaissent dans la liste).

\begin{table}[!htb]
		\centering\footnotesize
	\begin{tabular}{|c|p{0.8\textwidth}|}
		\hline
		{Catégorie} & \multicolumn{1}{c|}{Termes-clés appris} \\ \hline
		\textit{acpa}    & amende civile                                                                                                                                                                                                                               \\ \hline
		\textit{concdel} & titre de la concurrence déloyale, somme de 15000euros à titre, réparation de son préjudice financier, payer la somme de 15000euros, condamner in solidum les sociétés, agissements constitutifs de concurrence déloyale                   \\ \hline
		\textit{danais}  & dommages et intérêts pour procédure, 32-1 du code de procédure, intérêts pour procédure abusive, titre de dommages-intérêts pour procédure, intérêts pour procédure, article 32-1 du code, dommages-intérêts pour procédure abusive         \\ \hline
		\textit{dcppc}   & admet la créance déclarée, admet la créance, passif de la procédure collective, passif de la procédure, hauteur de la somme, créance déclarée, titre chirographaire, admission de la créance, rejette la créance,                           \\ \hline
		\textit{doris}   & préjudices, abusive, condamner solidairement, solidairement, réparation du préjudice, réparation, titre de dommages et intérêts, dommages, titre de dommages, dommages et intérêts, titre de dommages-intérêts, payer aux époux, jouissance \\ \hline
		\textit{styx}    & 700 du code de procédure, article 700 du code, 700 du code, article 700, 700                                                                                                                                                                \\ \hline
	\end{tabular}

Les termes candidats sont des $n$-grammes de taille variant d'1 à 5 mots consécutifs
\caption{Premiers termes sélectionnés lors de la première itération de la validation croisée} \label{tab:quanta:exemples_termes}
\end{table}

Enfin, les expérimentations ont été réalisées sur des décisions d'appel mais les énoncés, de demande et résultat renvoyant aux décisions de jugements antérieurs, ne sont pas encore traités. Ces références aux décisions antérieures représentent une part importante des demandes des décisions d'appel. Il est donc nécessaire de les intégrer explicitement dans le processus d'extraction, pour compléter les données extraites.

\section{Conclusion}
\label{sec:quanta:conclusion}
Ce chapitre décrit le problème d'extraction de données pertinentes relatives aux paires demande-résultat mentionnées dans les décisions de justice. Les divers défis relatifs à la tâche  y sont discutés en remarquant des analogies avec d'autres tâches classiques de la fouille de données textuelles. Il a été démontré la solvabilité du problème par la proposition et l'expérimentation d'une approche d'extraction basée sur la terminologie de la  catégorie des demandes à extraire et autres connaissances du domaine judiciaire telles que les motifs d'énoncés de demandes et de résultat, ainsi que leur position conventionnelle dans les documents.  Les expérimentations démontrent que l'approche permet d'extraire plus ou moins bien des demandes selon la catégorie traitée. Même si nos résultats ont été obtenus à partir de terminologies apprises, une liste de termes fournis par les experts pourrait être plus précise et mettrait à l'abris des biais liés aux échantillons d'apprentissage. A cause de la  forte dépendance aux subtilités de rédaction des décisions judiciaires, la méthode rencontre des limites qui ne peuvent être surmontées qu'en la rendant beaucoup plus complexe qu'elle ne l'est déjà. Des approches d'apprentissage automatique sont recommandées comme perspectives. Elles devront être capables d'apprendre l'emplacement des données à extraire de manière semi-supervisée à l'aide de faibles quantités de longs documents annotés.
 % INCLUDE: quanta
\chapter{Identification du Sens du Résultat par Classification}
\label{sec:sensresultat}

\section{Objectif et Motivation}
\label{sec:sensresultat:motivation}
Comme le précédent, ce chapitre est relatif à l'extraction de données sur les demandes, mais  il est question ici d'extraire uniquement le sens du résultat d'une demande connaissant sa catégorie. Cette étude est intéressante parce que le problème devient plus simple. En se passant de la localisation précise des quanta ou plus globalement des passages d'expression des demandes et résultats, l'extraction du sens du résultat peut être formulée comme une tâche de classification à partir d'une représentation vectorielle des document. Nous modélisons la tâche comme une classification binaire où un algorithme est entraîné à déterminer si la demande a été rejetée (sens = rejette) ou acceptée (sens = accepte). Nous avons proposé cette modélisation sur une restriction du problème définie par les postulats \ref{postulat:sens:unedemande} et \ref{postulat:sens:sensbinaire}.

\begin{postulat}\label{postulat:sens:unedemande}
Pour toute catégorie de demande $C$, la majorité des documents ne contiennent qu'une demande de catégorie $C$.
%Pour toute catégorie de demande $C$, on ne considère que les décisions dans lesquelles n'apparaît qu'une seule demande de catégorie $C$. 
\end{postulat} 
Ce postulat est légitime car les statistiques sur les données labellisées de la Figure \ref{fig:quanta:hist-repartition-docs} montre bien que pour la majorité des catégories, peu de décisions contiennent plus d'une demande. On remarque néanmoins l'exception de la catégorie STYX (dommage-intérêt sur l'article 700 CPC), où dans la majorité des documents, on a plutôt 2 demandes. Cette exception peut se justifier par le fait que chaque partie fait généralement ce type de demande car elle porte sur le remboursement de frais de justice. Ce postulat présente cependant un inconvénient dû au fait que la majorité des demandes se trouvent dans des décisions à plus d'une demande. Il est donc possible de manquer un grand nombre de demandes. %On pourrait peut-être porter la classification à un modèle multi-label qui déterminera plusieurs sens à partir d'un seul document. Par exemple <SENS1, SENS2, SENS3> avec des valeurs prédéfinies sur les SENS 2 et 3 par exemple NO-DMD pour indiquer que la décision ne comprend pas de seconde ou de troisième demande.

\begin{postulat}\label{postulat:sens:sensbinaire}
Le sens du résultat est généralement binaire: accepte ou rejette.
\end{postulat} 
Ce postulat est justifié car le sens d'un résultat est pratiquement toujours une de ces deux valeurs (Figure \ref{stat-sensrst}). Les autres sens étant très rares, nous ne les considérons pas.

\begin{figure}
\includegraphics[width=\textwidth]{chartDistrSens.png}
\caption{Répartition des sens de résultat dans les données annotées.}\label{stat-sensrst}
\end{figure}
\section{Formulation du Problème}
\label{sec:sensresultat:probleme}

Cette étude porte sur l'analyse de l'impact de différents aspects techniques généralement appliqués dans le cadre de la classification de texte:
La représentation de texte (généralement vectorielle) faisant intervenir les notions de BagOfWord, TFIDF, poids global, poids local, matrice document-terme
La transformation de dimension a pour but de projeter le vecteur représentant un document dans un nouvel espace où généralement la distinction des classes est plus facile. Ceci est généralement effectuée en déterminant des composantes discriminantes à partir de la base d'entraînement: sélection de caractéristique, réduction matricielle, représentation vectorielle “sémantique” des documents (LatentDA, LSA, word embedding, ...)

 Cette analyse permettra de savoir s'il existe une certaine configuration permettant de déterminer le sens du résultat à une demande sans nécessairement l'avoir identifiée précisément dans le document. 


\section{Synthèse bibliographique: Classification de Texte}
\label{sec:sensresultat:biblio_classif}

La classification de texte permet d'organiser des documents dans des groupes prédéfinis. Elle reçoit depuis longtemps beaucoup d'attentions. Deux choix techniques influencent principalement les performances: la représentation des textes et l'algorithme de classification. Parmi les meilleurs algorithmes de classification binaire, les méthodes les plus populaires sont le NBSVM et FastText dont les performances pour l'analyse de sentiment sont très bonnes. Le principe du NBSVM \citep{wang2012nbsvm} consiste à transformer les caractéristiques des textes, réduites à leur simple présence en réalisant leur produit élément à élément avec le vecteur poids du classifieurs bayésien multinomial (calculé avec le vecteur présence de caractéristique). Le nouveau vecteur issu de ce produit représente le texte en entrée d'un SVM classique.
  
  Quant à FastText \citep{grave2017fasttextcls}, il s'agit d'un modèle de réseau de neurones dont l'architecture est semblable à celle de la variante CBOW de la méthode de plongement sémantique Word2Vec dans laquelle le mot du milieu a été remplacé par le label de la classe du texte. La classification est opérée par la fonction softmax $f(z) = \left[ \frac{e^{z_j}}{\sum\limits_{k=1}^K e^{z_k}} \right]_{\forall j \in \lbrace 1, ..., K \rbrace} $ et l'entraînement consiste à minimiser la fonction objectif $-\frac{1}{N}y_n \cdot \sum\limits_{n=1}^N y_n \cdot \log{f(B\cdot A\cdot x_n)}$ qui estime la distribution de probabilité des classes.

Le fonctionnement de ces deux méthodes intègrent leur propre représentation, contrairement aux algorithmes traitant des vecteurs comme le SVM. Il existe un très grand nombre de schémas de représentations vectorielles des documents


\section{Méthode: Extensions de la Regression PLS}
\label{sec:sensresultat:pls}
\textcolor{red}{Justification: Pourquoi le PLS?:}
https://link.springer.com/content/pdf/10.1007\%2FBF02174528.pdf, 
https://www.stat4decision.com/fr/regression-pls/
La regression PLS est une méthode de regression avec laquelle l'on tente d'expliquer une ou plusieurs variables Y (dite dépendantes) par des variables $X=x_1,x_2,...,x_p$ (dites explicatives). Elle consiste principalement à transformer les variables explicatives en un nombre réduit de composantes principales orthogonales $t_1, t_2, ..., t_h$. Les composantes $t_h$ sont construites étapes par étapes en applicant l'algorithme du PLS de façon récurrente sur les données mal prédites (résidus). Malgré quelques faiblesses comme celles liées au choix du nombre de composantes, à la complexité des sorties et la linéarité du modèle, la regression PLS présente quelques atouts qui ont notamment de l'intérêt dans notre cas de figure. Par exemple, le PLS gère assez bien la forte dispropotion entre le nombre de variables explicatives et le nombre d'observations, lorsque ce dernier est faible comme on peut l'observer dans nos données (faible quantité de données d'apprentissage). Nous avons aussi la prise en compte de la multicolinéarité qui peut exister entre les variables explicatives, notamment quand celles-ci sont associées aux mots/termes souvent cooccurrents de nos documents.

Il est intéressant de noter la floraison d'extensions proposées pour répondre aux différentes limites du PLS. Notamment, nous pouvons citer la "\textit{sparse}" PLS introduite pour palier à la "\textit{sparsité}" et la colinéarité des variables explicatives [?], la PLS non-linéaire proposée pour les cas de données non-linéairement séparables [?], ou encore la PLS discriminante combinant la régression PLS et l'analyse discriminante [?]. Nos nous sommes intéressés à deux extensions particulières: la régression Gini-PLS \citep{souissi2013ginipls} dont l'intérêt est de réduire la sensibilité aux valeurs aberrantes des variables, et la regression Logit-PLS \citep{tenenhaus2005logitpls}  combinant la regression logistique et la PLS.
\subsection{Gini-PLS}
\subsection{Logit-PLS}
\subsection{Logit-Gini-PLS}

\section{Expérimentations et interprétation des résultats}
\label{sec:sensresultat:experimentations}

\subsection{Données}
\label{subsec:sensresultat:xp:data}

\begin{figure}[!h]
\includegraphics[width=\textwidth]{chartDataset1dmd.png}
\caption{Répartition des documents à une demande de la catégorie considérée.}\label{stat-1dmd}
\end{figure}

\subsection{Robustesse au déséquilibre des classes}

\subsection{Robustesse à la faible quantité des données d'apprentissage}

\section{Conclusion}
\label{sec:sensresultat:conclusion}% INCLUDE: sensresultat
 \chapter{Modélisation des Circonstances Factuelles}
\label{chap:similarite}

%\epigraph{Le plus important, c'est d'avoir un langage métaphorique ; car c'est le seul mérite qu'on ne puisse emprunter à un autre et qui dénote un esprit naturellement bien doué ; vu que, bien placer une métaphore, c'est avoir égard aux rapports de ressemblance.}{Aristote, Poétique}
% 

\section{Introduction}
\label{sec:similarite:introduction}
Les circonstances factuelles définissent les contextes possibles dans lesquels une catégorie de demande peut être formulée (\textcolor{red}{N'est-ce pas simplement l'accusation ou l'objet du litige}). Les analyses descriptives ou prédictives ne prennent sens que lorsqu'elles sont appliquées à un ensemble de décisions aux circonstances similaires. Par exemple, il serait imprudent de considérer toutes les décisions pour analyser les chances d'acceptation d'une demande de dommages-intérêts fondée sur l'\og article 700 du code de procédure civile \fg{} en cas de trouble de voisinage. Les taux d'acceptation ou de rejet peuvent être différents entre des affaires de licenciement et celles portant sur les troubles anormaux du voisinage, et même plus spécifiquement entre les troubles de voisinage entre particulier et entreprises. % (par exemple: chantier de construction), ou simplement entre particuliers (par exemple: troubles sonores). 
 Il serait préférable de travailler uniquement avec des décisions similaires à la situation d'intérêt. L'identification des circonstances factuelles devient donc une étape préalable indispensable à l'analyse du résultat. Malheureusement, les circonstances sont très diverses et quasi infinies pour être identifier par classification supervisée à l'aide d'annotation manuelle d'exemples comme dans les chapitres précédent. Il est donc plus indiquer d'adopter une approche non-supervisée capable modéliser les circonstances factuelles à partir d'un corpus de documents d'une même catégorie de demande. Plus précisément, la méthode doit construire des sous-ensembles de décisions selon qu'elles traitent de contextes similaires.  Les objectifs de ce chapitre sont d'expérimenter des algorithmes  de \textit{clustrering} et des métriques de similarité. Il démontre aussi qu'une distance entraînée  sur des documents d'une catégorie de demande permet de mieux mesurer la (dis-) similarité sémantique définie par les circonstances factuelles.

%\section{Formulation du Problème}
%\label{sec:similarite:probleme}

\section{Regroupement non-supervisé de documents}
\label{sec:similarite:biblio}

Cette section fait une synthèse bibliographique des différents aspects rentrant dans la conception d'un modèle de \textit{clustering}. Elle aborde principalement le choix de l'algorithme, la définition d'une mésure de similarité adéquate, la représentation des documents, la détermination du nombre de \textit{clusters}, l'affectation de labels aux \textit{clusters}, et l'évaluation du regroupement généré.

\subsection{Choix de l’algorithme de clustering}

Le clustering de documents est une tâche dont l'objectif est d'identifier, sans supervision\footnote{Sans exemples annotés.}, une structure pertinente (pour le domaine expert) dans un ensemble $\mathcal{D} = \lbrace d_1, \dots, d_N \rbrace$ de $N$ documents non annotés en construisant des groupes représentants des catégories inconnues au départ. Ces groupes, appelés clusters, peuvent être disjoints ou à chevauchements, et plates ou hiérarchiques suivant les contraintes du domaine expert. L’algorithme à utiliser dépend généralement de la forme qu’on souhaite donner à l’organisation. 
\subsubsection{Partitionnement disjoint}
Pour réaliser des partitions distinctes\footnote{Chaque document n'appartient qu'à un seul cluster.} (\textit{hard clustering}), des algorithmes tels que celui des K-moyennes \citep{forgey1965kmeans} et celui des \textit{K-medoïdes} \citep{kaufman1987kmedoids} seront préférés \citep{balabantaray2015kmeanskmedoids}. Ces deux algorithmes fonctionnent de manière similaire, et nécessitent que le nombre $K$ de clusters soient prédéfini. Ils commencent par une définition aléatoire de $K$ centres initiaux de clusters (centroïdes) et l'affectation des différents documents au cluster dont le centre est le plus proche. S'en suit une boucle dans laquelle le centroïde est recalculé (le point à distance totale minimale avec les membres du cluster) et les documents sont réaffectés chacun au cluster dont le centroïde est le plus proche. L'algorithme s'arrête si aucune amélioration n'est plus observée, ce qui se traduit soit par l'atteinte d'une valeur minimale prédéfinie de l'erreur de \textit{clustering}\footnote{Somme des distances au carré entre les points et leur centre respectif.} ou d'une mesure d'évaluation non supervisée (\ref{sec:similarite:biblio:unsupeval}). La différence entre l'algorithmes des K-moyennes et celui des \textit{K-medoïdes} tient principalement au fait que les centroïdes du premier ne sont pas nécessairement des points (documents) de l'ensemble d'origine, mais des points moyennes des représentations vectorielles des membres du cluster, contrairement à l'algorithme des \textit{K-medoïdes} qui ne considère que les documents originaux qui ont une distance minimale à tous les documents dans leur cluster. Cette différence donne l'avantage au \textit{K-medoïdes} de ne pas dépendre d'une représentation vectorielle nécessaire au calcul de la moyenne, mais elle a aussi l'inconvénient d'augmenter sa complexité en temps  car il faut calculer et stocker la distance entre toutes les paires de documents. Il existe plusieurs autres algorithmes de clustering disjoint dont le principe de fonctionnement est différent de celui des K-moyennes. Par exemple, l'algorithme DBSCAN (\textit{Density-based spatial clustering of applications with noise}) \citep{ester1996dbscan}  ne prend pas en paramètre le nombre de clusters à construire. Il est défini sur le concept de régions de densité caractérisées par la distance minimale $\epsilon$ autorisée entre deux points d'une même région, et le nombre maximal de points qui doivent être dans le voisinage de rayon $\epsilon$ d'un point pour que ce voisinage soit une région de densité (le point central est appelé "point noyau" (\textit{core point}). Le principe du DBSCAN est de construire les clusters successivement en reliant les régions (voisinages) dont les noyaux sont à distance plus ou moins inférieure à $\epsilon$. Les points qui sont seul dans leur cluster sont qualifiés d'\textit{outliers}. 
% amélioration par réduction de dimension
En outre, le clustering spectral est une autre méthode efficace de regroupement qui effectue préalablement une réduction de dimensions à l'aide du spectre de la matrice de similarité $M \in \mathbb{R}^{N \times N}$ \footnote{$M_{ij}$ est la mesure de la similarité entre les points (documents) $d_i$ et $d_j$ du corpus $D$.} des données  avant d'appliquer un algorithme traditionnel comme celui des K-moyennes. Les dimensions du nouvel espace sont définies par les vecteurs propres de la matrice Laplacienne $L$ de $M$ \citep{shi2000spectralClustering, von2007tutorialSpectralClustering} qui peut être normalisée ($L = T^{-1/2}(T-S)T{-1/2}$) ou pas ($L = T - M$), $T$ étant la matrice diagonale déduite de $M$ i.e. $T_{ii} = \sum\limits_j M_{ij}$. 

Il est aussi possible d'utiliser les arbres de décision pour améliorer les résultats des K-moyennes. En effet, les forêts aléatoires \citep{breiman2001randomforest} permettent d'estimer la similarité entre deux points. Le principe consiste à générer un ensemble de $n$ points synthétiques, et d'entraîner une forêt aléatoire à une classification binaire supervisée avec les points originaux considérés dans la classe des "originaux" et les données synthétiques dans la seconde classe des "synthétiques" \citep{afanador2016unsupervisedrandomforest}. Une forêt aléatoire étant un ensemble d'arbres de décision (classification) construit sur des parties de l'ensemble d'apprentissage duquel on a retiré une ou plusieurs variables prédictives, la similarité entre 2 points est la proportion d'arbres dans lesquels ces points se trouvent dans le même nœud feuille. Cette métrique "apprise" peut-être par la suite utilisée dans un algorithme de clustering classique comme les K-moyennes.

%\textcolor{red}{Random Forest - processus de construction: \url{https://onlinelibrary.wiley.com/doi/pdf/10.1002/cem.2790}}

%L'application de ces différents algorithmes aux documents n'est généralement basé que sur les statistiques d'occurrence des termes, et par conséquent les thématiques abordées dans les documents ne sont pas bien prise en compte, surtout que l'élimination des \og mots vides \fg{} (\textit{stop words}) peut laisser les deux documents sans sinon très peu de mots en commun \cite{kusner2015wordmoverdist}. \citet{xie2013MGCTM} démontrent empiriquement que l'intégration de la  modélisation thématique (\textit{topic modeling}) au clustering de documents améliore significativement les résultats. Cette intégration des modèles thématiques dans le clustering de documents peut être réalisée de multiples façons, mais deux méthodes semblent être les plus efficaces:
%\begin{enumerate}
%	\item l'intégration naïve \citep{lu2011kmeansLDApLSA} qui consiste à inférer $K$ thèmes à l'aide d'un algorithme comme le PLSA (aAnalyse sémantique probabiliste latente) \citep{hofmann1999PLSA} ou le LDA (allocation de Dirichlet latente)\citep{blei2003lda}, puis de considérer pour chaque document le thème $j \in [1..K]$ qui a la probabilité $\theta_j$ la plus élevé dans ce document suivant la distribution $\theta$ de probabilité des thèmes dans ce document; le thème choisi $j$ représente le cluster du document;
%	\item le modèle thématique multi-grain de clustering (\textit{multi-grain clustering topic model}) ou MGCTM proposé par \citet{xie2013MGCTM}, dont l'objectif est d'inférer de manière jointe les clusters et le modèle thématique.
%\end{enumerate}

\subsubsection{Regroupement avec chevauchements}

Lorsque des chevauchements sont observables entre clusters (un document peut faire partie de plusieurs groupes à la fois), chaque objet peut être affecté partiellement à chaque cluster grâce à la notion de degré d'appartenance (\textit{membership degree}) entre un point $x_i \in X$ et le cluster $j \in [1..K]$ estimé par une fonction $u_{ij}$  \citep{baraldi1999surveyfuzzyclstering}. Il est par conséquent préférable d'employer des algorithmes de partitionnement "mou" comme l'algorithme flou des c-moyennes (FCM) \citep{bezdek1984fcm, hathaway1989fuzzycmeans}, ou le fuzzy c-Medoids (FDMdd) \citep{krishnapuram2001fuzzycmedoids}, ou la version améliorée IFKM (\textit{improved fuzzy K-medoids})\citep{sabzi2011fuzzykmedoids}. \textcolor{red}{FONCTIONNEMENT DE CES DEUX ALGO}. Le principe des algorithmes de clustering flou consiste en deux étapes principales \citep{sabzi2011fuzzykmedoids}: 

Lorsque des chevauchements sont observables entre clusters (un document peut faire partie de plusieurs groupes à la fois), chaque objet peut être affecté partiellement à chaque cluster grâce à la notion de degré d'appartenance (\textit{membership degree}) entre en jeu \citep{baraldi1999surveyfuzzyclstering}. Il est par conséquent préférable d'employer des algorithmes de partitionnement "mou" comme l'algorithme flou des c-moyennes (\textit{fuzzy c-means}) \citep{bezdek1984fcm, hathaway1989fuzzycmeans}, ou le fuzzy c-medoid \citep{krishnapuram2001fuzzycmedoids}. \textcolor{red}{FONCTIONNEMENT DE CES DEUX ALGO}. Le principe des algorithmes de clustering flou consiste en deux étapes principales \citep{sabzi2011fuzzykmedoids}: 

\begin{enumerate}
 \item l'estimation des degrés d'appartenance de chaque instance $x_i \in X$ à chaque cluster $j \in [1..K]$ de centroïde $z_j$ réalisée par la minimisation de la fonction objective $P(X,Z) = \sum\limits_{i=1}^{n}\sum\limits_{j=1}^{k} \left[u_{ij}r(x_i,z_j)\right]$ \citep{krishnapuram2001fuzzycmedoids}  améliorée par \citet{sabzi2011fuzzykmedoids} en:
 \[P(X,Z) = \sum\limits_{i=1}^{n}\sum\limits_{j=1}^{K} \left[u_{ij}r(x_i,z_j)\right] + \lambda \sum\limits_{i=1}^{n}\sum\limits_{j=1}^{K} \left[ u_{ij}\log_2(u_{ij}) \right] \]
 \[s.c. \sum\limits_{j=1}^{k} u_{ij} = 1\]
 \[0 \leq u_{ij} < 1\]
 
 dont la valeur approximative généralement utilisée de la solution est \[u_{ij} = \frac{\exp\left(\frac{-r(x_i,z_j)}{\lambda}\right)}{\sum_{l=1}^{k}\exp\left(\frac{-r(x_l,z_j)}{\lambda}\right)},\] $r(x_i,z_j)$ étant la mesure de dis-similarité (distance) entre $x_i$ et  $z_j$;
 \item la détermination des nouveaux centres de clusters qui s'effectue toujours par la moyenne des membres du cluster chez le fuzzy c-means, mais par le choix de l'objet $x_q$ qui optimise la somme des distances de cet objet aux autres membres pondérée chacune par le degré d'appartenance de ces autres membres: \[\forall j \in \left[ \left[1;k \right] \right], q = \argmin\limits_{1 \leq l < s_j} \sum\limits_{l=1}^{s_j} \left[u_{lj}r(x_l,z_j)\right]\] $s_j$ étant le nombre de membres du cluster $j$.
\end{enumerate}
 Ainsi l'objectif de l'entrainement des algorithmes de clustering flou est double: déterminer les valeurs optimales du vecteur  $U$ des degrés d'appartenance et de l'ensemble $Z$ des centroïdes. 
 
 Les regroupements avec chevauchement sont intéressants parce qu'il n'est pas exclu qu'une décision traite de plus d'une circonstance factuelle.
%\textcolor{red}{A COMPLETER!!!!!!!!!!}
%Pour des regroupements hiérarchiques, des algorithmes comme celui du clustering par agglomération (\textit{agglomeration clustering}) sont mieux indiqués. Le principe du clustering par agglomération est de ...
%si les chevauchements sont négligeables ou n'existent pas, ou bien si la structure hiérarchique permettrait de mieux expliquer et distinguer les différences inter-groupes et les ressemblances intra-groupes. Nous souhaitons organiser des décisions de justices en fonction des circonstances factuelles auxquelles ces documents sont liés.  On pourrait par exemple faire une restriction des données aux cas où chaque document n’appartient qu’à une classe et proposer un système de clustering disjoint.

%\subsubsection{Limites des algorithmes de clustering}
%nombre prédéfini de clusters, initialisation aléatoire des centroïdes menant à des clusters différents entre plusieurs exécution \citep{sabzi2011fuzzykmedoids}. Nous noterons aussi la dépendance à la métrique de similarité.
%Algorithme kmeans + kmédoids pour les documents: https://pdfs.semanticscholar.org/a46f/efdb64a01d1e6390c8212d881b9c4414ffbf.pdf


\subsection{Métriques de dis-similarité (distances)}
\label{sec:similarite:distances}
Les algorithmes de clustering dépendent de la distance utilisée qui doit être bien choisie pour que les regroupements révèlent au mieux la sémantique visée. Une distance ou métrique de dis-similarité est une fonction réelle d'une paire de éléments $x$ et $x'$ d'un ensemble $\mathcal{X}$. Une métrique de dis-similarité mesure le degré de différence entre $x$ et $x'$  généralement estimée par une fonction distance $Dis$  qui satisfait aux propriétés suivantes $\forall x,x',x'' \in \mathcal{X}$ \citep{wang2015distancemetriclearningsurvey}:
\begin{enumerate}
\item $Dis(x,x') \geq 0$ ("non-négativité")
\item $Dis(x,x') = 0  \Leftrightarrow x = x'$ (identité discernable)
\item $Dis(x,x') = Dis(x', x)$ (symétrie)
\item $Dis(x,x'') \leq Dis(x,x') + Dis(x',x'')$ (inégalité triangulaire) \label{enum:sim:ineq-tri}
\end{enumerate}


La métrique peut être normalisée ($\forall a,b \in \mathcal{X};  0 \leq Dis(a,b) \leq 1$), à l'instar de la distance basée sur la similarité cosinus normalisée et celle de Jaccard. Dans ce cas, la relation entre la similarité $Sim$ et la dis-similarité $Dis$ est définie par $Sim(a,b) = 1 - Dis(a,b)$.
%On parle de \textbf{pseudo-métrique} lorsque la condition \ref{enum:sim:ineq-tri} n'est pas satisfaite.

Il existe de nombreuses métriques de similarité généralement expérimentées pour le clustering de textes \citep{huang2008similarityTextClustering, vijaymeena2016surveySim, afzali2018SimKmeans}:
\begin{itemize}
	\item Les distances de Minkowski de forme générale $Dis(x,x') = \norm{x - x'}_{Lp} = \sqrt[p]{\sum \vert x_i - x'_i \vert ^p}$, dont font partie la distance euclidienne ($p=2$) et la distance de Manhattan ($p=1$).
	\item La distance de Braycurtis calcule la distance entre deux documents $x$ et $x'$ par la formule: $Dis_{Braycurtis}(x,x') = \frac{\sum \vert x_i - x'_i \vert}{\sum \vert x_i + x'_i \vert}$.
	\item La similarité cosinus est basée sur la mesure de l'angle entre les vecteurs représentant les documents $x$ et $x'$, et est calculée par: $Sim_{cos}(X,Y) = \frac{X^tY}{\norm{X}\norm{Y}}$.
	Cette formulation considérant que tous les termes du vocabulaire $W$, définissant le modèle de représentation vectoriel, sont différents et ne partagent aucune relation, \citet{sidorov2014softcosine} la corrigent en proposant la fonction \textit{soft-cosine} en introduisant une matrice $S = {S_{ij}}_{1\leq i,j \leq \vert W \vert}$ de similarité entre  termes: 
	
	$Sim_{soft-cos}(X,Y)= \frac{X^t\cdot S\cdot Y}{\sqrt{X^t\cdot S\cdot X}\cdot \sqrt{Y^t\cdot S\cdot Y}} = \frac{\sum\limits_{i,j}^{\vert W \vert}s_{ij}x_iy_j}{\sqrt{\sum\limits_{i,j}^{\vert W \vert}s_{ij}x_ix_j}\sqrt{\sum\limits_{i,j}^{\vert W \vert}s_{ij}y_iy_j}}$,
	
	où $S$, la matrice de similarité entre les termes, peut être calculée à partir de n'importe quelle métrique comme la distance d'édition de Levenshtein (similarité lexicale) \citep{sidorov2014softcosine},  la similarité cosinus entre  plongements lexicaux \citep{charlet2017simbow_acl, charlet2017simbow_tal}, ou la similarité WordNet.
	
	La similarité cosinus  étant comprise entre -1 et +1, sa forme normalisée s'écrit  $\overline{Sim}_{cos}(x,x') = \frac{Sim_{cos}(x,x') + 1}{2}$, d'où la distance $Dis_{cos}(x,x') = 1 - \overline{Sim}_{cos}(x,x')$.
	
	\item Le coefficient similarité de Jaccard: $Sim_{Jaccard}(x,x') = \frac{x^Tx'}{\norm{x}^2+\norm{x'}^2 - x^Tx'}$, où $\norm{x}$ désigne la taille de l'ensemble des termes de $x$ dans le cas où $x$ est un document;
	%\item Le coefficient similarité de Dice: $Sim_{Dice}(x,x') = \frac{2\cdot \vert tok(x) \cap tok(x') \vert}{\vert tok(x) \vert + \vert tok(x')} $
	\item La similarité basée sur le coefficient de corrélation de Pearson: avec $w(t_i,x)$ le poids du terme $t_i$ dans le texte $x$\footnote{Comme à la section \ref{sec:quanta:classification}} et $TF_x = \sum\limits^m_{i=1} w(t_i,x)$, la similarité est calculée par:
	
	\[Sim_{Pearson}(x,x') = \frac{m \sum\limits^m_{i=1} w(t_i,x) \cdot w(t_i,x') - TF_x\cdot TF_{x'}}{\sqrt{[m \sum\limits^m_{i=1} w^2(t_i,x) - TF^2_x][m \sum\limits^m_{i=1} w^2(t_i,x') - TF^2_{x'}}}\]
et la distance par:

$Dis_{Pearson}(X,Y) =
\left\{ \begin{array}{ll}
1 - Sim_{Pearson}(X,Y) & \text{si } Sim_{Pearson}(X,Y) \geq 0 \\
\vert Sim_{Pearson}(X,Y) \vert & \text{si } Sim_{Pearson}(X,Y) < 0
\end{array}
\right.$
%	\item Distance de la divergence moyenne de Kullback-Leibler considère un document comme une distribution de probabilité de termes, et mesure donc la similarité entre deux distributions: 
%	\[Dis_{avgKL}(x,x') = \sum\limits_{i=1}^m\big(\pi_1 \cdot D(w(t_i,x) \vert\vert w_t) + \pi_2 \cdot D(w(t_i,x') \vert\vert w_t) \big)\]
%	avec $\pi_1 = \frac{w(t_i,x)}{w(t_i,x) + w(t_i,x')}$, $\pi_2 = \frac{w(t_i,x')}{w(t_i,x) + w(t_i,x')}$, $D(a \vert\vert b) = a\cdot  \log_2(\frac{a}{b})$, et $w_t = \pi_1 \cdot w(t_i,x) + \pi_2 \cdot w(t_i,x')$
%	\item Okapi BM25 est une métrique de similarité généralement utilisée en recherche d'information pour calculer un score de pertinence d'un document D par rapport à une requête Q: 
%	\[Sim_{BM25}(Q,D) = \sum\limits_{w \in Q \cap D} \left( \frac{(k_3+1) \cdot c(w, Q)}{k_3 + c(w, Q)} \cdot f(w,D) \cdot \log \frac{N+1}{df(w) + 0.5)}\right),\]
%	\[\text{avec } f(w,D) = \frac{(k_1+1)\cdot c(w,D)}{k_1(1-b+b\frac{\vert D \vert}{avgdl})} = \frac{(k_1+1)\cdot c'(w,D)}{k_1 + c'(w,D)},\] où $c'(w,D) = \frac{c(w,D)}{1-b+b\frac{\vert D \vert}{avgdl} }$. $c'(w,D)$ pouvant approcher 0 pour des documents très longs, \citet{Lv2011BM25L} propose BM25L, une formulation plus robuste à la longueur des documents obtenue en remplaçant $f(w,D)$ par :
%	\[
%	f'(w,D) =
%	\left\{ \begin{array}{ll}
%	\frac{(k_1+1)\cdot (c'(q,D)+\delta)}{k_1+ (c'(w,D) + \delta)} & \text{si } c'(w,D) > 0 \\
%	0 & \text{si } sinon
%	\end{array}
%	\right.
%	\]
	\item \og La distance du déménageur de mot \fg{} (\textit{word mover's distance - WMD}) \citep{kusner2015wordmoverdist} est une méthode dont l'objectif est similaire au notre, i.e. inclure la similarité sémantique entre les paires de mots de deux documents dans l'estimation de la distance entre ces derniers. En effet, elle est la solution optimale du problème de transport suivant \footnote{Valeur minimale du cout cumulatif pondéré nécessaire pour déplacer  tous les mots de $d$ à $d'$ i.e. transformer $d$ en $d'$.}:
	
	\begin{equation*}
	\begin{aligned}
Dis_{wmd}(d, d') = 	& \min\limits_{T>0}
	& & \sum\limits_{i,j=1}^n T_{ij} c(i,j) \\
	& \text{s.c.}
	& & \sum\limits_{j=1}^n T_{ij} = d_i, \forall i \in {1, \dots, n} \\
	& 
	& & \sum\limits_{i=1}^n T_{ij} = d'_j, \forall j \in {1, \dots, n}	
	\end{aligned}
	\end{equation*} 
	
	$n$ est le nombre de mots considérés; $T$ est une matrice dont $T_{ij}$ est interprété comme étant la quantité du mot $i$ de $d$ qui est va ("voyage") au mot $j$ dans $d'$; $c(i,j)$ est la distance euclidienne entre les vecteurs des mots $i$ et $j$; $d_i$ et $d'_j$ sont les composantes aux mots $i$ et $j$ resp. des vecteurs normalisés sac-de-mots de $d$ et $d'$ reesp. i.e. $d_i = \frac{compte(i, d)}{\sum\limits_{k=1}^n compte(k, d)}$, où $compte(i, d)$ est le nombre d'occurrences du mot $i$ dans $d$.
	
\end{itemize}


%Par contre, les métriques {apprises} sont définies à partir de connaissances des données labellisées. Ces métriques sont apprises pour répondre à la difficulté d'identifier la métrique statique appropriée pour un problème. L'apprentissage exploite un corpus préalablement annoté. L'apprentissage peut être supervisé si l'annotation du corpus consiste soit en classifiant des documents\footnote{Organisation des documents d'entraînement en des groupes aux labels prédéfinis.} \citep{weinberger2005LMNN}, soit en affectant des mesures de similarité à des paires de documents \citep{bibid}.  Un apprentissage semi-supervisé typique utilise des données annotées par jugements relatifs sur des pairs ou triplets de documents. Les contraintes de couples consistent en deux ensembles, l'un comprenant des couples de documents qui doivent être similaires, et l'autre contenant des couples de documents dis-similaires. Les contraintes de triplets consistent à définir pour un triplet de documents $(x_1,x_2,x_3)$ une comparaison de degré de similarité entre les paires (par exemple, $x_1$ est plus similaire à $x_2$ qu'à $x_3$). La métrique apprise est néanmoins une véritable métrique à valeur réelle positive écrite sous la forme d'une distance de Mahalanobis $f(x,y) = \sqrt{(x-y)^T M^{-1}(x-y)}$ (où $M$ est la matrice à apprendre). 
 
% L'apprentissage expérimenté dans ce chapitre est supervisé, même s'il utilise des données synthétiques. Nous supposons étant donné que les documents du corpus à \textit{clusteriser} sont tous de la même catégorie de demande, la différence entre les clusters et leur homogénéité se remarquera au niveau des faits. Par cette hypothèse, il reste un risque que d'autres types de regroupements se forment comme par exemple suivant d'autres catégories de demande co-occurrentes. Parmi les divers algorithmes réalisant un apprentissage supervisé, notons par exemple:
% \begin{itemize}
% 	\item Les plus-proches-voisins-dans-la-large-marge (LMNN) \citep{weinberger2005LMNN} plus adapté à l'annotation par classification;
% 	\item L'analyse des composants du voisinage (NCA) \citep{goldberger2005NCA};
% 	\item L'apprentissage de métrique pour la régression noyau (\textit{MLKR}) \citep{weinberger2007MLKR};
% 	\item L'analyse discriminante locale de Fisher (LFDA) \citep{sugiyama2007LFDA, } méthode supervisée (données labellisées) de réduction de dimension
% \end{itemize}

\subsection{Représentations des textes}
La formulation des distances exploite très souvent une représentation vectorielle des texte (cf.  \ref{sec:similarite:distances}). 


\subsection{Déterminer le nombre approprié de clusters (validation)}

%\textcolor{red}{faire un tableau des indices comme dans l'article, et comparer les combinaison indices-algo-distance}

 Au delà de l’algorithme à utiliser, le nombre $K$ approprié de clusters doit être déterminé mais pas prédéfini, puisqu'il est difficile de savoir à l'avance le nombre de groupes, le clustering permettant de proposer automatiquement un regroupement. Une méthode très connue est celle du \og coude \fg{}  (ou \og genou \fg{}) \citep{halkidi2001clustvalidation}, qui est basé sur le principe de base des algorithmes de partitionnement (e.g. K-moyennes) i.e. minimiser le critère d'inertie\footnote{la variance intra-cluster qui est la somme au carré des erreurs (distance d'un membre au centre).}:
\[J(K) = \sum\limits_{j=1}^K\sum\limits_{x_i \in C_j}\norm{x_i-\overline{x_j}}^2\]

$C_j$: ensemble des objets du cluster $j$

$\overline{x_j}$: échantillons moyens du cluster $j$

La méthode du coude consiste à essayer différentes valeurs consécutive de $K$ (de $K_{min}$ à $K_{max}$) puis de choisir celle qui correspond au coude de la courbe du critère d'inertie $J(K)$. Le choix de ce coude est visuel et peut être ambigu (plusieurs valeurs de $K$ sur le coude par exemple). 

La méthode de la silhouette moyenne \citep{rousseeuw1987silhouetteclusternumber} est une alternative qui consiste à choisir comme valeur optimale de $K$, celle qui maximise le critère de la largeur moyenne de la silhouette: $S(k) = \frac{1}{K}\sum\limits_{i=1}^N s(d_i)$. La largeur $s(d_i)$ de la silhouette est un indice qui compare la ressemblance d'un document $d_i$ aux autres membres de son cluster $C_t$ par rapport à ceux d'autres clusters $C_l, l \neq t$:
\[s(d_i) = \frac{b(d_i) - a(d_i)}{\max\lbrace a(d_i),b(d_i)\rbrace}\]

où $a(d_i) = \frac{1}{\vert C_t \vert} \sum\limits_{j=1}^{\vert C_t \vert} Dis(d_i, d_j)$, et $b(d_i) = \min\limits_{l \neq t} \frac{1}{\vert C_l \vert} \sum\limits_{j=1}^{\vert C_l \vert} Dis(d_i, d_j)$.

$K$ a une valeur optimale lorsque la largeur moyenne $S(k)$ atteint sa valeur maximale. Salvador et Chan (2004) propose d’utiliser l’intersection des deux lignes approximant la courbe. Mais plus récemment, Zhang et al. (2016) trouvent que cette approche n’est pas appropriée pour les cas où le graphe d’évaluation n’est ni lisse, ni monotone. Ils proposent d’exploiter la courbure du graphe i.e. la valeur dont un objet géométrique s'écarte d'être plat ou droit dans le cas d'une ligne.

%Etant donné le grand nombre de méthodes existantes \citep{liu2010interclustvalidation, Amorim2015recoveringnumclust}, la majorité des votes peut être appliquée pour choisir le bon $K$ \footnote{\url{https://www.datanovia.com/en/lessons/determining-the-optimal-number-of-clusters-3-must-know-methods/}}.

%\subsection{Initialisation des centroïdes}

%\subsection{Définir une représentation appropriée pour les textes}
%\url{https://arxiv.org/pdf/1509.01626.pdf}

%\url{http://ad-publications.informatik.uni-freiburg.de/theses/Bachelor_Jon_Ezeiza_2017_presentation.pdf}


%\subsection{Labeliser les clusters}

\subsection{Validation du regroupement}
L'évaluation d'un regroupement peut être supervisée ou non selon que l'on dispose ou pas respectivement d'exemples de données annotés avec les groupes attendus.

\subsubsection{Métriques supervisées ou mesures externes}
\label{sec:similarite:biblio:supeval}
Il s'agit de métriques comparant deux regroupements $X = \lbrace X_1, X_2, ..., X_r \rbrace$ et $Y = \lbrace Y_1, Y_2, ..., Y_s \rbrace$ pour mesurer leur ressemblance. Elles sont définies généralement à partir du tableau de contingence résumant les chevauchements que partagent $X$ et $Y$:
\begin{table}[!htb]
	\centering
	\begin{tabular}{|c|c|c|c|c|c|}
		\hline
		& $Y_1$    & $Y_2$    & $\cdots$ & $Y_s$    & $\sum$   \\ \hline
		$X_1$    & $n_{11}$ & $n_{11}$ & $\cdots$ & $n_{11}$ & $a_1$    \\ \hline
		$X_2$    & $n_{21}$ & $n_{21}$ & $\cdots$ & $n_{21}$ & $a_2$    \\ \hline
		$\cdots$ & $\cdots$ & $\cdots$ & $\ddots$ & $\cdots$ & $\cdots$ \\ \hline
		$X_r$    & $n_{r1}$ & $n_{r1}$ & $\cdots$ & $n_{r1}$ & $a_r$    \\ \hline
		$\sum$   & $b_1$    & $b_2$    & $\cdots$ & $b_s$    &          \\ \hline
	\end{tabular}
	\caption{Tableau de contingence des chevauchement entre les regroupements $X = \lbrace X_1, X_2, ..., X_r \rbrace$ et $Y = \lbrace Y_1, Y_2, ..., Y_s \rbrace$}
\end{table}

Même s'il existe un très grand nombre de mesures d'évaluation de la qualité du regroupement \citep{im2003clusteringsurvey}, trois métriques sont couramment employées pour l'évaluation supervisée:
\begin{itemize}
	\item l'information mutuelle normalisée (NMI) \citep{strehl2000nmi, vinh2010clusteringComparison} normalise l'information mutuelle entre les labels attendus $Y$ et ceux des clusters $C$ par une agrégation de leur entropie respective. Par exemple, l'incertitude symétrique \citep{kvalseth1987entropy_NMI} utilise la moyenne comme agrégateur:  $NMI(Y,C) = \frac{2 \cdot I(Y;C)}{H(Y) + H(C)} $
	Avec $I(Y;C) = H(Y) - H(Y \vert C)$ et $H(Y) = \sum\lim\limits_{Y_i \in Y}\big(- \p(Y_i)\cdot\log_2\p(Y_i)\big)$
	\item l'indice ajusté par chance de Rand (\textit{ajusted Rand index} - ARI) \citep{hubert1985adjustedrandidx} corrigent l'indice de Rand (RI) \citep{rand1971randidx} pour obtenir une valeur très proche de 0.0 pour les clusterings aléatoires et exactement 1.0 lorsque les clusters sont identiques aux classes attendues: %\[ARI(Y,C) = \frac{RI(Y,C) - E_{perm}\left[RI(Y,C)\right]}{1.0 - E_{perm}\left[RI(Y,C)\right]} \]
	\[ARI(Y,C) = \frac{\sum\limits_{i,j}\binom{n_{ij}}{2} - \frac{\sum\limits_{i}\binom{a_{i}}{2}\sum\limits_{j}\binom{b_{j}}{2}}{\binom{N}{2}}}{\frac{1}{2}\big[\sum\limits_{i}\binom{a_{i}}{2}+\sum\limits_{j}\binom{b_{j}}{2}\big] - \frac{\sum\limits_{i}\binom{a_{i}}{2}\sum\limits_{j}\binom{b_{j}}{2}}{\binom{N}{2}}}\]
	avec $\binom{n}{2} = \frac{n(n-1)}{2}$.
	% \item la précision du clustering (ACC) \citep{}.   \textcolor{red}{pas utilisé car nécessite de trouver le meilleur mapping entre les classes et les clusters}
	\item les métriques ARI et NMI se contentent de mesurer la différence des proportions entre les clusters de deux regroupements indépendamment des affectations des documents. D'autres méthodes appelées mesures de comptage de pair (\textit{pair counting measures}) mesurent la capacité du modèle à mettre deux documents similaires (de labels identiques dans les données annotées) dans le même groupe, et des documents dis-similaires (de label différents dans les données annotées) dans des clusters différents. Par exemple, des mesures de précision, rappel, et F1-mesure \citet{manning2009irbook-flatclustering} sont définies par les formules:
	\[P = \frac{{TP}}{{TP} + {FP}}, R = \frac{{TP}}{{TP} + {FN}}, F1 = \frac{2 \times P \times R}{P + R}.\]
	Les métriques de bases vrais/faux positifs/négatifs qui servent à les calculer, sont définies comme suit:
	\begin{itemize}
		\item un vrai positif (TP) survient si le modèle place deux documents similaires dans le même cluster (groupe généré par le modèle);
		\item un faux négatif (FN) survient si deux documents similaires sont dans des clusters différents;
		\item un vrai négatif (TN) est une décision qui place deux documents dissemblables dans deux clusters différents;
		\item un faux positif (FP) survient si deux documents dissemblables sont dans le même cluster.
	\end{itemize}
	
	
\end{itemize}
Ces métriques doivent être utilisées ensemble  pour compenser les limites de chacune d'elles \citep{yang2017kmeansfriendlyspaces}.



\subsubsection{Métriques non-supervisées ou indices internes}
\label{sec:similarite:biblio:unsupeval}

 La cohésion et la séparation des clusters sont les principale indices internes. La cohésion mesure le degré de proximité entre objets d'un cluster à partir du carré de la somme des erreurs\footnote{Erreur: distance entre un point et le centre du cluster dont il est membre.} dans les clusters: $WCSS(C) = \sum\limits_{j=1}^K\sum\limits_{x \in C_j} Dis(x, m_j)$, où $C = \lbrace C_1, C_1, \cdots, C_K \rbrace$ est l'ensemble des clusters du regroupement, $m_j$ le centre de $C_j$, et $Dis(x,m_j)$ la distance (généralement euclidienne) entre un point $x$ et $m_j$. En général, une valeur faible de la cohésion indique que les clusters sont plus compactes, et donc de meilleur qualité. Tandis qu'une valeur élevée révèle une grande variabilité entre les objets à l'intérieur les clusters. La séparation quant à elle mesure l'éloignement de chaque cluster des autres à partir du carré de la somme des distances entre clusters: $BCSS(C) = \sum\limits_{j = 1}^{K} \vert C_j \vert (m - m_j)$, 
  $m$ étant le centre l'ensemble des objets (la moyenne des vecteurs de tous les documents, où le document qui se trouve à une distance moyenne minimale de tous les autres). Une grande valeur de séparation indique que les clusters sont isolés les uns des autres, et par conséquent elle doit être maximisée pendant le regroupement.
 
 Le coefficient de silhouette de \citet{rousseeuw1987silhouetteclusternumber} combine les idées de cohésion et séparation mais pour chaque objet. Il se calcule par $s(x) = \frac{b(x
) - a(x)}{\max\lbrace a(x)b(x) \rbrace}$, où $x$ est un objet du cluster $A \in C$, $a(x) = \frac{1}{\vert A \vert} \sum\limits_{y\in A} Dis(x,y)$, $b(x) = \min\limits_{B \in C}\frac{1}{\vert B \vert} \sum\limits_{y\in B} Dis(x,y)$. Ses valeurs varient entre -1 (pire valeur) et +1 (meilleure valeur). Les valeurs proches de zéro indiquent que les clusters se chevauchent en $x$, et il est difficile de savoir à quel cluster $x$ doit être affecté. Une valeur négative indique que $x$ a été affecté à cluster inapproprié. Le coefficient de silhouette du regroupement $C$ est la moyenne des coefficients de l'ensemble des  objets.

%\section{Méthodes proposées}

%\subsection{K-médoïdes et \og Word Mover's Distance \fg}

%Les approches de clustering sont généralement appliquées à une représentation vectorielle des objets. Particulièrement la méthodes des K-moyennes qui met à jour le centroïde en faisant la moyenne des menbres de son cluster. Cependant, \citet{kusner2015wordmoverdist} ont proposé récemment \textit{la distance du déplaceur de mot} (\textit{word mover's distance - WMD}), une métrique non-supervisée qui permet à la méthode des \textit{K plus proches voisins} (KNN) d'obtenir des performances sans précédents. De plus, l'algorithme de clustering K-médoïdes \citep{kaufman1987kmedoids}, similaire aux K-moyennes, choisi comme centroïde le membre du cluster qui minimise la distance aux restes des membres; ce qui n'impose pas de représentation vectorielle. Ainsi, nous pouvons utiliser la métrique WMD dans  l'algorithme des K-médoïdes. Tout en nous appuyant sur un algorithme établi de clustering, nous évitons aussi la recherche de la meilleure représentation vectorielle qui influence souvent les performances du clustering. 

%Algorithme: \url{http://isiarticles.com/bundles/Article/pre/pdf/79087.pdf}


%Un des désavantage de l'algorithme des K-médoïdes est son long temps de calcul dû à ???. Nous avons, par conséquent, remplacé la distance euclidienne par la WMD dans la version plus rapide de \cite{Park2009fastkmedoids} avec nombre de clusters prédéfinis, et celle de \cite{sabzi2011fuzzykmedoids} qui intègre une optimisation du nombre de clusters.
%\section{Méthode 2: cartes auto-organisatrice de Kohonen et concaténation de plongement lexical de phrases (sentence embedding)}

\section{Apprentissage d'une distance basée sur la transformation de document}
Nous définissons la métrique suivante qui est fonction des transformations permettant de passer d'un document $d_i$ à un autre $d_j$ :
\begin{equation}
\begin{array}{cccccc}
Dis_\mathcal{M} & : & \mathcal{D} \times \mathcal{D} & \to & \mathbb{R} & \\
& & d_i, d_j & \mapsto & Dis_{\mathcal{M}}(d_i, d_j) & = f(\mathcal{M}_{d_i, d_j}). \\
\end{array} \label{eq:similarite:distance-modif}
\end{equation}

$\mathcal{D}$ est le corpus. $\mathcal{M}_{d_i, d_j}$ est l'ensemble des modifications de $d_i$ permettant pour obtenir $d_j$ i.e. les paires de mots différents $(d_{ik}, d_{jk})$ telles que le mot $d_{ik}$ à la position $k$ dans $d_i$ a été remplacé par $d_{jk}$ à la position $k$ dans $d_j$. $f$ est une fonction qui croît avec le nombre de modifications. Après une légère modification, le sens d'un texte reste assez similaire à celui de l'original. Tandis qu'après un grand nombre de modifications, le sens du texte sera très différent de l'original. 
 %La distance croît donc proportionnellement avec le taux de modification (d'où le produit de $Dis_{soft-cos}$ et $f$). $Dis_{soft-cos}$  est étendue car elle tient compte 

%Un intérêt de cette distance est qu'elle tient compte de l'ordre des mots contrairement à une distance classique comme la distance cosinus. Par exemple, en considérant les deux textes  $t_1 = \text{\og \textit{le chien poursuit le chat}\fg{}}$ et $t_2 = \text{\og \textit{le chat poursuit le chien} \fg{}}$, et en supposant que $f$ est simplement le taux de mots modifiés:
%\begin{equation}
%\forall (d_i, d_j) \in \mathcal{D} \times \mathcal{D}, f(\mathcal{M}_{d_i, d_j}) = \frac{\norm{\mathcal{M}_{d_i, d_j}}}{\norm{d_i}}, \label{eq:similarite:taux-modif}
%\end{equation}
%
%nous avons $Dis_{cos}(d_1, d_2) = 0.$ (car les deux textes contiennent exactement les mêmes mots) et $Dis_\mathcal{M}(d_1, d_2)=0.4$ (car $\mathcal{M} = \lbrace (\textit{chien}, \textit{chat}), (\textit{chat}, \textit{chien})\rbrace$,  et $\norm{d_1} = 5 \text{ mots}$). La limite de cette distance est d'être très stricte sur l'ordre des mots. C'est pourquoi sa combinaison avec la distance cosinus est explorée dans ce chapitre. 

Pour des textes de même taille, toute formulation mathématique permettrait de calculer $Dis_\mathcal{M}$ car il est facile de faire correspondre les mots par leur position. Par exemple, cette distance peut donc se formuler comme étant la proportion de mots modifiés:
\begin{equation}
{Dis_\mathcal{M}}(d_i, d'_i) = {f}(\mathcal{M}_{d_i, d'_i}) = \frac{\vert \mathcal{M}_{d_i, d'_i} \vert}{\vert d_i \vert } \label{eq:similarite:taux-modif}
%  $\vert d_i \vert$ étant le nombre de mots de $d_i$.
\end{equation}
 Par contre, pour des textes de tailles différentes, il est impossible de savoir les positions où des mots ont été supprimés ou ajoutés, et par conséquent, il devient impossible d'estimer leur distance par une formule. La distance étant une valeur continue, en entraînant un modèle de régression sur un ensemble de paires de documents pour lesquelles on connaît la distance, il est possible de la prédire pour des paires de textes de tailles quelconques. Nous proposons de générer une base synthétique de paires de documents dont l'un est un document du corpus original mais l'autre est le résultat de substitutions et suppression de mots du premier. En contrôlant ces modifications, il est facile de calculer une valeur de $Dis_\mathcal{M}$ pour chaque paire générée de documents, même s'ils sont de tailles différentes (en considérant la suppression d'un mot comme son remplacement par le \og mot vide \fg{} considéré comme faisant partie du vocabulaire $W$).

\subsection{Génération d'une base d'apprentissage}
La génération de la base synthétique nécessite de définir une formulation de la fonction $f(\mathcal{M}_{d_i, d_j})$ pour les documents de taille égale, comme par exemple celle de l'Equation \ref{eq:similarite:taux-modif}. Cette formulation ne considérant pas la similarité entre les mots substituants et les remplacés, nous proposons de pondérer chaque modification par la distance entre les mots substitués (le vecteur du  \og mot vide \fg{} étant nul):
\begin{equation}
{Dis_\mathcal{M}}(d_i, d'_i) = {f}(\mathcal{M}_{d_i, d'_i}) = \frac{\sum\limits_{(d_{ik}, d'_{ik}) \in \mathcal{M}_{d_i, d'_i}} Dis_{cos}(\vec{d_{ik}}, \vec{d_{ik}})}{\vert d_i \vert} \label{eq:similarite:somme-dist-mots}
\end{equation}
$d_i$ est un document du corpus original $\mathcal{D}$, et $d'_i$ est le document synthétique obtenu par modifications contrôlées de $d_i$. $\vec{d_{ik}}$ désigne le plongement lexical du mot $d_{ik}$. Pour garantir la symétrie et la réflexivité de la métrique, nous imposons respectivement ${Dis_\mathcal{M}}(d_i, d'_i) = {Dis_\mathcal{M}}(d'_i, d_i)$ et ${Dis_\mathcal{M}}(d_i, d_i) = {Dis_\mathcal{M}}(d'_i, d'_i) = 0, \forall d_i \in \mathcal{D}$ sur le jeu d'entraînement généré. L'algorithme de génération de document synthétique utilise une valeur seuil de probabilité $0<p<1$ contrôlant le taux de modifications à effectuer sur le document original (Algorithme \ref{algo:similarite:modifierdoc}). 

\begin{algorithm}[!htb] % Version française avec : https://pierre.chachatelier.fr/latex/index.php
 \KwData{document $d_i \in \mathcal{D}$, valeur seuil de probabilité $p$, le vocabulaire $W$}
 \KwResult{$d'_i$, $\mathcal{M}_{d_i, d'_i}$}
 ${d'_i} = [] $\; 
 $\mathcal{M}_{d_i, d'_i} = \emptyset$\;
 \For{ $k \in [1\twodots \vert d_i \vert ]$ }{
 	$v = random(0,1)$\;
    \eIf{v < p}{
       $d'_{ik} = modifier\_mot(d_{ik}, W)$; // mot aléatoire de $W$ différent de $d_{ik}$\;
       $\mathcal{M}_{d_{ik},d'_{ik}} = \mathcal{M}_{d_i, d'_i} \cup \lbrace (d_{ik},d'_{ik}) \rbrace$\;
     }{
     $d'_{ik} = d_{ik}$\;
     }     
 }
 \Return $d'_i, \mathcal{M}_{d_i, d'_i}$\;
 \caption{\textit{modifier\_document($d_i, p, W$)}} \label{algo:similarite:modifierdoc}
\end{algorithm}

Pour une même valeur ou des valeurs différentes de $p$, plusieurs documents sont ainsi générés pour chaque $d_i \in \mathcal{D}$ pour former un ensemble de données d'entraînement $B_\mathcal{M} = \lbrace (d_i, d'_i, {Dis}(d_i, d'_i))\rbrace_{1 \leq i \leq \norm{B_\mathcal{M}}}$.

\subsection{Entraînement de la métrique}

Sur $B_\mathcal{M}$, on entraîne un modèle régressif $m$ pour prédire la distance entre deux documents quelconques en fonction de leur représentation vectorielle. Ce modèle de régression $Reg_\mathcal{M}$ peut être utilisé comme distance dans un algorithme de regroupement comme l'algorithme des K-moyennes. Cependant, les modèles de régression ne supportent généralement qu'un seul vecteur en entrée, et pas deux comme en dispose la base $B$. Les vecteurs $d_i$ et $d'_i$ doivent donc être agrégés en un seul. L'agrégation qui fonctionne le mieux est la soustraction avec laquelle les documents similaires auront une agrégation avec un grand nombre de composantes tendant vers 0. La fonction d'estimation automatique de la distance sémantique entre $x$ et $y$ s'écrit: $Dis_\mathcal{M}(d_i, d_j) = Reg_\mathcal{M}(\vec{d_{i}} - \vec{d_{j}})$. 

%\textcolor{red}{Issues:}
%\begin{itemize}
%\item les docs sont généralement de tailles différentes, ne faudrait il pas intégrer une perturbation ajout de mots? \textcolor{blue}{la suppression peut être considérée comme le remplacement d'un mot par le mot vide, qui doit être ajouté aux word2vec}
%\item il faudrait intégrer la composante taille du document: \textcolor{blue}{agréger sur le nombre minimal de phrases des paires de documents}
%\item comment assurer les propriétés d'une fonction similarité? par exemple si aucune perturbation n'est opérée, alors la similarité est maximale et si tous les mots sont modifiés alors la similarité est minimale: \textcolor{blue}{agrégation par soustraction des vecteurs du couple de docs. plus deux doc seront similaire, plus le vecteur de leur paire tendra vers le nul}
%\item Ne faudrait il pas prendre en compte un poids pour les mots, car peut-être la modification de certains mots ne devrait pas avoir le même impact sur la similarité ou le taux de perturbation que celle d'autres mots:  \textcolor{blue}{lissage par la somme des distance des vecteurs de mots substitués Eq. \ref{equation:similarite:somme-dist-mots}}
%\item ne faudrait il pas intégré une métrique proche de la tâche: la ressemblance n'est pas forcément globale à tous le corps du document mais plus à certaines régions; donc un document auquel on rajoute quelques phrases ne devrait pas voir  son sens trop changer:  \textcolor{blue}{peut-être agréger les distances minimales entre les paires de phrases}
%\end{itemize}

\section{Interprétation des résultats expérimentaux}
\label{sec:similarite:experimentations}
Cette section discute la validité, l'adéquation, et l'efficacité de la métrique apprise en comparaison avec d'autres métriques d'estimation de la similarité sémantique entre documents. La validité de la métrique est établie si cette dernière respecte les propriétés d'une distance. L'adéquation de la métrique avec le problème à résoudre mesure la capacité de la métrique à estimer une distance très faible entre documents du même label (annotation manuelle), en même temps qu'une similarité quasi nulle entre documents de labels différents, indépendamment de l'algorithme de clustering appliqué. Enfin, l'efficacité de la métrique est liée à la qualité de clustering résultant de l'application d'un algorithme de clustering combiné avec la métrique apprise.


\subsection{Annotation manuelle de données d'évaluation}
Pour l'évaluation supervisée, nous disposons d'une base annotée sur la catégorie de demande "dommage-intérêts / action en responsabilité civile professionnelle contre les avocats" (\textit{DIRA}) qui concerne les contentieux impliquant des avocats.  L'expert annotateur a identifié quatre cas différents :
\begin{itemize}
\item cas $a$: il s'agit d'un avocat qui est négligent et envoie son assignation de manière tardive (champ sémantique: retard/délai/prescription)
\item cas $b$ il s'agit d'un avocat qui n'a pas donné un conseil opportun, qui n'a pas soulevé le bon argument
\item cas $c$ un avocat qui n'a pas rédigé un acte valide ou réussi à obtenir un avantage fiscal (champ sémantique: rédacteur d'actes)
\item cas $d$: il s'agit d'un avocat attaqué par son adversaire et non par son propre client.
\end{itemize}

Cet ensemble de données comprend 81 documents répartis dans les 4 groupes (Figure \ref{fig:similarite:dira-data-distrib}) avec 6 documents appartenant chacun à 2 groupes (cas de chevauchements).

\begin{figure}[!htb]
	\centering \includegraphics[scale=0.5]{dira-data-distrib.png}
	\caption{Répartition des documents annotés par circonstances factuelles (\textit{DIRA}).}\label{fig:similarite:dira-data-distrib}
\end{figure}

Pour l'évaluation non supervisée, les catégories de demande utilisée dans les chapitres \ref{chap:quanta} et \ref{chap:sensresultat} sont utilisés en plus.

\subsection{Protocole}
Pour analyser la validité et l'adéquation de la métrique apprise, nous l'entraînons sur la base générée puis nous l'évaluons sur le corpus annoté ($\mathcal{D_{\text{DIRA}}}$). Quant à l'efficacité de la métrique apprise, nous l'entraînons sur toutes les données annotées générées.

Les documents sont pré-traités avant leur représentation sous forme vectorielle. Ce prétraitement consiste à sectionner les documents (chapitre \ref{chap:structuration}), à n'utiliser que la concaténation des section Litige et Motifs (\textcolor{red}{pourquoi?}), à mettre en minuscule et lemmatiser ce contenu, puis à éliminer des caractères de ponctuation et des mots inutiles (\textit{stop words})  car ils sont généralement indépendants de toute catégorie.

Le vocabulaire $W$ utilisé est restreint au mots du corpus original $D$ sur lequel il faut appliquer les regroupements. La représentation vectorielle est TF-IDF avec des n-grammes de 1 à 3 mots (pour prendre en compte l'ordre entre les mots).

\subsection{Validité de la distance apprise}
La base d'entraînement $B$ comprend 10 documents synthétiques générés pour chacun des 74 documents n'ayant qu'un seul label. Nous vérifions ici que la métrique respecte les propriétés des distances, et aussi si elle reste normale après l'entraînement. D'après la matrice des distances entre toutes les paires de document de la base annotées $\mathcal{D_{\text{DIRA}}}$ (Figure \ref{fig:similarite:distance_matrix}),  la "non-négativité", l'identité discernable, et la symétrie sont respectée car toutes les valeurs sont non-négative, seule la diagonale est nulle, et la matrice est symétrique. De plus, toutes les distances sont comprises entre 0 et 1, et par conséquent la métrique est normée (donc la similarité est déduite par $Sim_\mathcal{M} = 1 - Dis_\mathcal{M}$).

\begin{figure}[!htb]
	\centering \includegraphics[scale=0.8]{distance_matrix.png}
	\caption{Matrice des distances de la métrique apprise sur les documents labellisés}\label{fig:similarite:distance_matrix}
\end{figure}

L'inégalité triangulaire est vérifiée car $Reg_\mathcal{M}(X,Z) - (Reg_\mathcal{M}(X,Y) + Reg_\mathcal{M}(Y,Z)) \leq 0, \forall (X,Y,Z) \in \mathcal{D_{\text{DIRA}}} \times \mathcal{D_{\text{DIRA}}} \times \mathcal{D_{\text{DIRA}}}$ (Figure \ref{fig:similarite:matrice_inegalite_triangulaire}).

\begin{figure}[!htb]
	\centering \includegraphics[width=\textwidth]{inegalite_triangulaire.png}
	\caption{Matrice de $Reg_\mathcal{M}(X,Z) - (Reg_\mathcal{M}(X,Y) + Reg_\mathcal{M}(Y,Z)), \forall (X,Y,Z) \in \mathcal{D_{\text{DIRA}}} \times \mathcal{D_{\text{DIRA}}} \times \mathcal{D_{\text{DIRA}}}$, avec les $X$ indicés en lignes et les paires $(Y,Z)$ en colonnes. }\label{fig:similarite:matrice_inegalite_triangulaire}
\end{figure}
 


\subsection{Adéquation de la métrique apprise avec le problème}
L'adéquation est mesurée par la formule suivante qui doit tendre vers 0: \[A(Dis) = \sum\limits_{\substack{(l,k) \in L \times L \\ l \neq k}} \sum\limits_{\substack{d \in D_l \\ d' \in D_k}} Sim(d,d') + \sum\limits_{l \in L} \sum\limits_{\substack{d \in D_l \\ d' \in D_l}} Dis(d,d). \] $L = \lbrace a, b, c \rbrace$ est l'ensemble des labels. Pour tout $l \in L$, $D_l \subset \mathcal{D}$ désigne le sous-ensemble des documents de $\mathcal{D}$ qui ont le label $l$. Cette formule tend vers 0 lorsque la somme des similarités entre documents de labels différents (premier terme) et la somme des distances entre documents de labels égaux (second terme) tendent tous les deux vers 0.
Sur les 74 document annotés, nous obtenons $A(Dis_\mathcal{M}) = 872.635 + 519.151 = 1391.786$ avec une moyenne de similarité entre labels de $0.537$ et de distance entre labels égaux de $0.482$.

L'évaluation des algorithmes de regroupement considèrent parfaite la distance utilisée. En considérant les données annotées (regroupement parfait), les indices internes (section \ref{sec:similarite:biblio:unsupeval}) permettent d'évaluer les distances. Ces dernières sont ainsi comparées dans le Tableau \ref{tab:similarite:compare-dist-adequation}.

\begin{table}[!htb]
	\centering
  \begin{tabular}{|l|c|c|c|c|}
  	\hline
	$Dis$ & Cohésion & Séparation & Silhouette &  $A(Dis)$ \\ \hline
	$Dis_\mathcal{M}$ &&&& 1391.786 \\ \hline
	$Dis_{cos}$ &&&& \\ \hline
	$Dis_{soft-cos}$ &&&&  \\ \hline
	$Dis_{manhattan}$ &&&& \\ \hline
	$Dis_{euclidienne}$ &&&&  \\ \hline
	$Dis_{jaccard}$ &&&&  \\ \hline
	$Dis_{pearson}$ &&&& \\ \hline
	$Dis_{wmd}$ &&&&  \\ \hline	
  \end{tabular}
	\caption{Tableau comparatif de l'adéquation des distances au problème} \label{tab:similarite:compare-dist-adequation}
\end{table}

\subsection{Comparaison des distances de similarité pour le regroupement}
%Comparer la vectorisation du document sur tout son contenu vs. sur la restriction aux énoncés de demande de la catégorie (du type "constater", "dire et juger") vs restriction aux conclusions (le raisonnement des parties décrits les circonstances factuelles) + motifs sur la catégorie

%avec choix du nombre de clusters

Nous comparons ici les distances lorsqu'elles sont utilisées pour un partitionnement disjoint avec les algorithmes de k-means et k-medoids. 

\begin{table}[!htb]
	\centering
	\begin{tabular}{|l|c|c|c||c|c|c|}
		\hline
		$Dis$ & $K$ & $ARI$ & $NMI$ & $precision$ & $rappel$ & $F1-mesure$  \\ \hline
		$Dis_\mathcal{M}$ &3&&&&&  \\ \hline
		$Dis_{cos}$ &3&&&&&  \\ \hline
		$Dis_{soft-cos}$ &3&&&&&  \\ \hline
		$Dis_{manhattan}$ &3&&&&&  \\ \hline
		$Dis_{euclidienne}$ &3&&&&&  \\ \hline
		$Dis_{jaccard}$ &3&&&&&  \\ \hline
		$Dis_{pearson}$ &3&&&&&  \\ \hline
		$Dis_{wmd}$ &3&&&&&  \\ \hline	
		
	\end{tabular}
	\caption{Tableau comparatif de l'efficacité des distances pour le regroupement (évaluation supervisée)}\label{tab:similarite:compare-dist-evalsup}
\end{table}


\begin{table}[!htb]
	\centering
	\begin{tabular}{|l|c|c|c|c|}
		\hline
		$Dis$ & $K$ & Silhouette & Cohésion & Séparation \\ \hline
		$Dis_\mathcal{M}$  &3&&&  \\ \hline
		$Dis_{cos}$  &3&&&  \\ \hline
		$Dis_{soft-cos}$  &3&&& \\ \hline
		$Dis_{manhattan}$  &3&&&  \\ \hline
		$Dis_{euclidienne}$  &3&&&  \\ \hline
		$Dis_{jaccard}$  &3&&&  \\ \hline
		$Dis_{pearson}$  &3&&&  \\ \hline
		$Dis_{wmd}$  &3&&&  \\ \hline	
	\end{tabular}
	\caption{Tableau comparatif de l'efficacité des distances pour le regroupement (évaluation non-supervisée)}\label{tab:similarite:compare-dist-evalnonsup}
\end{table}


\subsection{Comparaison des algorithmes de regroupement}

\section{Conclusion}
\label{sec:similarite:conclusion}
jhk
lk

lkjkl %INCLUDE: similarite
\chapter{Application à l'analyse descriptive d'un grand corpus}
\label{chap:demo}
Ce chapitre décrit des résultats d'analyses statistiques observés lors de l'application de la chaîne proposée (Figure \ref{fig:intro:pipeline-globale}) sur un corpus formé de la base CAPP de la \citet{dila2019capp} (+65k XML sur 1997-2019), et plus de 10k autres documents de cours d'appels. Le module d'extraction d'information (Figure \ref{fig:demo:module-extraction}) est un système qui comprend les modèles développés durant les expérimentations de cette thèse. 

%  une base du tribunal de commerce de Paris (300k MS DOC sur ?-?), 500k décisions collectées de Legifrance
%à CAPP\_20190805-214041.tar.gz et Freemium_capp\_global\_20180315-170000.tar.gz 

%\verb|wget -c --accept='*.tar.gz' -r  ftp://echanges.dila.gouv.fr/CAPP/|

\begin{figure}[!htb]
	\centering 
	\includegraphics[width=0.8\textwidth]{pipeline-demo.png}
	\caption{Détails du module d'extraction d'information}\label{fig:demo:module-extraction}
\end{figure}

Après extraction, la base de données comprend environ 65k documents répartis dans l'espace (ville) comme sur la figure \ref{fig:demo:doc-per-city} et dans le temps comme sur la figure \ref{fig:demo:doc-per-year}. Les demandes extraites se répartissent par catégorie comme suit: 476 ACPA, 409 CONCDEL, 160 DANAIS, 0 DCPPC, 34 DORIS, et 45928 STYX.  

\begin{figure}[ht]
	\centering
	\begin{subfigure}[ht]{0.5\textwidth}
		\centering
		\centering
		\includegraphics[width=\textwidth]{demo-pourcentage-decision-par-ville.png}
		\caption{Par ville (pourcentage>0)} \label{fig:demo:doc-per-city}
	\end{subfigure} 
	\begin{subfigure}[ht]{0.45\textwidth}
		\centering
		\includegraphics[width=\textwidth]{demo-repartition-decision-par-annee-1900_2019.png}
		\caption{Par an (entre 1900 et 2019)} \label{fig:demo:doc-per-year}
	\end{subfigure}
	\caption{Répartition des décisions} \label{fig:structuration:learning-curves}
\end{figure}


La structuration des données dans la base de données permet de mieux comprendre la jurisprudence à l'aide de graphiques appropriés. Une application de visualisation a notamment été développée suivant les besoins d'un expert juriste par \citet{PRYSIAZHNIUK2017jurisprudence-demo-web}.
Les analyses des sections suivantes sont restreintes aux 5 villes ayant les plus grands nombres de décision: Paris, Lyon, Versailles, Angers, Bastia; sur la période 2000-2019

\section{Analyse du sens du résultat}
A partir de la base des données extraites, l'évolution du pourcentage de demandes acceptées  peut être observée sur une courbe. En traçant une telle courbe pour chaque ville il est possible de comparer les villes.
Par exemple, pour les dommages intérêts sur l'article 700 du Code de Procédure Civile (STYX), la Figure \ref{fig:demo:analyse-sens-resultat-styx} compare l'évolution du sens du résultat entre les villes citées précédemment. On remarque que les demandes sont beaucoup plus rejetées qu'acceptées. La courbe du nombre total de demandes doit être associée pour savoir si le pourcentage de succès est considérable\footnote{Pour une année ou une seule demande est extraite et acceptée, le pourcentage est à 100\%, mais ce n'est pas considérable.}.

\begin{figure}[!htb]
	\centering 
	\includegraphics[width=0.9\textwidth]{evolution_sens_resultat_styx.png}
	\includegraphics[width=0.9\textwidth]{evolution_nbdmd_styx.png}
	\caption{Evolution du sens du résultat des demandes STYX dans le temps à Paris, Lyon, Versailles, Angers, Bastia.}\label{fig:demo:analyse-sens-resultat-styx}
\end{figure}

La visualisation par l'application de \citet{PRYSIAZHNIUK2017jurisprudence-demo-web} permet de comparer les villes en observant sur un arbre l'épaisseur des branches associées aux catégories de demande (Figure \ref{fig:demo:web-styx}). On peut ainsi facilement observer quelles villes acceptent les demandes d'une certaine catégorie plus que d'autres par exemple.

% webdemo-sensresultat-5villes.png

\begin{figure}[!htb]
	\centering 
	\includegraphics[width=0.6\textwidth]{webdemo-sensresultat-5villes.png}
	\caption{Comparaison des Paris, Lyon, Versailles, Angers, Bastia sur l'acceptation des demandes STYX à partir d'une visualisation arborée.}\label{fig:demo:web-styx}
\end{figure}

\section{Analyse des quanta}
\subsection{Evolution dans le temps}
De même l'évolution des quanta demandés et accordés peut être facilement visualisée par un diagramme en barre comme celui de la Figure \ref{fig:demo:evolution-quanta-styx} qui correspond aux demandes STYX entre 2000  et 2019. Même si le nombre total de demandes est à prendre en compte, un tel diagramme donne un aperçu des sommes d'argent demandés et accordés chaque année. 

\begin{figure}[!htb]
	\centering 
	\includegraphics[width=0.6\textwidth]{evolution_quanta_styx.png}
	\caption{Evolution des quanta moyens par année des demandes STYX entre  2000  et 2019.}\label{fig:demo:evolution-quanta-styx}
\end{figure}


\subsection{Différence dans l'espace}

Pour avoir une idée du montant qu'on peut recevoir sur pour une catégorie de demande, l'évolution des valeurs généralement accordés peut être comparée entre deux villes en visualisant les diagrammes boîtes (\textit{boxplot}) 
des quanta accordés dans ces villes. La Figure \ref{fig:demo:evolution-qr-styx-compare-ville} permet d'effectuer une comparaison entre Bastia et Lyon. 

\begin{figure}[!htb]
	\centering 
	\includegraphics[width=0.6\textwidth]{qr_STYX_Bastia.png}
	\includegraphics[width=0.6\textwidth]{qr_STYX_Lyon.png}
	\caption{Evolution des quanta accordés par année sur les demandes STYX entre 2000 et 2016 à Bastia et à Lyon.}\label{fig:demo:evolution-qr-styx-compare-ville}
\end{figure}


\subsection{Quantum demandé vs. quantum accordé}
La méthode de prédiction du quantum résultat doit définir un modèle dont la forme s'accorde avec celle du nuage de points correspondant. Le quantum demandé ne semble pas suffisant seul pour déterminer le quantum accordé\footnote{Différentes valeurs de quantum résultat observées pour la même valeur de quantum demandé.}. Il sera ainsi nécessaire de tenir compte des circonstances factuelles et autres spécificités du cas traité qui permettront de filtrer les décisions sur lesquelles se basera la modélisation. On remarque néanmoins une ressemblance de forme entre les nuages de points des différentes villes. 

\begin{figure}[!htb]
	\centering 
	\includegraphics[width=0.47\textwidth]{qr-vs-qd_STYX_Bastia.png}
	\includegraphics[width=0.47\textwidth]{qr-vs-qd_STYX_Lyon.png}
	\includegraphics[width=0.47\textwidth]{qr-vs-qd_STYX_Paris.png}
	\includegraphics[width=0.47\textwidth]{qr-vs-qd_STYX_Angers.png}
	\caption{Nuages des points (quantum accordé, quantum demandé) pour les demandes STYX entre 2000 et 2019 à Bastia et à Lyon (quantum demandé < 10000) .}\label{fig:demo:qr-vs-qd-styx-compare-ville}
\end{figure}

\section{Conclusion}
\label{sec:demo:conclusion}
Les démonstrations de ce chapitre donnent quelques exemples de statistiques qui informent de l'état de la jurisprudence à partir d'informations extraites à l'aide des approches proposées dans cette thèse. Les analyses du sens du résultat et des quanta sont les principales applications directes de la chaîne de traitement développée. Ce chapitre se limite aux filtres sur l'année, la ville, et la catégorie de demande, mais les analyses peuvent déjà être affinées en associant d'autres filtres comme  des mot-clés,  les normes appliqués, ou le type de juridiction. Les analyses pourront être plus riches grâce l'extraction future de nouvelles informations comme les motivations des juges et de meilleurs modèles de circonstances factuelles. %INCLUDE: demo


\backmatter

% changement du style des chapitres
%\renewcommand{\appendixname}{Annexe }
\renewcommand{\thechapter}{\Alph{chapter}}
%\renewcommand{\cftchappresnum}{\hspace{0.5cm}Annexe: }
%{\usefont{OT1}{ptm}{b}{n}
 \titleformat{\chapter}[display]
{\bfseries\Large\filleft}
{\filright {\appendixname } 
\Large\thechapter}
{1ex}
{\titlerule[3pt]%
\vspace{2pt}%
\titlerule
\vspace{2ex}%
\filright}
[\vspace{2ex}%
\titlerule]

\def\chaptertitlename{Chapitre }
%\renewcommand{\thepart}{\arabic{part}}
%{\usefont{OT1}{ptm}{b}{n}
 \titleformat{\part}[display]
{\bfseries\Huge\filleft}
{\filright {\partname } %\textcolor{blue}
\Huge}
{1ex}
{\titlerule[3pt]%
\vspace{5pt}%
\titlerule
\vspace{5ex}%
\filright} % \textcolor{blue}
[\vspace{5ex}%
\titlerule]

%
%}

% % partie
%  \titleformat{\part}[frame]
% {\normalfont}
% {\filright
% \footnotesize
% \enspace \bf \Huge \thepart\enspace}
% {18pt}
% {\Large\bfseries\filcenter}

\chapter*{Conclusion générale}
\addcontentsline{toc}{chapter}{Conclusion générale}
\label{chap:conclusion}

\textcolor{red}{Pourquoi n'avoir pas utilisé des méthodes de deep learning la thèse? disponibilité des approches de l'état de l'art, peu de données labellisées.}

\section{Évaluation des contributions}
\label{sec:conclusion:contributions}
Cette thèse  porte essentiellement sur la proposition et l'exploration d'approches adressant des problèmes d'analyses de données textuelles rencontrés lors de l'étude de corpus jurisprudentiels par des experts juristes. Trois problèmes principaux y sont abordés. Premièrement, l'annotation, dans les documents, des sections de textes et des entités nommées propres au domaine judiciaire qui peuvent aider à se repérer dans le document et à améliorer la recherche d'information. Le chapitre \ref{chap:structuration}  démontre empiriquement, sur des documents annotés pour la circonstance, l'efficacité de l'application de modèles probabilistes d'étiquetage de séquences, HMM et CRF, sur les deux tâches. Par la suite, l'extraction de données relatives aux demandes, suivant leur catégorie juridique, est discutée dans les chapitres \ref{chap:quanta} et \ref{chap:sensresultat}. Le problème impose d'effectuer les extractions pour une catégorie de demande à la fois car il est impossible d'annoter suffisamment de données pour toutes les catégories prédéfinies. Pour cela nous proposons de filtrer à l'entrée les documents de la catégorie à traiter par une classification binaire. Ensuite, il est proposé une approche ad-hoc identifiant d'une part les quanta demandés et accordés à l'aide de la position de termes clés appris sans exemple grâce à des métriques de pondération des termes, et d'autre part le sens du résultat à l'aide d'un ensemble prédéfini de mots-clés particulier à la rédaction des résultats. Cette méthode, bien que dépendante d'heuristiques, parvient à reconnaître un grand nombre de demandes avec plus ou moins de difficultés selon les catégories traitées. Ensuite, la classification de documents est expérimentée comme approche plus généraliste. Sur l'ensemble des algorithmes explorés, les extensions de l'analyse PLS, appliquées ici pour la première fois sur du texte, démontrent une efficacité proche de celle du meilleur algorithme testé, l'arbre de décision.  L'utilité de la restriction des documents à des passages relatifs à la catégorie est aussi démontrée empiriquement. Enfin, le chapitre \ref{chap:similarite} aborde la problématique de similarité entre deux textes dans un contexte de catégorisation non supervisée des documents. Le but est ici de révéler les circonstances factuelles faisant appel à une catégorie de demande particulière. Une approche d'apprentissage de distance est proposée: elle repose sur le coût d'une transformation d'un des deux textes en l'autre.
Cette distance est comparée à d'autres métriques avec l'algorithme des K-moyennes dans des expérimentations qui explorent différents aspects des problèmes de regroupement comme la détermination du nombre de clusters ou la représentation de documents. En somme, les problématiques abordées sont certes variées mais indispensables aux différents maillons de la chaîne complète de traitement automatique de corpus de décisions dont le chapitre \ref{chap:demo} montre l'utilité pour visualiser l'état de la jurisprudence, une des nombreuses applications possibles des données extraites. 


\section{Critique du travail}
\label{sec:conclusion:critique}
Au delà des nombreuses problématiques abordées et expérimentations discutées, cette thèse reste limitée par son niveau de contribution théorique d'une part. La proposition globale est une chaîne de traitement employant à chaque niveau des approches soit existantes soit plus techniques. Aussi, un très grand nombre de méthodes de la littérature sont absentes, surtout les plus récentes; ceci est dû fait à l'ampleur du travail et à la multitudes d'approches existantes.  D'autre part, les études menées ont rencontrées comme obstacles la disponibilité d'exemples de référence annotées manuellement. La lenteur et la pénibilité de l'identification des informations à la main se traduit par la faible quantité des données employées pour les expérimentations. De plus, ne disposant que d'un expert, le degré d'accord entre annotateurs n'a été analysé que pour la première problématique de reconnaissance d'entités et de sections. Par conséquent, certaines subtilités propres à l'expert ou des données manquées lors de l'annotation manuelle, peuvent biaiser les résultats observés. Néanmoins, les nombreux résultats obtenus servent de base pour la continuité des études. 


\section{Travaux futurs de recherche}
\label{sec:conclusion:extensions}
Les propositions données dans la conclusion des chapitres \ref{chap:structuration} à \ref{chap:demo} pour continuer les travaux peuvent être résumées en 4 catégories principale. En premier, l'exploration de méthodes récentes que celles étudiées permettra d'étendre les résultats expérimentaux. Ensuite, la formalisation des problèmes abordées permettra de définir des approches plus théoriques. Par exemple, la formalisation des demandes comme des relations, entre quantum demandé et quantum accordé, permettra d'explorer le cadre probabiliste et neuronale de la littérature en matière d'extraction des relations. Puis, l'exploration d'autres formulation des problèmes permettra probablement de découvrir des méthodes plus efficaces. Par exemple, on peut percevoir la détermination des circonstances factuelles comme une tâche de modélisation de thématiques (\textit{topic modeling}). Enfin, l'exploration plus approfondie d'autres aspects des problèmes. Par exemple, la reconnaissance d'entités nommées comprend l'identification que nous avons étudiée, et la résolution qui unifie les mentions variantes d'une entité sous un identifiant prédéfinir ou à définir automatiquement.

Il faut aussi remarquer qu'il reste encore des types d'information dont le problème d'extraction n'est pas abordé par cette thèse. Par exemple, les raisons, qui  font penchés les juges en faveur d'une décision sur une demande, sont indispensables pour être capable d'anticiper la prise de décision des juges. L'extraction des raisons concernera l'identification et l'analyse des arguments des parties et les motivations des juges.

 Par ailleurs, il faudra aussi mieux évaluer la qualité des annotations manuelles expertes ce qui révélera le niveau d'accord non seulement sur les données annotées mais aussi sur leur perception des informations ciblées comme les circonstance factuelles qui semblent subjectives. 

\section{Perspectives du domaine}
\label{sec:conclusion:perspectives}

D'une part, le conflit entre la qualité des données annotées manuellement et la rapidité de l'automatisation encore imprécise est important. \cite{Galgani2015lexa} supportent, par exemple, qu'il est possible en un temps raisonnable d'annoter manuellement un nombre considérable de texte. Il se pose alors la question de savoir à quel point l'exhaustivité est-elle nécessaire pour contraindre les experts à supporter la marge d'erreurs infligée par les outils d'extraction automatique.

D'autre part, cette thèse est l'un des premiers travaux de recherche de cet largeur sur les décisions françaises. Ainsi, elle ouvre la voie à bien des problématiques comme l'analyse des réseaux de normes, l'anonymisation des décisions, ou l'analyse des arguments, déjà largement étudiés dans d'autres pays, notamment aux États-Unis. En cela, cette thèse encourage la recherche en analyse de donnée textuelle à s'intéresser à l'analyse automatique de la jurisprudence française dont les défis, la disponibilité d'un grand volume de données et la lucrativité du domaine judiciaire ne rendent ce champ d'application que plus attractif. Les cas d'utilisation des données extraites sont très nombreuses dans la recherche en droit, l'aide à la décision des juristes, dans l'enseignement du droit, et surtout dans l'accessibilité des profanes au droit par une estimation automatique de leurs risques judiciaires.
 % INCLUDE: conclusion
%
%%Bibliographie
\def\bibname{Bibliographie}
\addcontentsline{toc}{chapter}{\bibname}
\phantomsection
\bibliography{bib/references-pub}
%
%annexes
\appendix
\renewcommand\thesection{A.\roman{section}}
\renewcommand\thefigure{A.\arabic{figure}}
\renewcommand\thetable{A.\arabic{table}}
\chapter*{Annexes}
\addcontentsline{toc}{chapter}{Annexes}
\section{Exemple de décision judiciaire annotée}
%\chaptermark{Exemple de décision de justice}
%\addcontentsline{toc}{chapter}{Exemple de décision de justice française annotée}
\label{appendix:exemple-decision}
\lstinputlisting[language=XMl]{content/CALYO1406777.xml}

\end{document}