\chapter{Application pour l'analyse descriptive sur un grand corpus}
\label{chap:demo}
Ce chapitre décrit le processus et les résultats observés lors de l'application de la chaîne proposée (Figure \ref{fig:intro:pipeline-globale}) sur un corpus formé de la base CAP (+64k sur 1997-2018) et une base du tribunal de commerce de Paris (300k). Le module d'extraction d'information () est un système qui comprend les modèles développés et implémentés durant nos expérimentation. 

\begin{figure}[!htb]
	\centering 
	\caption{Détails du module d'extraction d'information}
\end{figure}

% répartition par ville?
 
 
 
\section{Description du Pipeline}
\label{sec:demo:motivation}

jbkj

\section{Illustration d'analyses descriptives}
\label{sec:demo:experimentations}

bkjbj


\subsection{Implémentation du système}

bkjbj;


\subsection{Données}
\subsubsection{Distribution de la base dans l'espace et dans le temps}

Données de la base CAP (64733 docs de la période 1997-2018 CA et 1er jugements) + scrapping de LegiFrance (? cour d'appel + ? 1er jugements) +  (300k Tribunal de commerce de paris)


\subsection{Analyse du sens du résultat}
;bkjkl
\subsubsection{Evolution dans le temps}
\subsubsection{Différence dans l'espace}

\subsection{Analyse des quanta}
,bkjlihio
\subsubsection{Evolution dans le temps}
\subsubsection{Différence dans l'espace}
\subsubsection{Quantum demandé vs. quantum accordé}

\section{Conclusion}
\label{sec:demo:conclusion}
hgfgh
lkhk