\chapter{Application pour l'analyse descriptive sur un grand corpus}
\label{chap:demo}
Ce chapitre décrit l'application l'enchaînement des approches proposées dans cette thèse pour extraire les données à partir d'un grand corpus jurisprudentielle. Les données extraites sont stockées dans une base de données sur laquelle peuvent être visualisées des descriptions statistiques de la jurisprudence sur les échantillons traitées.
 
 
\section{Description du Pipeline}
\label{sec:demo:motivation}

jbkj

\section{Illustration d'analyses descriptives}
\label{sec:demo:experimentations}

bkjbj


\subsection{Implémentation du système}

bkjbj;


\subsection{Données}
\subsubsection{Distribution de la base dans l'espace et dans le temps}

Données de la base CAP (64733 docs de la période 1997-2018 CA et 1er jugements) + scrapping de LegiFrance (? cour d'appel + ? 1er jugements) +  (300k Tribunal de commerce de paris)


\subsection{Analyse du sens du résultat}
;bkjkl
\subsubsection{Evolution dans le temps}
\subsubsection{Différence dans l'espace}

\subsection{Analyse des quanta}
,bkjlihio
\subsubsection{Evolution dans le temps}
\subsubsection{Différence dans l'espace}
\subsubsection{Quantum demandé vs. quantum accordé}

\section{Conclusion}
\label{sec:demo:conclusion}
hgfgh
lkhk