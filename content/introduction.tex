% !TEX root = ../thesis-example.tex
%
\chapter*{Introduction Générale}
\label{sec:intro}
\addcontentsline{toc}{chapter}{\nameref{sec:intro}}

\section{Contexte et Motivations}
\label{sec:intro:contexte}

\section{Enoncé des problèmes}
\label{sec:intro:probleme}
L'analyse automatique de décisions jurisprudentielles fait références à de nombreux problèmes rencontrés par les juriste lors de leur analyse manuelle (\textcolor{red}{par exemple?}). Dans le cadre de notre travail, nous avons définies et traités les quelques problèmes principaux décrits dans les sous-sections suivantes.
\subsection{Sectionnement des documents}
Le sectionnement des décisions de justice est important pour donner une première structure grossière aux documents. Cette structuration pourrait faciliter tout autant la lecture manuelle des documents par la localisation plus rapide des informations, et l'organisation des tâches d'extraction automatique d'information. Notre principale contribution lors du traitement de ce problème a été de réaliser une étude comparative de modèles d'étiquetage de séquence, bien établis au sein de la communauté de traitement automatique du langage naturel (TALN),  pour la reconnaissance de sections prédéfinies. 

\subsection{Annotation de métadonnées}

\subsection{Identification de données relatives aux demandes}

\subsection{Identification des circonstances factuelles}

\section{Méthodologie}
\label{sec:intro:methodologie}


\section{Résultats}
\label{sec:intro:résultats}

\section{Structure du mémoire}
\label{sec:intro:organisation}

\textbf{Chapitre \ref{sec:literature}} \\[0.2em]
%blindtext

\textbf{Chapitre \ref{sec:structuration}} \\[0.2em]
%\blindtext

\textbf{Chapitre \ref{sec:quanta}} \\[0.2em]
%\blindtext

\textbf{Chapitre \ref{sec:sensresultat}} \\[0.2em]
%\blindtext

\textbf{Chapitre \ref{sec:similarite}} \\[0.2em]
%\blindtext

\textbf{Chapitre \ref{sec:conclusion}} \\[0.2em]

\textbf{Les annexes \ref{sec:demo}} \\[0.2em]
