% !TEX root = ../thesis-example.tex
%
%\chapter{Analyse sémantique de Corpus Textuel par Traitement Automatique du Langage Naturel}
\chapter{Analyse automatique de corpus judiciaires}
\label{chap:literature}

%\cleanchapterquote{A picture is worth a thousand words. An interface is worth a thousand pictures.}{Ben Shneiderman}{(Professor for Computer Science)}

\section{Analyses pour prédiction}
\label{sec:literature:legalpredict}
%\subsection{Historique}
Le sujet de ce mémoire vise à assister l'analyse quantitative de la prise de décision judiciaire (faisant partie des comportements judiciaires). Il s'agit d'une méthode d'analyse en rupture avec les techniques juridiques traditionnelles déjà critiquées au début des années 60, par des experts de la jurisprudence tels que Karl N. Llewellyn, de souffrir d'une excéssive focalisation sur les règles juridiques qui ne représentent qu'une partie de l'institution juridique \citep{llewellyn1962jurisprudence}. Pour anticiper le comportement judiciaire, plusieurs variables plus ou moins controlables sont indispensables comme le temps, le lieu et les circonstances \citep{ulmer1963quantitative}. Dans notre cas, l'analyse quantitative d'application souhaite profiter du grand nombre de décisions de justice car il est bien connu qu'une collection suffisante d'une certaine forme de données revèle des motifs qui une fois observés sont projetables dans le futur \citep{ulmer1963quantitative}. La prédiction de la prise de décision est ainsi plus accessible que par le formalisme des lois qui ignore l'aléa judiciaire et la spécificité des cas.  Nous discutons dans cette section de quelques techniques d'automatisation de l'analyse quantitative et prédictive des comportements judiciaires, afin de décrire le contexte et justifier les concepts et la granularité judiciaire de notre étude.



%\subsection{Pourquoi une centralité des demandes?}

%\subsection{Floraison d'outils commerciaux}
%\textbf{Mise en relation avec l'avocat approprié:}
%\url{https://actoowin.com/} \url{http://www.legalup.io/} \url{https://www.legalvision.fr/}

%\textbf{calculer automatique des chances de succès d’un litige}: \url{www.predictice.com}



\section{Le Traitement Automatique du Langage Naturel Appliqué aux Décisions de Justice}
\label{sec:literature:legaltal}


\section{Extraction d'Information}
\label{sec:literature:generalie}

\section{Classification de Documents}
\label{sec:literature:classification}

\section{Catégorisation non-supervisée de documents}
\label{sec:literature:clustering}

\section{Conclusion}
\label{sec:literature:conclusion}
