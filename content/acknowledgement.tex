\chapter*{Remerciements}
\addcontentsline{toc}{chapter}{Remerciements}
\label{sec:acknowledgement}
Ces années de thèse n'ont pas  toujours été faciles. Tout au long de mes travaux de recherche et de rédaction, j'ai néanmoins reçu un précieux soutien administratif, financier, et moral de plusieurs organismes et personnes. Ce mémoire est l'occasion pour moi de tous les remercier.

En premier, je remercie mes directeurs et encadrants pour leur confiance, leur disponibilité, leurs contributions, leur effort et leur patience. Je leur prie de m'excuser pour en avoir  très souvent abusés. Merci Jacky et Stéphane pour avoir accepté de diriger mes travaux. Ça a été un honneur pour moi de vous avoir eu comme directeurs de thèse. Merci Guillaume pour avoir initié ce sujet très passionnant sur lequel j'espère avoir l'occasion de continuer à contribuer. Merci aussi pour les nombreuses journées de travail que nous avons passées ensemble. Elles m'ont été indispensables pour mieux comprendre les problèmes métiers des juristes, obtenir des données annotées, et valider mes résultats. Merci à Sébastien pour les nombreuses sessions de travail durant lesquelles nous avons formulé les problèmes métiers en problèmes informatiques, réfléchi sur des approches, et discuté des résultats expérimentaux. 

Je remercie Madame Sandra BRINGAY, Professeur de l'Université de Montpellier, et 
Monsieur Boughanem MOHAND, Professeur de l'Université Toulouse III Paul Sabatier, de m'honorer en acceptant d'être les rapporteurs de ma thèse. Je remercie aussi Madame Françoise Seyte, Maître de Conférences (HDR) de l'Université de Montpellier, et  Monsieur Fabrice MUHLENBACH, Maître de Conférences de l'Université Jean Monnet de Saint-Étienne, de me faire l'honneur d'être examinateurs de ma thèse.

Je suis reconnaissant envers l'IMT Mines Alès pour avoir financer mes travaux. Je remercie le LGI2P et l'équipe CHROME pour m'avoir accueilli. Ces remerciements s'adressent en particulier à Yannick VIMONT, ancien directeur du LGI2P, à Jacky MONTMAIN, actuel directeur du LGI2P, et à Benoît Roig, Président de l'université de Nîmes et ancien directeur de l'équipe d'accueil CHROME. Je remercie aussi Valérie Roman, Claude Badiou, Édith  Teychene, et Corinne VINCENT, pour leur aide précieuse pour les démarches administratives. 

Je remercie les chercheurs et doctorants du LGI2P et de l'équipe Chrome pour leur gentillesse et les bons moments que j'ai partagés avec eux. Tout a commencé par l'encouragement de Sylvie à postuler au LGI2P. Ensuite, c'était le quotidien avec Valentina, ma collègue de bureau. C'était les séminaires du LGI2P organisés par Christelle. C'était des discussions scientifiques avec mes collègues doctorants Pierre-Antoine, Diadie, Jean-Christophe, et bien d'autres. C'était du foot, du basket, et du ski avec Mirsad, Michel, Abdelhak, Hassan, Yannick, Sébastien, Kouadio, Brahim, Baptiste, Blazo, Alexandre, Diadie, Mouhamadou, Frank, Behrang, et bien d'autres. C'était les trajets quotidiens entre Nîmes et Alès avec Alexandre, Cécile, Behrang, Frank, Christelle et les autres. C'était des week-ends avec Clément et Julien. 

Je remercie aussi ma famille pour m'avoir soutenu et pour avoir supporté ma longue absence auprès d'eux. 

Je remercie enfin l'INRA, le CIRAD, et la société ESII pour m'avoir accordé du temps pour travailler sur mon mémoire malgré les contrats qui nous liaient.
