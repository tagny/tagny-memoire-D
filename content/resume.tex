\chapter*{Résumé}
\addcontentsline{toc}{chapter}{Résumé}
%titre
\textbf{Titre:} \textsc{\titlefr}

\textcolor{red}{la thèse n'est pas une chaine de traitement, mais un ensemble d'automatisation de tâches métiers. Pour chaque chapitre il faut préciser l'importance pour le métier, et faire la transition vers le choix de formulation et justifier ce choix. }
%Contexte
Une jurisprudence est un corpus de décisions judiciaires représentant la manière dont sont interprétées les lois pour résoudre un contentieux. Elle est utile aux juristes qui l'analysent pour comprendre et anticiper la prise de décision des juges. Son analyse exhaustive est difficile manuellement du fait de son immense volume et de la non-structuration des documents. L'estimation du risque judiciaire par des particuliers devient ainsi impossible car ils sont en outre confrontés à la complexité du système et du langage judiciaire. Une automatisation d'une telle analyse permettrait de retrouver les connaissances pertinentes pour structurer la jurisprudence à des fins d'analyses descriptives et prédictives.  
%Méthode
%Résultats
Afin de rendre la compréhension exhaustive de la jurisprudence accessible au plus grand nombre, cette thèse propose une automatisation de tâches d'importance pour l'analyse métier de la jurisprudence. Le premier, basé sur des modèles graphiques probabilistes, réalise la reconnaissance de sections et d'entités nommées juridiques de références de la décision. Ensuite, sont extraits les quanta demandés et accordés, ainsi que la polarité de la réponse des juges aux différentes demandes. L'approche proposée est basée sur la classification de documents, et la proximité entre les sommes d'argent et des termes-clés extraits automatiquement. Enfin, les décisions relatives à une catégorie de demande sont catégorisées sans supervision pour découvrir les types d'affaires appelés circonstances factuelles. A cet effet, une méthode d'apprentissage d'une distance de similarité est proposée.
%Conclusion
Cette thèse discute des résultats empiriques obtenus sur des données réelles annotées manuellement par un expert. Le mémoire est clôturé par une démonstration d'application des propositions pour l'analyse descriptive d'un grand corpus de décisions judiciaires françaises.


\textbf{Mots clés:} analyse de données textuelles, décisions jurisprudentielles, extraction d'information, classification de textes, regroupement non supervisé de documents.

