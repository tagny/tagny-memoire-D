\chapter*{Résumé}
\addcontentsline{toc}{chapter}{Résumé}
%titre
\textbf{Titre:} \textsc{\titlefr}

%\textcolor{red}{la thèse n'est pas une chaine de traitement, mais un ensemble d'automatisation de tâches métiers. Pour chaque chapitre il faut préciser l'importance pour le métier, et faire la transition vers le choix de formulation et justifier ce choix. }
%Contexte
Une jurisprudence est un corpus de décisions judiciaires représentant la manière dont sont interprétées les lois pour résoudre un contentieux. Elle est indispensable pour les juristes qui l'analysent pour comprendre et anticiper la prise de décision des juges. Son analyse exhaustive est difficile manuellement du fait de son immense volume et de la non-structuration des documents. L'estimation du risque judiciaire par des particuliers est ainsi impossible car ils sont en outre confrontés à la complexité du système et du langage judiciaire. L'automatisation permettrait de retrouver exhaustivement des connaissances pertinentes pour structurer la jurisprudence à des fins d'analyses descriptives et prédictives.  
%Méthode
%Résultats
Afin de rendre la compréhension de la jurisprudence exhaustive et plus accessible, cette thèse aborde l'automatisation de tâches d'importante  d'analyse métier de la jurisprudence. En premier, est étudiée l'application de modèles graphiques probabilistes d'étiquetage de séquences pour la détection des sections, d'entités nommées juridiques, et de citations de lois dans la décision. Ensuite, les catégories prédéfinies de demandes sont identifiées par classification de documents. Pour chaque catégorie reconnue, sont extraites les demandes des parties.  L'approche proposée pour la reconnaissance des quanta demandés et accordés exploite la proximité entre les sommes d'argent et des termes-clés extraits automatiquement. Nous montrons par ailleurs que le sens du résultat des juges est identifiable soit à partir de termes-clés prédéfinis soit par classification de documents. Enfin, les situations ou circonstances factuelles où est formulée une catégorie de demandes sont découvertes par regroupement non supervisé des décisions. A cet effet, une méthode d'apprentissage d'une distance de similarité est proposée et comparée à des distances établies.
%Conclusion
Cette thèse discute des résultats empiriques obtenus sur des données réelles annotées manuellement par un expert. Le mémoire est clôturé par une démonstration d'applications à l'analyse descriptive d'un grand corpus de décisions judiciaires françaises.
\newline
\textbf{Mots clés:} analyse de données textuelles, décisions jurisprudentielles, extraction d'information, classification de textes, regroupement non supervisé.

