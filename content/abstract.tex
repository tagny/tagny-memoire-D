%abstract
\chapter*{Abstract}
\addcontentsline{toc}{chapter}{Abstract}
%titre
\textbf{Title:} \textsc{\titleen}

A jurisprudence is a corpus of court decisions representing the way in which laws are interpreted to resolve a dispute. It is useful for lawyers who analyze it to understand and anticipate the decision-making process of judges. Its exhaustive analysis remains difficult manually because of its huge size and the non-structuring shape of documents. The estimation of the judicial risk by individuals thus becomes impossible because they are moreover confronted with the complexity of  the judicial system and language. An automation of such an analysis would enable the extraction of the knowledge relevant to structure the case law for the purpose of descriptive and predictive analyzes.
%Method
%Results
In order to make the exhaustive insight of jurisprudence accessible to as many people as possible, this thesis proposes a automation of tasks of importance for the expert analysis of the jurisprudence. The first one, based on probabilistic graphical models, realizes the recognition of sections and legal named entities of references of decision. Then, are extracted the quanta requested and granted, as well as the polarity of the response of the judges to the different requests. The proposed approach is based on the classification of documents, and the proximity between money mentions and key-phrases extracted automatically. Finally, decisions related to a category of requests are clustered to discover the types of cases known as factual circumstances. For this purpose, a method for learning a similarity distance is proposed.
%Conclusion
This thesis discusses the empirical results obtained on real data annotated manually by an expert. The thesis is closed with a demonstration of the application of the proposed pipeline concerning the descriptive analysis of a large corpus of French judicial decisions.

%keywords
\textbf{Keywords:} textual data analysis, judicial decisions, information extraction, text classification, document clustering

%legal descriptive analytics, legal textual data analytics, jurisprudencial decisions


%\textbf{Keywords}: Automation of the analysis of court decisions, information extraction, sequence labeling, structured prediction, metric learning, clustering
