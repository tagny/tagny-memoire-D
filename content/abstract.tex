%abstract
\chapter*{Abstract}
\addcontentsline{toc}{chapter}{Abstract}
%titre
\textbf{Title:} \textsc{\titleen}

A case law is a corpus of judicial decisions representing the way in which laws are interpreted to resolve a dispute. It is essential for lawyers who analyze it to understand and anticipate the decision-making of judges. Its exhaustive analysis is difficult manually because of its immense volume and the unstructured nature of the documents. The estimation of the judicial risk by individuals is thus impossible because they are also confronted with the complexity of the judicial system and language. The automation of decision analysis enable an exhaustive extraction of relevant knowledge for structuring case law for descriptive and predictive analyses. In order to make the comprehension of a case law exhaustive and more accessible, this thesis deals with the automation of some important tasks for the expert analysis of court decisions. First, we study the application of probabilistic sequence labeling models for the detection of the sections that structure court decisions, legal entities, and legal rules citations. Then, the identification of the demands of the parties is studied. The proposed approach for the recognition of the requested and granted quanta exploits the proximity between sums of money and automatically learned key-phrases. We also show that the meaning of the judges' result is identifiable either from predefined keywords or by a classification of decisions. Finally, for a given category of demands, the situations or factual circumstances in which those demands are made, are discovered by clustering the decisions. For this purpose, a method of learning a similarity distance is proposed and compared with established distances. This thesis discusses the experimental results obtained on manually annotated real data. Finally, the thesis proposes a demonstration of applications to the descriptive analysis of a large corpus of French court decisions.

%keywords
\textbf{Keywords:} textual data analysis, case law decisions, information extraction, text classification, document clustering.

%% short: (100 mots)
% A court decision is a text document containing the description of a case, the result of the judges and the reasons for this result. Lawyers use them regularly as a source of interpretation of the law and to understand the opinion of judges. The available huge quantity of decisions requires automated tools to help the actors of law. We propose to address some of the challenges related to research and analysis of the growing volume of court decisions in France. Research work mainly concerns the definition of techniques to extract relevant information from  texts and the organization of the legal corpus in order to index decisions to facilitate the predictive and descriptive analysis of jurisprudence.