%abstract
\chapter*{Abstract}
\addcontentsline{toc}{chapter}{Abstract}
%titre
\textbf{Title:} \textsc{\titleen}
\textcolor{red}{A retraduire}

A case law is a corpus of judicial decisions representing the way in which laws are interpreted to resolve a dispute. It is essential for lawyers who analyze it to understand and anticipate the decision-making of judges. Its exhaustive analysis is difficult manually because of its huge size and the non-structuring state of the documents. The estimation of the judicial risk by individuals is thus impossible because they are also confronted with the complexity of the judicial system and language. Automation can enable an exhaustive extraction of relevant knowledge for structuring case law for descriptive and predictive analysis.
In order to make the comprehension of the case-law exhaustive and more accessible, this thesis deals with the automation of important tasks of business analysis of jurisprudence. First, the application of probabilistic graphical sequence labeling models for the detection of sections, legal named entities, and legal rules citations in decisions is investigated. Then, predefined categories of requests are identified by document classification. For each recognized category, the requests of the parties are extracted. The proposed approach to the recognition of claimed and granted quanta exploits the proximity between money mentions and automatically extracted key-phrases. We also show that the meaning of the judges' result is identifiable either from predefined key terms or by classification of documents. Lastly, situations or factual circumstances in which a category of claims is formulated are discovered by clustering decisions. For this purpose, a method of learning a similarity distance is proposed and compared with established distances.
%Conclusion
This thesis discusses the empirical results obtained on real data annotated manually by an expert. The thesis is closed by a demonstration of some applications to the descriptive analysis of a large corpus of French judicial decisions.

%keywords
\textbf{Keywords:} textual data analysis, case law decisions, information extraction, text classification, document clustering