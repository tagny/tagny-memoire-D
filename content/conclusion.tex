\chapter*{Conclusions Générales}
\addcontentsline{toc}{chapter}{Conclusion}
\label{chap:conclusion}

\section{Contributions}
\label{sec:conclusion:contributions}
Introduction de nouvelles tâches d'extraction d'information motivées par des applications au monde réel.

Ouverture de la reflexion sur la nécessité ou pas de définir des approches propres NLP au domaine juridique.

Annotation manuelle de données

modélisation

expérimentation


\section{Critique du travail}
\label{sec:conclusion:critique}
Quelle représentativité ont les données utilisées dans les expérimentations




\section{Travaux futurs de recherche}
\label{sec:conclusion:extensions}

\section{Perspectives du domaine}
\label{sec:conclusion:perspectives}
Premier pas pour d'autres voies de recherche: legal / norm Citation network analysis, Anonymisation, analyse des arguments (raison influençant le sens d'un résultat, 

Cas d'utilisation: exhaustivité, rapidité, et perspectives multiples dans l'analyse des décisions, aide à la décision, assistance à l'enseignement du droit

Critiques: fiabilité des analyses descriptives (biais des données: nombre et type de documents analysés, biais d'erreur des modèles: faux négatifs (données manquées), faux positifs (données en trop), quelles marges d'erreur tolérées)
