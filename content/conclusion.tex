\chapter*{Conclusions Générales}
\addcontentsline{toc}{chapter}{Conclusion}
\label{chap:conclusion}

\section{Evaluation des Contributions}
\label{sec:conclusion:contributions}
Introduction de nouvelles tâches d'extraction d'information motivées par des applications au monde réel.

relier les obstacles rencontrés par les juristes à la qualité des résultats obtenus. (eg. la classification est assez précise pour retrouver rapidement des décisions en fonction des catégories de demande)

Ouverture de la réflexion sur la nécessité ou pas de définir des approches propres NLP au domaine juridique.

Annotation manuelle de données

modélisation

expérimentation


\section{Critique du travail}
\label{sec:conclusion:critique}
Quelle représentativité ont les données utilisées dans les expérimentations

\section{Travaux futurs de recherche}
\label{sec:conclusion:extensions}

\section{Perspectives du domaine}
\label{sec:conclusion:perspectives}

Le conflit entre la qualité des données et l'automatisation est important. \cite{Galgani2015lexa} montrent par exemple qu'il est possible en un temps raisonnable d'annoter manuellement un nombre considérable de texte. Il se pose alors la question de savoir à quel point l'exhaustivité est-elle nécessaire pour contraindre les experts à supporter la marge d'erreurs infligée par les outils d'extraction automatique.

Premier pas pour d'autres voies de recherche: legal / norm Citation network analysis, Anonymisation, analyse des arguments (raison influençant le sens d'un résultat, 

Cas d'utilisation: exhaustivité, rapidité, et perspectives multiples dans l'analyse des décisions, aide à la décision, assistance à l'enseignement du droit

Critiques: fiabilité des analyses descriptives (biais des données: nombre et type de documents analysés, biais d'erreur des modèles: faux négatifs (données manquées), faux positifs (données en trop), quelles marges d'erreur tolérées)
